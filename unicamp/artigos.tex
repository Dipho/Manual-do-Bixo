% Este arquivo .tex será incluído no arquivo .tex principal. Não é preciso
% declarar nenhum cabeçalho

\section{Acesso a artigos e revistas científicas}

Os resultados de pesquisas científicas, no Brasil e no mundo todo, costumam ser
publicados por meio de periódicos e conferências, os quais normalmente são
disponíveis pela internet.

No Brasil, quase todas as instituições públicas de ensino superior, como a
Unicamp, participam de um sistema conhecido como \textbf{Portal de Periódicos da
Capes} (\url{periodicos.capes.gov.br}), que garante acesso a grande parte das
publicações científicas das principais editoras do mundo sem necessidade de
pagar nada a mais por isso.

Nas áreas de engenharia e de computação, quase todas as publicações relevantes
são acessíveis através desse sistema. Mas é importante você saber que esse tipo
de acesso só é possível a partir de endereços IP da Universidade, então se você
quiser acessar algum artigo quando estiver em casa, o ideal é usar o sistema de
acesso VPN (Virtual Private Network) disponibilizado pela Unicamp, como pode ser
visto no site: \url{www.ccuec.unicamp.br/ccuec/acesso_remoto_vpn}.

Através da \textbf{Comunidade Acadêmica Federada (CAFe)}, foi disponibilizado
recentemente um método de acesso remoto aos periódicos sem necessidade de usar a
VPN, mais informações aqui: \url{www.ccuec.unicamp.br/biti/download/instrucoes_acesso_CAFe.pdf}

Na Unicamp, você ainda tem acesso a diversas outras publicações e e-books que
não são cobertos pelo sistema da Capes, além de alguns periódicos impressos, que
podem ser encontrados nas bibliotecas. Caso você queira buscar algo no material
que há disponível física ou virtualmente na Universidade, acesse o site do
\textbf{Sistema de Bibliotecas da Unicamp}: \url{www.sbu.unicamp.br}.

Para uma busca mais abrangente de artigos científicos na internet, você pode
usar o \textbf{Google Acadêmico} (\url{scholar.google.com}). Mas atenção! Você
pode encontrar artigos que não são cobertos pelo Portal da Capes nem pela
Unicamp e exigem pagamento.

Além do Portal de Periódicos, existe também um novo modelo de publicações
científicas de acesso gratuito, chamado \textbf{open access}. Esse modelo tem
origem muito próxima do movimento pelo software livre. Publicações feitas nesse
sistema são acessíveis a qualquer momento, de qualquer IP e sem qualquer custo.
Alguns exemplos de grandes repositórios e editoras open access são:

\begin{itemize}
    \item  \textbf{SciELO:} \url{www.scielo.org}
    \item  \textbf{PLOS:} \url{plos.org}
    \item  \textbf{arXiv:} \url{arxiv.org}
    \item  \textbf{PMC:} \url{ncbi.nlm.nih.gov/pmc}
\end{itemize}

A rede \textbf{SciELO} é onde a maior parte dos artigos em português é
publicada. O acervo \textbf{PMC} é de publicações da área biomédica.

Bixo, guarde bem esta seção do Manual! Pode ser que você não vá usá-la logo de
cara, mas quando você fizer iniciação científica ou um trabalho de disciplinas
mais avançadas, você aproveitará bastante essas informações.
