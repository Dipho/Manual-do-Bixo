% Este arquivo .tex será incluído no arquivo .tex principal. Não é preciso
% declarar nenhum cabeçalho

\section{Intercâmbio}

A Unicamp é uma das universidades brasileiras que têm maior prestígio fora do
país e a CORI-Unicamp (Coordenaria de Relações Instituicionais e
Internacionais), IC e FEEC têm vários acordos bilaterais de intercâmbio. Então,
para você que quer dar um salto em algum idioma, conhecer outras culturas e
sentir na pele a aventura de ser estrangeiro, comece a se preparar desde já.

A França hoje recebe grande número de estudantes de computação, devido a acordos
que a FEEC tem com os INSA e com as Écoles Centrales e também graças a bolsas de
estudo oferecidas pela Capes e pelo governo francês. Mas não significa que a
França seja seu destino certo: no site do CORI existem oportunidades para ir
para Estados Unidos, Japão, América Latina, Alemanha, Espanha etc., muitos com
boas bolsas de estudo ou com incentivos que valem a pena caso você tenha um
pouco de grana para se sustentar no início. Visite sempre o site da CORI e
participe das reuniões que ela faz, pois ficar ligado é a chave para conseguir
encontrar uma boa oportunidade. Existem outras opções de intercâmbio, como a
AIESEC, que promove um intercâmbio para estágios no exterior. Se seu interesse é
mais profissional, procure se informar.

Fique de olho também no programa Ciência sem Fronteiras, do governo federal, que
desde o ano passado distribui bolsas para instituições de excelência dos Estados
Unidos, Alemanha, Itália, Reino Unido, entre outros.

``Mas vale a pena? Poxa, vou atrasar meu curso, ficarei deslocado de turma, vou
ficar em um país estranho, para quê? Acho que não vale a pena{\dots}''

Vamos começar pelos motivos profissionais: ter no currículo que você fala uma
língua estrangeira fluentemente devido à sua imersão no país é algo muito
valorizado pelas empresas, além do fato que o pessoal do RH vai ver que você tem
capacidade de se virar sozinho, uma vez que não é tão óbvio sair do país e
recomeçar sua vida fora. Você não atrasará tanto seu curso, pois a Unicamp conta
com um sistema de equivalências de matérias, e se você escolher bem pode fazer
matérias que serão convalidadas na Unicamp. Agora, o que realmente é importante:
você está na faculdade, está na hora de deixar o colo da mamãe e partir para o
mundo! A experiência de conhecer outras culturas, criar laços de amizade
internacionais, viajar por terras desconhecidas não tem preço! Pense que é no
seu tempo de facul que terá oportunidade de fazer uma aventura destas, não
desperdice.

Se você abriu um sorriso e pensa que está preparado para sair do país, comece a
estudar, bixo! Não que ter uma boa nota seja a única forma de conseguir uma vaga
em uma bolsa de estudos, mas com certeza é a mais fácil.  Busque atirar em todas
as frentes, mantenha seu CR num bom nível, busque conhecer organismos como
AIESEC e procure grupos de trabalho (no IC existem vários) que podem levar
alunos ao exterior. Boa sorte!

Se você realmente se interessou, aí vão uns links com mais informações:

\begin{itemize}
    \item  CORI: \url{www.cori.unicamp.br}
    \item  AIESEC: \url{aiesec.org.br}
    \item  Ciência sem Fronteiras: \url{cienciasemfronteiras.gov.br}
    \item  Site da Pró-reitoria de graduação sobre o CsF: \url{www.prg.unicamp.br/csf}
\end{itemize}

Fique atento aos e-mails que você receberá do IC e da FEEC. Muitos deles são
sobre programas de intercâmbio.
