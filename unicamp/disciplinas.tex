% Este arquivo .tex será incluído no arquivo .tex principal. Não é preciso
% declarar nenhum cabeçalho

\section{Disciplinas}

\subsection{Matrícula}

A Unicamp é muito diferente da sua escolinha onde a tia Gertrudes entregava o
seu horário impresso coloridinho para você colar na capa do seu fichário.  À
exceção do primeiro semestre letivo, no qual você já entra matriculado em todas
as matérias obrigatórias, na Unicamp você vai ter que se virar.  O GDE
(\url{gde.ir}) é uma ferramenta criada por um aluno da Engenharia e adotada pela
DAC que facilita muito o planejamento do seu horário, além de servir como uma
rede social interna.

Peça sempre a ajuda dos seus veteranos quando for montar seu horário. Informe-se
sobre todos os professores que oferecem as matérias, se eles são coxas ou
carrascos, bons ou ruins, se demoram para entregar as notas{\dots} Você vai
poupar muita dor de cabeça. O melhor lugar para essas discussões é o grupo de
e-mail da sua turma.  Pode ter certeza que ela terá um assim que você entrar na
Unicamp.

\subsection{Cancelamento, Trancamento e Desistência}

Embora praticamente todos os alunos da Unicamp usem esses três termos
indiscriminadamente, como se fossem sinônimos, para a DAC, esses três termos têm
significados bastante distintos. Aí vai o que cada termo significa:

\begin{itemize}
    \item Desistência de matrícula em disciplinas
        (\url{www.dac.unicamp.br/portal/grad/regimento/capitulo_iii/secao_v}):
        Processo que é chamado pelos alunos de ``trancamento''.  O aluno não
        mais cursa essa disciplina no semestre, tendo de cursá-la em algum
        semestre posterior. Só é possível desistir uma vez da disciplina e
        pode-se pedir desistência até que se tenha passado metade do semestre.
    \item Cancelamento de matrícula
        (\url{www.dac.unicamp.br/portal/grad/regimento/capitulo_iii/secao_vii}):
        Processo em que o aluno se desliga da Unicamp, por motivo de jubilação,
        por ter faltado às duas primeiras semanas do ano de ingresso, por ter
        sido reprovado em todas as disciplinas do primeiro ou do segundo
        semestre de ingresso, por ter sido expulso, por ter sido aprovado em
        outra universidade pública (não é permitido fazer mais do que um curso
        de universidade pública simultaneamente), ou por vontade própria do
        aluno.
    \item Trancamento de matrícula
        (\url{www.dac.unicamp.brportal/grad/regimento/capitulo_iii/secao_vi}):
        Processo em que o aluno não cursa qualquer disciplina da Unicamp durante
        o semestre. O aluno tem direito a fazer até dois trancamentos de
        matrícula, em semestres seguidos ou não e o aluno não pode trancar
        nenhum dos dois semestres do ano de ingresso. Desistência de todas as
        disciplinas configura-se como trancamento. O trancamento é pedido na
        DAC, e pode ser pedido até que se tenha transcorrido 2/3 do semestre
        (geralmente de dezembro até fim de maio para trancamento de primeiro
        semestre; e de julho até fim de outubro para trancamento de segundo
        semestre). Para cada trancamento, o prazo máximo de integralização é
        postergado.
\end{itemize}

\subsection{Eletivas e Teste de Proficiência}

A Unicamp oferece a seus alunos a oportunidade de personalizar seu currículo de
acordo com seu interesse por meio das \textbf{disciplinas eletivas}. Ao
contrário das disciplinas obrigatórias, com as eletivas você pode escolher a
matéria que vai cursar. Alguns créditos podem ser cumpridos com qualquer
disciplina oferecida pela Universidade, outros estão restritos a um determinado
conjunto. Para mais detalhes, consulte seu catálogo em
\url{www.ic.unicamp.br/graduacao/catalogos-de-graduacao}.

Mas não se esqueça de que com um grande poder vem uma grande responsabilidade!
A Unicamp lhe dá liberdade para escolher o melhor jeito de se preparar para seu
futuro, e espera que você saiba o que fazer com essa liberdade. Você pode
socializar com outros cursos, aprender uma língua estrangeira, assistir a
seminários ou obter um certificado de estudos na FEEC ou no IC.

\textbf{Teste de proficiência} é uma prova que permite dispensa de cursar uma
disciplina (desde que você obtenha a nota mínima, é claro). Se você acha que
sabe o suficiente sobre eletromagnetismo, por exemplo, pode tentar a
proficiência de Física Geral III.  Nem todas as disciplinas oferecem o teste, e
você só pode fazê-lo uma vez por disciplina -- e se você já se matriculou na
disciplina e não passou, não pode fazer.  Além disso, fazer o teste de
proficiência também é obrigatório para se matricular nas disciplinas de língua
inglesa e japonesa, independentemente de conhecimento prévio na língua.

Fique ligado no calendário da DAC para não perder as datas de inscrição nos
testes de proficiência! As datas dos testes de línguas são sempre no começo do
ano, diferentes das demais, que são no fim de cada semestre.

Disciplinas eletivas e teste de proficiência estão relacionados porque muitas
pessoas, especialmente nos cursos de computação, fazem proficiência em
disciplinas de línguas, eliminando créditos de eletivas, em alguns casos para
evitar o jubilamento, outros para não ter que passar mais um semestre na
faculdade. Apesar de registrar a familiaridade do aluno com uma outra língua em
sua integralização, essa prática não enriquece a graduação de nenhum estudante
que faça tal escolha. Além disso, muitos bixos arrependem-se de terem feito a
prova e perdido preferência na hora de pegar uma disciplina interessante,
podendo até mesmo não conseguir se matricular. Isso acontece porque, depois que
seus créditos de eletivas se esgotam, você começa a puxar matérias
não-obrigatórias como \textbf{extracurriculares} e tem prioridade menor na
atribuição de vagas.

Converse com seus amigos e veteranos para descobrir o melhor jeito de usufruir
dessa liberdade que poucas universidades oferecem! Dificilmente você não
encontrará algo com o qual se identifica ou que não ensine lições interessantes.

Para mais informações sobre teste de proficiência, acesse:
\url{www.dac.unicamp.br/portal/grad/avaliacao_e_frequencia/teste_de_proficiencia}.

\subsection{Avaliações de professores}

Achou que o professor ensinou muito mal? Ele falou da vida, do universo e tudo
mais -- menos sobre a disciplina? Foi incoerente? Ou, pelo contrário, achou o
professor o máximo e a sala do CB a oitava maravilha do mundo? Não adianta
xingar nem elogiar no Twitter!

Nas últimas aulas de cada semestre, todos os professores devem disponibilizar um
formulário de avaliação. Esse é o momento para que você possa separar os acertos
dos erros, portanto preencha com seriedade. Os dados serão analisados pelas
Comissões de Graduação de cada unidade e os comentários escritos serão
repassados para o professor.

Além dos formulários, a PRG (Pró-reitoria de Graduação) realiza o Programa de
Avaliação da Graduação no fim de cada semestre. Trata-se de uma pesquisa on-line
semelhante aos formulários de cada unidade, porém unificada para toda a Unicamp
e mais abrangente em suas perguntas.

O GDE (\url{gde.ir}) também tem um sistema de avaliação de professores, cuja
nota costuma ser usada pelos alunos como um dos critérios no momento de decidir
com que professor puxar uma matéria.

Durante o semestre, ocorre a Reunião de Avaliação de Curso. A data e o horário
serão divulgados pelas unidades e pelo CACo. Essa é uma oportunidade de passar
para as coordenadorias do curso não só suas impressões sobre professores e
disciplinas, mas sobre qualquer assunto relacionado ao curso. Antes da Reunião,
o CACo também promove um PipoCACo de Avaliação de Curso, motivando uma
pré-discussão.

Tenha sempre em mente que a nossa percepção sobre o oferecimento de uma
disciplina não é óbvia para os professores. Preencha todas as avaliações com
sinceridade e use sempre os espaços dedicados a comentários. Além disso, cultive
o hábito de realizar uma avaliação informal do professor no fim de cada semestre
-- mandando um e-mail, por exemplo. O valor deste tipo de avaliação é muito
grande.
