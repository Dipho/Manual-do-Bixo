% Este arquivo tex vai ser incluído no arquivo tex principal, não pe preciso
% declarar nenhum cabeçalho

\section{Bem-vindo a um mundo novo!}

Olá, bixo! Agora que você está na faculdade verá que aquele estereótipo de nerd
não lhe garante boas notas e que o lugar onde você mais vai aprender não
é a sala de aula. É verdade, calouro, a faculdade é diferente, quase tudo
dependerá de você, desde quais matérias quer cursar até quando vai estudar para
as provas. Portanto, leia isto com atenção.

Os professores (excetuando o Ary do IMECC, o José Mario da FEEC e o Cid (e alguns outros) do IC)
dificilmente saberão seu nome ou quem é você; provavelmente, eles somente verão
no final do semestre se o RA 127654  passou ou não passou. Mas não se sinta
desamparado, afinal, você também vai logo esquecer a matéria dele! O quanto você
estuda não é diretamente proporcional à sua nota, portanto aprenda a estudar
melhor. Atividades em grupo, resolução de exercícios e monitorias são boas
(ótimas. Aliás, excelentes) pedidas.

Cada graduando tem uma cultura própria (não se esqueça que tem gente de todo
o Brasil na UNICAMP), portanto não se surpreenda com pessoas que te chamam de
"pessoa", "piá", "meu", "mano", "jovem", "moleque", "guri", "moço", "rapaz",
"brou". Essas pessoas são bem legais quando você se acostuma com esta
diversidade. Aliás, diversidade é algo bem marcante na UNICAMP: Veja os vários
grupos que existem pelo campus, mas não se deixe levar pelas aparências, afinal
aquele cara que mal fala "paraboloide hiperbólico" pode ser o cara mais crânio
do curso todo.

O campus é esta doideira toda mesmo, você pode estar andando e se perguntando
como as TV's sabiam para onde você apontava a pistola do Dynavision (ou do
Master System) e cruzar com alguém tocando gaita de fole, alguém parecendo uma
estátua em plena praça, ou então um grupo na grama lendo a Bíblia, treinando
Wushu ou Taijiquan. Não se sinta sozinho, afinal todos seus colegas estão assim
também e depois de um mês você estará adorando isto e perceberá porque todos
dizem que a época de faculdade é a melhor da vida.

Este manual foi organizado pelo CACo (Centro Acadêmico da Computação -- o SEU
CA), com apoio da AAACEC (a Valorosa Atlética) e escrito por diversos veteranos
seus para ajudá-lo nesse começo de vida Universitária! Onde comer? Onde estudar?
Onde morar? Tudo isso são dúvidas comuns, que aqui tentamos ajudar a resolver.
Não há respostas prontas, cada um tem suas preferências, mas a gente dá uma mão.

O que é CA? E Atlética? E DCE? E Bandejão? E Moradia? E essas siglas e códigos
malucos? Como eu faço para pegar uma bolsa? A gente também tenta responder todas
essas perguntas. E também damos algumas dicas de onde comprar coisas, onde se
divertir e alguns telefones úteis!

Parabéns pela aprovação!! Seja bem-vindo e aproveite a vida acadêmica!!
