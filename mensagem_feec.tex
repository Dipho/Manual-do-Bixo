% Este arquivo tex vai ser incluído no arquivo tex principal, não pe preciso
% declarar nenhum cabeçalho

\section{Mensagem do Diretor da FEEC}

Mensagem de Boas Vindas aos Novos Alunos de Engenharia de Computação

Campinas, 3 de janeiro de 2011.

Prezados novos alunos e alunas de Engenharia de Computação:

Nas próximas semanas vocês estarão iniciando um curso compartilhado entre
o Instituto de Computação (IC) e a Faculdade de Engenharia Elétrica e de
Computação (FEEC). O Prof. Hans Liesenberg, Diretor do IC, estará também se
dirigindo a vocês em mensagem publicada neste Guia. Em nome da comunidade da
FEEC, venho lhes apresentar as boas vindas à nossa escola e a essa importante
etapa que iniciam. O início das aulas e a chegada dos novos alunos marcam um
período alegre e animado da vida universitária. Há uma agradável sensação de
novidade no ar e um renovado entusiasmo que contagia a todos. Espero que estes
primeiros meses na Unicamp possam ser uma boa transição para a nova rotina de
atividades e um período efetivo para cultivar novas amizades.

Desde 2008, a FEEC mantém um programa de mentorias para os alunos ingressantes
de Engenharia Elétrica e de Engenharia de Computação. Cada novo aluno recebe
a indicação de um mentor entre os docentes participantes do IC e da FEEC.
O papel de um mentor neste programa é o de alguém com uma boa experiência
universitária que pode contribuir com outros pontos de vista em assuntos do
currículo, da profissão e da vida em geral. Recomendo que procurem estabelecer
contato com seus mentores na primeira oportunidade que tiverem. Em abril,
planejamos organizar um evento de congraçamento entre alunos e mentores, para
facilitar os contatos.

Uma característica importante dos cursos da FEEC é a participação expressiva de
matérias de laboratório nos currículos. Isto contribui para a formação de
engenheiros com boa experiência de bancada e visão dos aspectos práticos da
profissão. Ademais, nossos cursos tem uma natureza generalista, que reforça
o aprendizado dos fundamentos das várias vertentes de Engenharia Elétrica ou de
Computação. Embora seja uma característica tradicional de nossos cursos, esta
é também uma moderna tendência internacional, voltada para a formação de
engenheiros com boas condições de adaptação às frequentes mudanças nos cenários
da profissão, a novas tecnologias, novas aplicações e modelos de negócio. Mais
do que preparar o engenheiro para um bom emprego, os cursos se dedicam à sua
preparação para uma longa e produtiva carreira.

Em relação à empregabilidade da área, as engenharias seguem como uma das áreas
de maior oferta de empregos. No Brasil, a demanda por engenheiros é ainda bem
maior que o número de engenheiros que aqui se formam. A Unicamp contribui
expressivamente para esta demanda com profissionais de nível internacional.
Segundo importantes agências de classificação, a Unicamp se encontra entre as
300 melhores universidades do mundo e entre as duas melhores do Brasil (cf.
www.topuniversities.com). Nossos cursos de Engenharia Elétrica e de Computação
tem recebido a cada ano, cinco estrelas nas avaliações do Guia do Estudante da
Editora Abril. Estes são alguns dos aspectos que valorizam sua opção pelo curso
que ora iniciam na Unicamp.

Saliento ainda, que a FEEC dispõe de um número significativo de entidades
estudantis. São 14 entidades entre centros acadêmicos, empresas juniores, ligas
atléticas, ramos de institutos internacionais, grupos de estudos específicos, de
promoção de oportunidades de trabalho e do Trote da Cidadania. Algumas dessas
entidades são exclusivas da FEEC e outras são compartilhadas com outras unidades
da Unicamp. Convido-os a se inteirar dos trabalhos dessas entidades
e a participar de suas atividades.

Meus votos são de que possam aproveitar intensamente a adaptação à vida
universitária e que este seja o início de uma inesquecível jornada, que os leve
a uma sólida formação como profissionais de engenharia. Estou à disposição de
vocês para conversar sobre os assuntos que desejarem, em minha sala ou através
do e-mail max@fee.unicamp.br.

Cordiais saudações,

Max Costa Diretor
FEEC -- Unicamp
