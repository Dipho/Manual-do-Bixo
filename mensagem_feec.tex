% Este arquivo tex vai ser incluído no arquivo tex principal, não pe preciso
% declarar nenhum cabeçalho

\section{Mensagem do Diretor da FEEC}

Caros calouros de Engenharia de Computação,

É com grande satisfação que os recebemos na Unicamp e, em especial, na FEEC.
Nossa faculdade traz em sua denominação duas Engenharias de imensa importância:
Engenharia de Computação e Engenharia Elétrica.  O mundo moderno está assentado
sobre alguns pilares, dois destes são a Informação e a Energia.  E é exatamente
sobre nesses temas que a FEEC realiza suas atividades de ensino, pesquisa e
extensão há mais de 40 anos.  Ter os melhores cursos do Brasil é uma grande
responsabilidade. temos clareza de que isso foi construído por, principalmente,
contarmos com os melhores estudantes, ao que se associam excelentes professores
e uma infraestrutura que procura se manter renovada e atualizada.  Como vocês
experimentarão, Engenharia de Computação é um curso exigente, que requer
empenho e espera desempenho dos estudantes. Em contrapartida, vocês terão
oportunidade de participar da grande aventura do conhecimento, com descobertas,
talvez invenções, certamente com método.  Mas a Universidade é muito mais do
que o curso no qual vocês estão ingressando. É um mundo em si, com todas as
áreas do conhecimento, com convivência com pessoas dos mais diferentes lugares,
com arte e cultura, com debates e comemorações.  Aproveitem tudo isso, ao
máximo, sem esquecer que vocês estão em uma escola pública e gratuita, que sua
permanência aqui se deve ao trabalho de milhões de cidadãos do Brasil e, em
particular, de São Paulo. Honrem o trabalho de todas essas pessoas, dando o
melhor de si, com ética e dedicação, seja durante o curso, seja daqui a pouco,
quando já forem profissionais.  Bem-vindos à FEEC e a diretoria e as
coordenações de curso estarão sempre à disposição apra ajudar no que for
possível.


José Antenor Pomilio - Diretor da FEEC
