\documentclass[a4paper,11pt, twocolumn]{article}
\usepackage{fullpage}
%\usepackage[scaled=0.85]{helvet}
\renewcommand{\familydefault}{\sfdefault}
\usepackage{ucs}
\usepackage[utf8x]{inputenc}
\usepackage[brazil]{babel}
\usepackage[T1]{fontenc}
\usepackage{lmodern}
\usepackage[pdftex]{graphicx}
\usepackage{url,color,float}
\usepackage{multirow}
\usepackage{hyperref}
\hypersetup{
    colorlinks=false,
    pdfborder=0 0 0
}
\usepackage{tabularx}
\newcolumntype{Y}{>{\raggedright}X}


\begin{document}

Bem vindo!

Este é o Manual do Bixo do CACo, o Centro Acadêmico da Computação.

Ele foi elaborado por seus veteranos e inclui diversas dicas para auxiliar a sua
sobrevivência neste primeiro ano na Unicamp =).

Você provavelmente recebeu a versão impressa nas primeiras semanas de aula(se
não, passe na salinha do CACo/AAACEC para pegar).

Entretanto, aqui está a wiki utilizada para escrever o texto, aqui
disponibilizada para facilitar a consulta.

Aproveite!

\tableofcontents

\twocolumn
\section{Bem-vindo a um mundo novo!}
Olá, bixo! Agora que você está na faculdade verá que aquele estereótipo de nerd
não lhe garante boas notas e que o lugar onde você mais vai aprender não
é a sala de aula. É verdade, calouro, a faculdade é diferente, quase tudo
dependerá de você, desde quais matérias quer cursar até quando vai estudar para
as provas. Portanto, leia isto com atenção.

Os professores (excetuando o Ary do IMECC, o José Mario da FEEC e o Cid (e alguns outros) do IC)
dificilmente saberão seu nome ou quem é você; provavelmente, eles somente verão
no final do semestre se o RA 117654  passou ou não passou. Mas não se sinta
desamparado, afinal, você também vai logo esquecer a matéria dele! O quanto você
estuda não é diretamente proporcional à sua nota, portanto aprenda a estudar
melhor. Atividades em grupo, resolução de exercícios e monitorias são boas
(ótimas. Aliás, excelentes) pedidas.

Cada graduando tem uma cultura própria (não se esqueça que tem gente de todo
o Brasil na UNICAMP), portanto não se surpreenda com pessoas que te chamam de
"pessoa", "piá", "meu", "mano", "jovem", "moleque", "guri", "moço", "rapaz",
"brou". Essas pessoas são bem legais quando você se acostuma com esta
diversidade. Aliás, diversidade é algo bem marcante na UNICAMP: Veja os vários
grupos que existem pelo campus, mas não se deixe levar pelas aparências, afinal
aquele cara que mal fala "paraboloide hiperbólico" pode ser o cara mais crânio
do curso todo.

O campus é esta doideira toda mesmo, você pode estar andando e se perguntando
como as TV's sabiam para onde você apontava a pistola do Dynavision (ou do
Master System) e cruzar com alguém tocando gaita de fole, alguém parecendo uma
estátua em plena praça, ou então um grupo na grama lendo a Bíblia, treinando
Wushu ou Taijiquan. Não se sinta sozinho, afinal todos seus colegas estão assim
também e depois de um mês você estará adorando isto e perceberá porque todos
dizem que a época de faculdade é a melhor da vida.

Este manual foi organizado pelo CACo (Centro Acadêmico da Computação -- o SEU
CA), com apoio da AAACEC (a Valorosa Atlética) e escrito por diversos veteranos
seus para ajudá-lo nesse começo de vida Universitária! Onde comer? Onde estudar?
Onde morar? Tudo isso são dúvidas comuns, que aqui tentamos ajudar a resolver.
Não há respostas prontas, cada um tem suas preferências, mas a gente dá uma mão.

O que é CA? E Atlética? E DCE? E Bandejão? E Moradia? E essas siglas e códigos
malucos? Como eu faço para pegar uma bolsa? A gente também tenta responder todas
essas perguntas. E também damos algumas dicas de onde comprar coisas, onde se
divertir e alguns telefones úteis!

Parabéns pela aprovação!! Seja bem-vindo e aproveite a vida acadêmica!!

\section{Mensagem do Diretor da FEEC}
Mensagem de Boas Vindas aos Novos Alunos de Engenharia de Computação

Campinas, 3 de janeiro de 2011.

Prezados novos alunos e alunas de Engenharia de Computação:

Nas próximas semanas vocês estarão iniciando um curso compartilhado entre
o Instituto de Computação (IC) e a Faculdade de Engenharia Elétrica e de
Computação (FEEC). O Prof. Hans Liesenberg, Diretor do IC, estará também se
dirigindo a vocês em mensagem publicada neste Guia. Em nome da comunidade da
FEEC, venho lhes apresentar as boas vindas à nossa escola e a essa importante
etapa que iniciam. O início das aulas e a chegada dos novos alunos marcam um
período alegre e animado da vida universitária. Há uma agradável sensação de
novidade no ar e um renovado entusiasmo que contagia a todos. Espero que estes
primeiros meses na Unicamp possam ser uma boa transição para a nova rotina de
atividades e um período efetivo para cultivar novas amizades.

Desde 2008, a FEEC mantém um programa de mentorias para os alunos ingressantes
de Engenharia Elétrica e de Engenharia de Computação. Cada novo aluno recebe
a indicação de um mentor entre os docentes participantes do IC e da FEEC.
O papel de um mentor neste programa é o de alguém com uma boa experiência
universitária que pode contribuir com outros pontos de vista em assuntos do
currículo, da profissão e da vida em geral. Recomendo que procurem estabelecer
contato com seus mentores na primeira oportunidade que tiverem. Em abril,
planejamos organizar um evento de congraçamento entre alunos e mentores, para
facilitar os contatos.

Uma característica importante dos cursos da FEEC é a participação expressiva de
matérias de laboratório nos currículos. Isto contribui para a formação de
engenheiros com boa experiência de bancada e visão dos aspectos práticos da
profissão. Ademais, nossos cursos tem uma natureza generalista, que reforça
o aprendizado dos fundamentos das várias vertentes de Engenharia Elétrica ou de
Computação. Embora seja uma característica tradicional de nossos cursos, esta
é também uma moderna tendência internacional, voltada para a formação de
engenheiros com boas condições de adaptação às frequentes mudanças nos cenários
da profissão, a novas tecnologias, novas aplicações e modelos de negócio. Mais
do que preparar o engenheiro para um bom emprego, os cursos se dedicam à sua
preparação para uma longa e produtiva carreira.

Em relação à empregabilidade da área, as engenharias seguem como uma das áreas
de maior oferta de empregos. No Brasil, a demanda por engenheiros é ainda bem
maior que o número de engenheiros que aqui se formam. A Unicamp contribui
expressivamente para esta demanda com profissionais de nível internacional.
Segundo importantes agências de classificação, a Unicamp se encontra entre as
300 melhores universidades do mundo e entre as duas melhores do Brasil (cf.
www.topuniversities.com). Nossos cursos de Engenharia Elétrica e de Computação
tem recebido a cada ano, cinco estrelas nas avaliações do Guia do Estudante da
Editora Abril. Estes são alguns dos aspectos que valorizam sua opção pelo curso
que ora iniciam na Unicamp.

Saliento ainda, que a FEEC dispõe de um número significativo de entidades
estudantis. São 14 entidades entre centros acadêmicos, empresas juniores, ligas
atléticas, ramos de institutos internacionais, grupos de estudos específicos, de
promoção de oportunidades de trabalho e do Trote da Cidadania. Algumas dessas
entidades são exclusivas da FEEC e outras são compartilhadas com outras unidades
da Unicamp. Convido-os a se inteirar dos trabalhos dessas entidades
e a participar de suas atividades.

Meus votos são de que possam aproveitar intensamente a adaptação à vida
universitária e que este seja o início de uma inesquecível jornada, que os leve
a uma sólida formação como profissionais de engenharia. Estou à disposição de
vocês para conversar sobre os assuntos que desejarem, em minha sala ou através
do e-mail max@fee.unicamp.br.

Cordiais saudações,

Max Costa Diretor
FEEC -- Unicamp

\section{Mensagem do Diretor do IC}
Prezado ingressante,

Parabéns pela conquista de uma vaga em uma das mais renomadas universidades do
país e benvindo à vida universitária!

O Instituto de Computação (IC) será uma de suas referências importantes nos
próximos cinco anos. O IC já possui uma “longa vida” na história da Unicamp. Em
1968, o Prof. Rubens Murillo Marques, então Diretor do IMECC, propôs a criação
de um curso de Computação na Unicamp. “Ciência da Computação? Que negócio
é esse?” perguntou o Prof. Zeferino Vaz, Reitor desde os primórdios da Unicamp
até a sua aposentadoria em 1978, ao que o Prof. Murillo respondeu, de forma
sucinta: “É o futuro, Zeferino!”.

Em 1969 foi criado o Curso de Bacharelado em Ciência da Computação juntamente
com o Departamento de Ciência da Computação (DCC) no então Instituto de
Matemática, Estatística e Ciência da Computação (IMECC). Ao longo dos anos,
o DCC foi se fortalecendo. Juntamente com tal fortalecimento também começou
a nascer e amadurecer o anseio do DCC em se tornar uma unidade independente.
A saída do IMECC foi negociada em todas as instâncias decisórias da Unicamp e,
finalmente, em 1996, o Instituto foi formalmente criado.

Você agora faz parte do “futuro” vislumbrado pelo Prof. Murillo, da trajetória
do IC e da história da Unicamp! O IC tem por objetivo formar lideranças na área
de Computação. A preparação de bons profissionais, contudo, requer muito estudo
e trabalho. Aulas bem como exercícios e projetos são recursos para auxiliá-lo na
reconstrução do conhecimento acumulado na área nas últimas décadas e em franca
expansão. A aquisição do conhecimento se dá através da transformação de
informações fornecidas em sala, disponibilizadas nas bibliotecas da Unicamp e na
internet bem como através da construção de artefatos, como código e modelos, que
visam uma melhor fixação de novos conhecimentos.

O sistema universitário difere daquilo que você já viveu no Ensino Médio. A sua
vida acadêmica não é planejada em grande detalhe e acompanhada de perto.
É delegado ao aluno a responsabilidade de gerenciar o seu tempo e organizar as
suas atividades estudantis. No início, tudo parece ser fácil e aparentemente as
cobranças são poucas. Ledo engano! As cobranças, quando ocorrem, são duras. Você
será tratado como adulto e, como tal, as suas responsabilidades são muito
maiores do que na sua vida pregressa. Ninguém vai correr atrás de você para
cobrar estudo ou a realização de atividades extra-classe. A falta de compreensão
dessas discrepâncias entre o Ensino Superior e o Ensino Médio resulta em muitas
reprovações no primeiro semestre, um número ainda significativo no segundo
e decrescendo conforme os alunos vão entendendo melhor como as “coisas
funcionam”.

A recomendação que posso dar a você para se tornar um profissional de primeira
linha é que você mantenha os estudos em dia e procure ajuda tão logo as
dificuldades apareçam. Não deixe as dúvidas se avolumarem, pois elas aumentam de
forma vertiginosa. A Unicamp provê muitos mecanismos de apoio tais como
monitoria e atendimento docente. Você também pode e deve procurar o Coordenador
de seu curso se você não conseguir resolver os seus problemas acadêmicos ou não
junto a monitores e docentes. Em última instância, procure a Diretoria do
Instituto.

O IC quer formar com qualidade o maior número possível de profissionais, mas
você precisa fazer a sua parte.  A Unicamp não se resume só ao IC. Relacione-se
com o maior número de pessoas de áreas diferentes para, assim, aumentar o seu
“capital social”: participe de atividades interdisciplinares e de extensão
universitária, atue nas associações estudantis e nos diferentes colegiados que
definem os rumos da Universidade. A teia de relacionamentos que você estabelecer
durante a sua passagem na Unicamp será de grande valia e poderá propulsionar
muito sua futura carreira profissional.

Bem-vindo, então, à comunidade do IC e espero que você viva uma rica
e proveitosa experiência acadêmica!

Hans Liesenberg

Diretor do IC

\section{Para pensar}
O objetivo desse tópico não é responder grandes questões, nem doutrinar ninguém,
apenas expor alguns assuntos que achamos importante que você, bixo, analise
neste momento em que você está iniciando sua vida na universidade.

\subsection{UNICAMP, uma Universidade Pública?} À primeira vista, um título como
esse pode parecer estranho. "Mas como? Acabei de entrar na UNICAMP, uma
universidade estadual e não vou pagar mensalidades para estudar nela. É óbvio
que ela é uma Universidade Pública". Será?

Público, no sentido do dicionário, refere-se a "todos", ao "povo", mas também no
sentido popular quando nos referimos a algo público, logo lembramos de acesso
irrestrito e gratuito, sem distinção alguma a todos os cidadãos e cidadãs, um
lugar que todos tem direitos de usufruir. Mais ainda, público significa que
todos pagam, que é gerido e que foi construído com o dinheiro de todas as
pessoas através dos impostos arrecadados pelos mecanismos do Estado. É isso que
acontece por exemplo com os hospitais, creches e escolas públicas que, embora
possuam atendimento muitas vezes precário, atendem a todos.

Mas e uma universidade, de que forma ela se encaixa dentro do "público"?

Para responder melhor essa questão, devemos analisar a universidade com base em
seu tripé essencial: ensino, pesquisa e extensão. Comecemos por analisar
o acesso ao ensino. Por um lado, o processo seletivo da universidades públicas
paulistas, o vestibular, é uma forma virtualmente imparcial de seleção, que deve
analisar apenas o nível de conhecimento do candidato (que muitas vezes está
ligado ao valor de sua renda), sem levar em conta a religião, o sexo, ou
a classe social do vestibulando. Logo, exceto por pontos extras no vestibular
(para negros, índios e estudantes de escola pública), para ingressar em uma
universidade como a UNICAMP não há nenhuma distinção entre os cidadãos. Por
outro lado, só no Estado de São Paulo há cerca de 2 milhões de jovens com idade
para estar na universidade, enquanto isso as três universidades estaduais
paulistas (USP, UNESP e UNICAMP) não chegam a oferecer 100 mil vagas de
graduação e pós. Portanto, nem 5\% dos jovens paulistas tem acesso ao ensino
público superior, sendo que a grande maioria dos ingressantes na universidade
pública pertence às classes média ou alta. Existem iniciativas pontuais que
buscam mudar este quadro: Cursinhos comunitários (como a Cooperativa do Saber,
Cursinho do Sindicato, dentre outros) tem obtido um bom nível de aprovação de
seus alunos, porém em cursos com baixa concorrência, sendo que cursos como
medicina e computação continuam praticamente restritos às classe mais altas.

Agora, tendo em mente a pesquisa, podemos dizer que a UNICAMP se destaca nesse
ponto: Ela é responsável por 15\% de toda a pesquisa brasileira, desenvolve
vários estudos sobre a sociedade brasileira (tendo publicado recentemente um
atlas social), projetou equipamentos de segurança para carros, está trabalhando
em sistemas computacionais para saúde, entre tantas outras pesquisas voltadas
para o progresso da ciência e tecnologia nacional. Isso, sem contar que
a UNICAMP foi responsável por 35\% das patentes registradas na década de 90 aqui
no Brasil. Mas grande parte dessas patentes são vendidas a grandes empresas,
muitas vezes multinacionais. Outro ponto crítico existente são as pesquisas
particulares desenvolvidas na UNICAMP (a sala .NET, no IC, é um exemplo disso,
onde as patentes são divididas entre universidade, pesquisador e Microsoft).

Finalmente, falaremos sobre a extensão. Na nossa universidade o principal
projeto de extensão é a administração de hospitais da região, como o HC, que
é o maior hospital público do interior paulista, e atende pessoas de toda
a região e até de fora do Estado. Fora ele, a universidade não desenvolve nenhum
projeto de extensão gratuito que tenha grande destaque. No Instituto de
Computação, por exemplo, os cursos de extensão custam mais de R\$ 2000,00 por
aluno, o que não pode realmente ser chamado de extensão universitária, uma vez
que não está distribuindo à comunidade o conhecimento produzido aqui dentro.
Além disso a UNICAMP também restringe o acesso a diversos espaços da
universidade, como por exemplo o controle do acesso noturno ao campus,
a coibição de festas, e a dificuldade de acesso aos espaços da Faculdade de
Educação Física. Outras universidades como a UNESP e a USP Leste tem uma
política bem melhor de extensão. Na UNICAMP alguns alunos tem se dedicado
a desenvolver projetos de extensão que busquem a inclusão da comunidade na
universidade, através de cursinhos pré-vestibulares, reforço escolar,
e atividades com a população carente. A Computação historicamente tem pouco
engajamento, sendo que um dos únicos projetos com destaque é o GPSL, que
desenvolve a integração digital através do Software Livre, mas atualmente está
meio parado.

Resgatar o sentido do público tanto conceitual quanto materialmente se faz
sempre necessário. Assim, desde já, participe das discussões do centro acadêmico
para questionar as deficiências e produzir novos caminhos para formarmos com
a colaboração de todos uma universidade cada vez mais pública.

\subsection{Eu, um estudante público?} Você já parou para pensar o que está
começando ocorrer em sua vida? A partir de agora você estuda em uma universidade
pública, ou, como já foi dito, a partir de agora o povo está pagando para você
estudar. E o que você fará com esse privilégio?

Se você for perguntar, encontrará milhares de maneiras de encarar o fato de ser
um estudante público, e provavelmente algumas das respostas provavelmente
incluiriam um pouco das visões a seguir:

Alguns veem a aprovação como o último passo do desafio de entrar na
universidade, uma conquista pessoal, e, desta maneira, a única pessoa a quem
estes tem de prestar contas sobre o que fizeram de seus estudos na UNICAMP
seriam eles mesmos. Outros acreditam que estudar em uma universidade pública
traz uma responsabilidade direta: estudar corretamente. Um aluno da UNICAMP
teria de aproveitar a universidade ao máximo, buscando sempre aprender, para que
saia daqui como um profissional competente para auxiliar o progresso tecnológico
da nossa sociedade, de modo a cumprir com o papel que lhe foi atribuído. Também
existem algumas pessoas que acreditam que assim que entramos na universidade nos
tornamos agentes públicos. Sendo assim, além de estudar também seria papel do
aluno interagir constantemente com a comunidade passando a ela os conhecimentos
que a universidade lhe proporcionou, buscando criar um elo
universidade-comunidade.

Afinal, qual dessas visões seria a mais correta sobre o que é um estudante
público? Essa é uma resposta que não daremos aqui (até porque, como foi dito no
início, respostas não fazem parte do objetivo deste tópico), ela é algo
individual. Mas seria bom que você pensasse qual a razão pela qual você está na
UNICAMP e qual o objetivo da sociedade quando ela paga para que você tenha essa
oportunidade.

E, bixo, nunca é demais desejar: Que você faça o melhor proveito do seu tempo
aqui na UNICAMP!

\section{Lugares para morar}
O preço de uma casa é a área do paralelepípedo formado pelos seguintes valores
nos eixos: proximidade da UNICAMP, tamanho da casa (quantidade de quartos)
e qualidade da casa (acabamento, quantidade de banheiros, etc). Quanto
à distância, a avenida 1 (avenida Dr. Romeu Tórtima) e avenida 2 (avenida Prof.
Atílio Martini) são bem caras por serem próximas da UNICAMP, as ruas entre elas
muitas vezes também são. A região que vai do centro de Barão até a moradia
geralmente é boa e barata para se morar: Tem bastantes serviços e ainda é perto
da UNICAMP (mais ou menos 10 minutos de bicicleta).

Uma boa dica para se informar a respeito de lugares para morar (repúblicas,
kitnets, pensões) é o site Morar UNICAMP
(\url{http://www.morarunicamp.com.br/}), criado por alunos da UNICAMP e que
contém informações como endereço, preço, contato e detalhamento do lugar. E mais
lugares podem ser adicionados ao site.

\subsection{Moradia Estudantil}
A moradia estudantil é um exemplo de conquista de todos os estudantes.
O processo de reivindicação de uma moradia estudantil para a UNICAMP começou com
o movimento TABA. Durante muito tempo, alunos ficaram acampados no CB (Ciclo
Básico) para reivindicar seu direito a estudar. Hoje em dia, graças à moradia,
várias pessoas que possivelmente não teriam condições de se manter em Campinas
pagando aluguel, tem como estudar na UNICAMP. A moradia existe desde 1989
e durante esse período a UNICAMP ampliou muito suas vagas. Isto fez as vagas na
moradias tornarem-se insuficientes para todos que querem. Alguns anos atrás
houve grandes movimentações pela ampliação e reforma da moradia, pois algumas
casas estavam caindo.

Cada "casa" (que normalmente é dividida entre quatro pessoas) se constitui de um
quarto, uma cozinha, um banheiro e uma sala. Ainda tem o Circular da Moradia, um
ônibus da UNICAMP que transporta a galera durante o dia todo da moradia até
a UNICAMP e vice-versa.

A moradia está localizada na Avenida Santa Isabel, 1125. Aproximadamente a uns
3 km do campus da UNICAMP.

Para saber mais sobre o processo seletivo entre no site da Moradia Estudantil
(\url{http://www.prg.unicamp.br/moradia/index.html}). Muito provavelmente você
será abordado por um assistente social na matrícula.

\subsection{Repúblicas}
Geralmente a melhor escolha, se você tiver condições de pagar por uma moradia.
Você pode levar quem quiser para sua casa (dependendo da aprovação dos
moradores), chegar no horário que quiser, e conhecerá muita gente nova. Tente,
se possível, morar em uma república de cursos mistos, pois assim você terá
contatos diversos. O custo de uma vaga em uma república é muito variável
(depende do nível de conforto que você quer). Em uma república "normal" o custo
está em torno de R\$170,00 o aluguel para dividir quarto e R\$270,00 para pegar
um quarto sozinho. Procure bem as pessoas com quem você vai morar para não ter
problemas com diferentes estilos de vida (tem gente que gosta de lavar louça
a cada 5 minutos e tem gente que gostaria de morar junto com porcos, veja com
quem você se dá melhor).

Outra coisa importante é o tamanho da república. Repúblicas menores (3 ou
4 pessoas) acabam dando mais certo e sendo mais organizadas, mas saem mais caro
do que morar com 13 pessoas (verdade! tem repúblicas com essa quantidade de
gente). Veja o que mais se adapta a você e às suas posses financeiras.

\subsection{Kitnets}
Tomem cuidado com elas, pois a especulação imobiliária em Barão Geraldo chega
a ser imbecil. E nos últimos anos ficou fora de controle. As kitnets mobiliadas,
normalmente um quarto-sala-cozinha-área de serviço e banheiro, estão com valores
variando entre R\$ 550,00 a R\$ 1200,00. Sim, mil e duzentos reais por um micro
espaço. Só porque é perto da UNICAMP. Fique de olho e tome cuidado com os
contratos.

As melhores relações custo-benefício de kitnets são aquelas próximas ao centro
de Barão Geraldo ou no espaço entre as avenidas. E lembre-se não é só de UNICAMP
que se vive. Não adianta pagar mais caro para estar do lado da UNICAMP, se você
fica muito longe dos mercados, farmácias etc.

\subsection{Pensões}
Dependendo da pensão que você conseguir pode tornar-se uma grande roubada.
Algumas pensões não deixam você levar pessoas para sua casa, reclamam se você
chegar tarde e não liberam festas, já outras não; então procure bem. O preço
também não é muito bom, mas às vezes é mais barato que kitnets. É bom se você
quer um esquema casa-comida-roupa-lavada. É muito importante que você saiba que
contratos de um ano (ou qualquer período) em pensionatos são ilegais e você não
precisa cumpri-los.

\subsection{Dicas de Segurança}
Além de um novo ambiente de estudos, conhecerá novos lugares e pessoas. E,
provavelmente, em breve estará planejando sua mudança para Campinas.

Muitos dos estudantes moram em Barão Geraldo, por ser mais perto da UNICAMP,
possibilitando que sua bicicleta se torne seu meio de transporte principal, por
ser o lugar com maior número de repúblicas e pensionatos e também por ser onde
a maior parte das festas acontece!

No entanto, por ser constituído em sua maioria por casas de famílias com muitas
posses e casas de estudantes (em geral desatentos), Barão Geraldo peca pela
falta de segurança.

Não é raro ouvir de alguém que foi assaltado enquanto voltava para casa à noite
sozinho ou que a casa foi saqueada durante um feriado prolongado. Portanto,
é importante zelar pela sua integridade e de seus pertences -- assim como seus
pais fazem em sua casa, não importa onde eles morem.

Se sua pensão ou república paga o segurança da rua, faça uso dele, seja pedindo
escolta ao chegar em casa ou telefonando caso ouça algum barulho suspeito. Ao
voltar para sua cidade em feriados prolongados, deixando a casa vazia, não se
esqueça de trancar todas as portas e janelas de casa, verificar se não há nada
no quintal que possa ser levado facilmente (colocar as bicicletas e aparelhos de
som na sala pode ser uma boa ideia), e trancar os objetos de valor
(computadores, televisões) nos quartos.

\subsection{Portal Imobiliário}
O VivaReal é um portal imobiliário, onde você pode encontrar diversos tipos de
\href{http://www.vivareal.com.br}{imóveis}. Acesse o portal e confira os
\href{http://www.vivareal.com.br/venda/sp/sao-paulo/}{imóveis em São Paulo}
e veja as opções para
\href{http://www.vivareal.com.br/aluguel/sp/campinas/}{aluguel de imóveis, casas
    e apartamentos em Campinas} -- SP

VivaReal Portal Imobiliário
\url{http://www.vivareal.com.br}
Tel:(11) 3150-4646

\subsection{Imobiliárias}
\begin{itemize}
\item  \textbf{Imobiliária Barão Housing}
\begin{itemize}
\item  Telefone: (19)3289-4113.
\item  Endereço: Rua Tranquilo Prosperi, 383 -- Jd. Sta. Genebra II.
\item  E-mail: atendimento [@] baraohousing [.] com[.] br.
\item  Site: \url{http://www.baraohousing.com.br/}
\end{itemize}
\end{itemize}

\begin{itemize}
\item  \textbf{Imobiliária Home Hunters}
\begin{itemize}
\item  Endereço: Rua Benedito Alves Aranha, 55.
\item  Telefone: (19) 3289-0044.
\item  E-mail: bgeraldo [@] homehunters [.] com [.] br.
\item  Site: \url{http://www.homehunters.com.br/}
\end{itemize}
\end{itemize}

\begin{itemize}
\item  \textbf{Imobiliária Barão Geraldo}
\begin{itemize}
\item  Endereço: Aavenida Dr. Romeu Tórtima, 183.
\item  Telefone: (19) 3289-2315.
\item  E-mail: atendimento [@] imobiliariabaraogeraldo [.] com [.] br.
\item  Site: \url{http://www.imobiliariabaraogeraldo.com.br/}
\end{itemize}
\end{itemize}

\begin{itemize}
\item  \textbf{Imobiliária Lanza}
\begin{itemize}
\item  Endereço: Rua Benedito Alves Aranha, 104.
\item  Telefone: (19) 3289-1717.
\item  E-mail: lanza [@] lanzaimoveis [.] com [.] br.
\item  Site: \url{http://www.lanzaimoveis.com.br/}
\end{itemize}
\end{itemize}

\begin{itemize}
\item  \textbf{Imobiliária Professor Sebastião}
\begin{itemize}
\item  Endereço: Avenida Dr. Romeu Tórtima, 344.
\item  Telefone: (19) 3289-2317.
\item  E-mail: ipsimoveis [@] ipsimoveis[.] com [.] br.
\item  Site: \url{http://www.ipsimoveis.com.br/}
\end{itemize}
\end{itemize}

\begin{itemize}
\item  \textbf{Amaral Imóveis}
\begin{itemize}
\item  Endereço: Avenida Dr. Luiz de Tella, 864.
\item  Telefone: (19) 3287-5255.
\item  E-mail: amaral [@] amaralimoveis [.] net.
\item  Site: \url{http://www.amaralimoveis.net/}
\end{itemize}
\end{itemize}

\begin{itemize}
\item  \textbf{Zaine Conquista Imóveis}
\begin{itemize}
\item  Endereço: Avenida Santa Izabel, 84.
\item  Telefone: (19)3289-4050.
\item  E-mail: zaine [@] correionet [.] com [.] br.
\item  Site: \url{http://www.zaineconquista.com.br/}
\end{itemize}
\end{itemize}

\begin{itemize}
\item  \textbf{W. Post Imóveis}
\begin{itemize}
\item  Endereço: Avenida Dr. Romeu Tórtima, 1140.
\item  Telefone: (19) 3289-2155.
\item  E-mail: w.post [@] ig [.] com [.] br.
\item  Site: \url{http://www.wpostimoveis.com.br/}
\end{itemize}
\end{itemize}

\begin{itemize}
\item  \textbf{Ismê Assessoria Imobiliária}
\begin{itemize}
\item  Endereço: Rua Christina G. Miguel, 250.
\item  Telefone: (19) 3289-4325.
\item  E-mail: isme [@] isme [.] com [.] br.
\item  Site: \url{http://www.isme.com.br/}
\end{itemize}
\end{itemize}

\begin{itemize}
\item  \textbf{Rute Svartman Imóveis}
\begin{itemize}
\item  Telefone: (19) 3287-4641
\item  E-mail: imoveis [@] rutesvartman [.] com[.] br
\item  Site: \url{http://www.rutesvartman.com.br/}
\end{itemize}
\end{itemize}

\begin{itemize}
\item  \textbf{Imobiliária Ávila \& Ferraris}
\begin{itemize}
\item  Endereço: Avenida Dr. Romeu Tortima, 714
\item  Telefone: (19) 3289-4468
\item  E-mail: dcaavila [@] terra [.] com [.] br
\item  Site: \url{http://www.avilaeferrarisimoveis.com.br/}
\end{itemize}
\end{itemize}

\begin{itemize}
\item  \textbf{Imobiliária Cidade Universitária}
\begin{itemize}
\item  Endereço: Avenida Dr. Romeu Tortima, 624.
\item  Telefone: (19) 3289-3322.
\item  E-mail: contato [@] cidadeuniversitariaimoveis [.] com [.] br.
\item  Site: \url{http://www.cidadeuniversitariaimoveis.com.br/}
\end{itemize}
\end{itemize}

\begin{itemize}
\item  \textbf{Denilson Imóveis}
\begin{itemize}
\item  Endereço: Avenida Dr. Luís de Tella, 55.
\item  Telefone: (19) 3289-8909.
\item  E-mail: denilsonimoveis [@] terra [.] com [.] br.
\end{itemize}
\end{itemize}

\begin{itemize}
\item  \textbf{Mega Barão Imóveis}.
\begin{itemize}
\item  Endereço: Rua Francisca Resende Merciai, 103.
\item  Telefone: (19) 3289-7101.
\item  E-mail: megabarao [@] megabaraoimoveis [.] com [.] br.
\item  Site: \url{http://www.megabaraoimoveis.com.br/}
\end{itemize}
\end{itemize}

\begin{itemize}
\item  \textbf{Libano Imóveis}
\begin{itemize}
\item  Endereço: Rua Francisca Resende Merciai, 112.
\item  Telefone: (19) 3289-0441.
\item  E-mail: contato [@] libanoimoveis [.] com [.] br.
\item  Site: \url{http://www.libanoimoveis.com.br/}
\end{itemize}
\end{itemize}

\begin{itemize}
\item  \textbf{Marco Imóveis}.
\begin{itemize}
\item  Endereço: Rua José Pugliesi Filho, 420 -- Guará.
\item  Site: \url{http://www.marcoimovel.com.br/}
\end{itemize}
\end{itemize}

\begin{itemize}
\item  \textbf{Imobiliária Canal}
\begin{itemize}
\item  Endereço: Avenida Professora Ana Maria Silvestre Adade, 555.
\item  Telefone: (19) 3256-2622.
\end{itemize}
\end{itemize}

\begin{itemize}
\item  \textbf{Lokal Imóveis}
\begin{itemize}
\item  Endereço: Rua José Próspero Jacobucci, 290.
\item  Telefone: (19) 3256-4616.
\end{itemize}
\end{itemize}

\begin{itemize}
\item  \textbf{Esparta Empreendimentos Imobiliários}
\begin{itemize}
\item  Endereço: Avenida Albino José Barbosa de Oliveira, 830.
\item  Telefone: (19) 3289-5353.
\end{itemize}
\end{itemize}

\begin{itemize}
\item  \textbf{Cássio Carvalho Imóveis}
\begin{itemize}
\item  Endereço: Avenida Santa Isabel, 750.
\item  Telefone: (19) 3288-0144.
\item  E-mail: cassio [@] cassioimoveis [.] com [.] br.
\item  Site: \url{http://www.cassioimoveis.com.br/}
\end{itemize}
\end{itemize}

\begin{itemize}
\item  \textbf{Roma Imóveis}
\begin{itemize}
\item  Endereço: Rua Agostinho Pattaro, 222.
\item  Telefone: (19) 3289-0441.
\end{itemize}
\end{itemize}

\begin{itemize}
\item  \textbf{Valter Imóveis}
\begin{itemize}
\item  Endereço: Rua Maria Ferreira Antunes, 22.
\item  Telefone: (19) 3289-6088.
\end{itemize}
\end{itemize}

\begin{itemize}
\item  \textbf{Imobiliária Marco Antônio}
\begin{itemize}
\item  Endereço: Avenida Dr. Romeu Tórtima, 1522.
\item  Telefone: (19) 3289-2417.
\end{itemize}
\end{itemize}

\begin{itemize}
\item  \textbf{André Imóveis}
\begin{itemize}
\item  Endereço: Rua Maria Nassif Mokarzel, 42.
\item  Telefone: (19) 3289-6607.
\end{itemize}
\end{itemize}

\begin{itemize}
\item  \textbf{VPR Imóveis}
\begin{itemize}
\item  Endereço: Av. Professor Mário Werneck, 2011.
\item  Telefone: (19) 3289-6607.
\item  Site: \href{http://www.vprimoveis.com.br/}{imóveis buritis}
\end{itemize}
\end{itemize}

\section{Comida}
\subsection{Bandejão}
Um dos momentos de glória do dia de um futuro engenheiro, cientista ou bacharel
é o bandejão. É a hora de intensas e indiscutíveis emoções. Caso sua salada
corra sobre a mesa, mantenha-se calmo. Evite discussões, jamais tente descobrir
o sabor do suco pelo paladar (limão ou pêssego?) é mais cômodo ler no cardápio
do dia. Uma dica: para cortar o bife faça muita força e quando começar
a amolecer pare, você chegou na bandeja.

Falando sério agora: O bandejão (Restaurante Universitário), Bandex, ou Bandeco,
fica ao lado da Biblioteca Central, bem em frente ao PB (Prédio Básico, ou Ciclo
Básico II) e, a menos que você não queira economizar uma boa grana com comida,
vai ser o lugar onde você vai estar na maioria dos seus horários de almoço. Com
o tempo, você vai ver que o Bandex é o "coração da UNICAMP": É o local de você
se encontrar com os amigos (combinando ou não antes), contar os micos nas aulas,
jogar conversa fora, e falar mal da comida, que nem é tão ruim assim como muitos
dizem. Sem dúvida, é o melhor custo-benefício da UNICAMP: Por R\$2,00, você tem
direito a arroz, feijão, salada, proteína de soja, suco, chá e café à vontade,
a carne e a sobremesa tem que dar uma choradinha para a tiazinha para poder
repetir, mas geralmente dá certo.

Existe também o RA (Refeitório da Administração), também conhecido como Pratex,
pelo fato da comida ser servida em pratos e não em bandejas. Fica atrás da
Elétrica (FEEC), perto do prédio da Engenharia Básica. Tem algumas diferenças em
relação ao Bandex: O espaço físico é bem menor, por exemplo. No RA você mesmo se
serve, apesar da carne às vezes ser servida pela tia que trabalha lá. Se você
for com um amigo, vá com paciência para esperar, porque é difícil pra arrumar
lugar, além ser ultra apertado. Dependendo de onde você vai ter aula antes ou
depois do almoço, é mais negócio almoçar no Pratex. Para poder usar o Bandex
e o RA, você deve estar com o seu cartão (C.U.) carregado.

Nota: se vocês derem sorte, pode ser que o novo Pratex já esteja pronto e tenha
sido aberto. Ele é localizado próximo ao IC3 e ao prédio azul da Civil.

\subsubsection{Como funciona o esquema de carregar o cartão?}
Simples. Você vai ao guichê ao lado esquerdo da entrada do Bandex, e faz, por
exemplo, um depósito de R\$20,00 para 10 créditos. Outra maneira de colocar
créditos no C.U. é fazer um depósito na conta do Bandex no Banespa (Ag.: 207
/ Conta: 43.010.009-2) ou no Banco do Brasil (Ag.: 4203-X / Conta: 66.315-8)
e depois carregar o seu cartão, no guichê do Bandex ou na Prefeitura do Campus
(próximo à Reitoria). Fique esperto para não ir ao RA sem créditos, porque lá
não dá pra carregar o cartão e você terá que andar até o Bandeco.

Os bandejões funcionam de segunda a sexta, nos seguintes horários:

\begin{itemize}
\item  RU, das 10h30 às 14h00 (almoço) e das 17h30 às 19h40 (jantar).
\item  RA, das 11h30 às 14h00 (almoço) e fechado para jantar.
\end{itemize}
Para saber previamente o cardápio do Bandejão, acesse o site do CACo (\url{http://www.caco.ic.unicamp.br}).

\subsection{Outros lugares para as refeições}

Lugares que servem pratos feitos são a Física (um os melhores da UNICAMP, serve
também meio-prato), o IFCH (a comida lá é bem barata) e a Química (bem parecido
com o da Física).

A cantina do DCE tem self-service barato e com variedade no almoço. Outros
lugares que tem self-service são a Artes, a Educação, a Física e a Mecânica.

Se você é vegetariano, uma boa dica é o Gatti (que fica do lado do IC-2, na
Cênicas/Dança).

Fora da Unicamp: próximo ao balão da Av. 1, temos também o Terraço, que vende
marmitex e tem self-service a um preço bom, além de churrasco às terças, quintas
e sábados. Um pouco mais acima na Av. 1, tem o Bardana (um com a fachada toda
verde), que está na mesma faixa de preço do Terraço, e costuma ser considerado
bem melhor; tem churrasco de carne bovina meio que dia-sim-dia-não, e nos outros
dias é de frango. Próximo ao Bardana, tem o Pepe Loco, que serve comida mexicana
no estilo fast-food. Na frente da reitoria há o Del Sol, o Ginza e o Moriá.
O Del Sol serve comida por quilo, sendo parecido (em preço e pratos) com
o Bardana, enquanto que o Ginza serve a la carte com preços bons (uma dica
é a feijoada completa às quartas, que sai por R\$10,00 e inclui uma
mini-capirinha!) e o Moriá serve pratos feitos a preços mais baratos. Próximo ao
Ginza, em frente à guarita do HC, há o Campus Grill, com comida boa a um preço
um tanto alto (um pouco mais caro que a cantina da Física). Tem o Aulus, na Av.
2, próximo ao balão, que é o mais caro dos citados aqui, mas é muito bom (e
bonito).

\subsection{Lanches e sucos}

Tá de tarde, bateu fome, quer comer um lanche (hamburger, pão-na-chapa, queijo
quente, x-salada, croissant, qualquer coisa do gênero)? Quase todas as cantinas
da UNICAMP servem lanche. Algumas muito boas são o IFCH, a Física, a lanchonete
do IEL e a lanchonete da Economia.

Quase todas as cantinas servem salgados prontos, lanches naturais, doces
e demais coisas do gênero.

Para sucos, tem dois lugares muito bons: a cantina da Física e a famosíssima
banca de sucos do CB, que tem milhões de sucos, vende frutas e também salgados.
Se você precisa almoçar rápido, provavelmente sua escolha será salgado
+ vitamina na banca de sucos do CB. Todo dia a banca de sucos do CB tem um sabor
na oferta, que é ótimo pra sair do tradicional suco de laranja.

Açaí: a cantina da física tem um açaí na tigela, caro, mas bom. Se por algum
motivo você tiver de andar até o quarteirão de salas de aula da medicina,
estiver cansado, e quiser um açaí, o da cantina de lá é caro
e inacreditavelmente zoado.

Nas quartas-feiras há uma feira no centro da praça do CB, na qual há opções bem
variadas, desde pastéis a comida japonesa, embora geralmente mais caras que as
cantinas. Algumas das barracas abrem também na quinta-feira.

\subsection{Padarias e café da manhã}
Cinco cantinas da UNICAMP abrem bem cedo e servem o bom pingado + pão na chapa
matinal. São elas a Mecânica, a cantina do IFCH, a cantina do DCE, a da Química
e a da Física.

A Padaria Alemã serve uma bandeja de café da manhã com suco,
café-com-leite/chocolate, croissant, mamão, bolo, pão francês, torradas,
manteiga e geléia. Ainda há a possibilidade de fazer trocas como: suco por
chocolate, croissant por dois pães-na-chapa, mamão por banana, coisas do gênero.
Também são servidos lanches gigantescos, com muitas opções de recheio, por um
preço relativamente barato, então tenha alguém para dividir (acredite, meio
lanche já serve como um almoço completo). Dependendo do recheio, a pizza é muito
barata, também, embora eles não façam delivery. A Alemã fica na Avenida 1 (a da
saída da FEEC). É bom lembrar que eles servem café-da-manhã das 7h até às 13h
(mas a padaria só fecha às 22h), então é uma boa pedida para se você não quiser
almoçar ou para sábado e domingo, acordar tarde e tomar um café da manhã para
valer pelo almoço.

Na Estrada da Rhodia, próximo à entrada da Cidade Universitária II, há
a Paneteria Di Capri, que tem um pão francês muito bom (a um preço legal)
e também muita variedade (incluindo tortas e lanches). Além disso você também
pode tomar seu café da manhã lá, pois como quase toda padaria eles também
oferecem um cardápio bom para logo cedo. Se você estiver com bastante apetite,
de sexta a domingo eles servem um buffet de café-da-manhã com muitas opções
e a um preço fixo (em torno de R\$10). Na hora do almoço também são preparados
alguns pratos (para comer no local e para levar) e também há um esquema onde
você pede um grelhado e tem acesso livre a um balcão com saladas e outras
coisas, como petiscos. À noite eles servem pizzas e também há o esquema do
grelhado, exceto no inverno, quando eles servem um buffet de sopas.

Já se você está na Unicamp e quer uma padaria, a dica é a Padaria da FEA (fica
próxima à Cantina da Mecânica). Lá eles tem pães, doces e bolos. Com uma
diferença: há produtos especiais, como pão de queijo com linhaça ou alho e pão
francês com soja. Mas não se assuste: por mais estranho que pareçam, os produtos
de lá são muito bons! E não deixe para ir lá depois das aulas, pois a Padaria da
FEA fecha às 17h.

\subsection{E no fim de semana?}
Nos fins de semana, nem o Bandex nem quase nenhuma cantina da Unicamp abrem (e
as que abrem só o fazem no sábado). Você vai ter que se virar fora da Unicamp.
Na Av. 1 e proximidades tem o Terraço, o Bardana e a Padaria Alemã já citados,
o Lilly (que assim como o Terraço e o Bardana não abre de domingo), além de
vários restaurantes próximos à Alemã. Na Av. 2 tem o Aulus (mais caro no sábado
que durante a semana; domingo, então, mais ainda, mas costuma ter camarão
à milanesa), o Clos Vert (também é caro), um pouco mais pra cima na avenida e,
pouco depois, há o Yaki-Ten, que serve comida chinesa por quilo e japonesa por
pessoa. Logo mais abaixo há o Ilha do Barão. No centro de Barão não faltam
opções. Tem (indo da entrada de Barão pela Estrada da Rhodia) o Estância Grill,
o Barão da Picanha, o Gordão Burguers, o Solar dos Pampas, o Universo Massas,
o Vila Santo Antonio, o Ki-Pizza, o restaurante Baroneza, o Salsinha
e Cebolinha, o Pão de Açúcar e vários restaurantes no Tilli Center (a dica
é o Subway, por menos de 10 reais você come bem). Na Av. Santa Isabel
e adjacências tem o Cronópio (numa rua paralela à Santa Isabel), o Frangonete
(próximo ao Banespa), o HotDog Central e as Pizzarias Sapore Pizza e Pizza
Fiori. Perto da moradia tem a Tonha (Canto do Acarajé), o Kalunga Lanches
e o famoso dogão da moradia. Por fim, próximo à padaria Di Capri, há alguns
restaurantes mais caros, como a Romana (serviço parecido com o da Di Capri,
porém um bocado mais cara), Pizzaria Gregória, o TBONE (eles também tem
marmitex), o Greg Burgers (o hambúrguer e o milk-shake são excelentes), o Tábua
dos Mares e o Morena-flor.

\subsubsection{Alguns telefones:}
\begin{itemize}
\item  \textbf{Restaurante Baroneza}
\begin{itemize}
\item  Telefone: (19) 3289-9087.
\item  Endereço: Rua Benedito Alves Aranha, 44 -- Centro de Barão Geraldo.
\item  Site: \url{http://www.restaurantebaronesa.embarao.com/}
\end{itemize}
\end{itemize}

\begin{itemize}
\item  \textbf{China In Box} (Faz entrega em Barão)
\begin{itemize}
\item  Telefone: (19) 3254-5601
\item  Endereço: Rua Romualdo Andreazzi, 333 -- Jd. Trevo.
\end{itemize}
\end{itemize}

\begin{itemize}
\item  \textbf{TBONE Steak Bar}
\begin{itemize}
\item  Telefone: (19) 3289-0485.
\item  Endereço: Rua Maria Tereza Dias da Silva, 700.
\end{itemize}
\end{itemize}

\begin{itemize}
\item  \textbf{Habibs} (Não está mais entregando em Barão)
\begin{itemize}
\item  Telefone: 0800-778-2828
\end{itemize}
\end{itemize}

\begin{itemize}
\item  \textbf{Ginza Bar}
\begin{itemize}
\item  Telefone: (19)3289-9281
\item  Endereço: Rua Roxo Moreira, 1768.
\end{itemize}
\end{itemize}

\begin{itemize}
\item  \textbf{Bardana}
\begin{itemize}
\item  Telefone: (19)3289-9073.
\item  Endereço: Avenida Dr. Romeu Tórtima, 1500.
\end{itemize}
\end{itemize}

\begin{itemize}
\item  \textbf{Terraço}
\begin{itemize}
\item  Telefone: (19) 3289-7920
\item  Endereço: Rua Roxo Moreira, 1344
\end{itemize}
\end{itemize}

\begin{itemize}
\item  \textbf{Pastelaria Oba-Oba}
\begin{itemize}
\item  Telefone: (19) 3249-1908
\item  Endereço: Rua Benedito Alves Aranha, 115
\end{itemize}
\end{itemize}

\begin{itemize}
\item  \textbf{Estância Grill}
\begin{itemize}
\item  Telefone: (19) 3289-6055
\item  Endereço: Avenida Albino José Barbosa de Oliveira, 271
\end{itemize}
\end{itemize}

\begin{itemize}
\item  \textbf{Pizzaria Borda de Ouro.}
\begin{itemize}
\item  Telefone: (19) 3289-0867.
\item  Endereço: Luiz Vicentin Sobrinho, 457.
\end{itemize}
\end{itemize}

\begin{itemize}
\item  \textbf{Barão das Pizzas.}
\begin{itemize}
\item  Telefone: (19) 3249-1630.
\item  Endereço: Rua Agostinho Pattaro, 187.
\end{itemize}
\end{itemize}

\begin{itemize}
\item  \textbf{Pizza Fiori.}
\begin{itemize}
\item  Telefone: (19) 3289-3514.
\item  Endereço: Avenida Santa Isabel, 405.
\end{itemize}
\end{itemize}

\begin{itemize}
\item  \textbf{Ki-Pizza.}
\begin{itemize}
\item  Telefone: (19) 3289-0863.
\item  Endereço: Rua Horácio Leonardi, 76.
\end{itemize}
\end{itemize}

\begin{itemize}
\item  \textbf{Bella Pizza.}
\begin{itemize}
\item  Telefone: (19) 3289-7777.
\end{itemize}
\end{itemize}

\begin{itemize}
\item  \textbf{Pizza Show}
\begin{itemize}
\item  Telefone: (19) 3324-7480
\end{itemize}
\end{itemize}

\begin{itemize}
\item  \textbf{Quero Mais.}
\begin{itemize}
\item  Telefone: (19) 3289-4072.
\end{itemize}
\end{itemize}

\begin{itemize}
\item  \textbf{Estação Santa Fe Pizza.}
\begin{itemize}
\item  Telefone: (19) 3289-4800.
\end{itemize}
\end{itemize}

\begin{itemize}
\item  \textbf{Pizza Gigante (Mega Pizza).}
\begin{itemize}
\item  Telefone: (19) 3289-0320.
\end{itemize}
\end{itemize}

\begin{itemize}
\item  \textbf{Vila Ré Pizza.}
\begin{itemize}
\item  Telefone: (19) 3289-0321.
\end{itemize}
\end{itemize}

\begin{itemize}
\item  \textbf{NADOG'S -- HOT DOG DO NADO}
\begin{itemize}
\item  Telefone: (19) 3029-2270.
\end{itemize}
\end{itemize}

\begin{itemize}
\item  \textbf{Casa da Moqueca} (prato mais caro, mas serve duas pessoas)
\begin{itemize}
\item  Telefone: (19) 3289-3131.
\end{itemize}
\end{itemize}

\subsection{E de madrugada?}
\begin{itemize}
\item  \textbf{Rob's Burgers:} (entregas até as 23:30).
\begin{itemize}
\item  Telefone: (19)3289-6541.
\item  Endereço: Avenida Santa Isabel 1510 (em frente ao portao 2 da moradia da UNICAMP).
\item  Tem vários tipos de lanches, não cobram taxa de entrega e é aberto das 18:30 as 23:30. E para quem estiver com muita fome tem o x-especial que é grande e muito bom.
\end{itemize}
\end{itemize}

\begin{itemize}
\item  \textbf{Hot-dog Independência:}
\begin{itemize}
\item  Telefone: (19)3289-8805
\item  Endereço: Rua Angela Signol Grigol, 742
\item  Tem vários tipos de hot-dogs (com catupiry, com cheddar, com frango{\dots}) e tem preços menores que os do Rod Burguers. O único problema é que eles cobram taxa de entrega para um lanche e fecham à meia-noite.
\end{itemize}
\end{itemize}

\begin{itemize}
\item  \textbf{Kalunga Lanches:}
\begin{itemize}
\item  Telefone: (19)3289-5236
\item  Endereço: Rua Sebastião Bonomi, 40
\item  Perto da moradia, eles não entregam, mas ficam abertos até altas horas. Destaque para o caldinho de feijão. Obs: o lugar é limpo e bom.
\end{itemize}
\end{itemize}

\begin{itemize}
\item  \textbf{Ponto Final:}
\begin{itemize}
\item  Telefone: (19)3288-0204.
\item  Endereço: Avenida Albino José Barbosa de Oliveira, 2287
\item  Lanchonete localizada na estrada da Rhodia e entrega lanches até a meia noite. Tem tradição de ter preços caros, por isso não se estranhe. Muitos gostam bastante dessa lanchonete pela famosa maionese temperada que eles servem, não se esqueça de pedir quando for comprar lanches.
\end{itemize}
\end{itemize}

\begin{itemize}
\item  \textbf{Gordão:}
\begin{itemize}
\item  Telefone: (19)3289-9753
\item  Endereço: Avenida Albino José Barbosa de Oliveira, 476
\item  Rua Localizada na entrada de Barão Geraldo servem lanches parecidos com o do Ponto Final, lá eles dão outro tipo de maionese e em geral os preços são tão caros quanto do Ponto Final. Também entregam até meia noite.
\end{itemize}
\end{itemize}

\begin{itemize}
\item  \textbf{Lanchão \& Cia:}
\begin{itemize}
\item  Telefone: (19)3289-3665
\item  Endereço: Avenida Albino José Barbosa de Oliveira, 1214
\item  Site: \url{http://www.lanchaoecia.com.br/}
\item  (Fechou, abriu um novo barzinho no lugar em Barão). Um dos melhores lanches de Campinas (quiçá o melhor). Os lanches geralmente são grandes e muito bons, e os preços são compatíveis com a qualidade e quantidade. Eles servem no carro se você preferir, com uma bandeja que fica presa no vidro. Fica no centro de Barão Geraldo, proximo ao Banespa e Pão de Açucar. Destaque para a batata frita, feita de uma forma muito diferente, extremamente crocante e quase cremosa por dentro. Há outras duas lojas próximas da avenida Norte-Sul. Uma na Rua Oriente, e outra na Orozimbo Maia. O Art Lanches (antigo Lanchão 2), que fica no Taquaral, serve lanches parecidos.
\end{itemize}
\end{itemize}

\begin{itemize}
\item  \textbf{Ponto 1:}
\begin{itemize}
\item  Telefone: (19)3289-2378.
\item  Rua Eduardo Modesto, 54
\end{itemize}
\end{itemize}

\begin{itemize}
\item  \textbf{Sapore Pizza:}
\begin{itemize}
\item  Telefone: (19)3289-0228
\item  Endereço: Avenida Santa Isabel, 326
\item  Para quando você estiver com pelo menos mais um amigo para rachar a pizza, acaba sendo uma boa pedida. Geralmente as pizzas de mussarela e de calabreza estão com preços bem acessíveis. Também entregam até meia-noite.
\end{itemize}
\end{itemize}

\begin{itemize}
\item  \textbf{Barraquinhas:}
\begin{itemize}
\item  Há várias barraquinhas de hot-dog no centro de Barão e perto da moradia. Destaque para o dog do terminal, o Hot Dog Central, o Pedrogue e o dogão da moradia. Se você quiser um lanche, uma boa pedida é o Star Trash (Raimundão ou Guarujá, chame como você quiser), que fica perto do balão da avenida 2 e costuma ficar aberto até altas horas. Perto da UNICAMP, ao lado do posto Ipiranga que fica na avenida 1 também tem um dog prensado muito bom e barato.
\end{itemize}
\end{itemize}

\subsection{Marmitex}
Entrega em casa. Bom e barato.

\begin{itemize}
\item  \textbf{Tia Rita.}
\begin{itemize}
\item  Telefone: (19) 3249-2899.
\end{itemize}
\end{itemize}

\begin{itemize}
\item  \textbf{Hailton}
\begin{itemize}
\item  Telefone: (19) 3249-0153.
\end{itemize}
\end{itemize}

\begin{itemize}
\item  \textbf{Copa e Cozinha}
\begin{itemize}
\item  Telefone: (19) 3249-0153.
\end{itemize}
\end{itemize}

\begin{itemize}
\item  \textbf{Império do Barão}
\begin{itemize}
\item  Telefones: (19) 3289-8054 e (19) 3289-2170.
\end{itemize}
\end{itemize}

Obs: A Sapore Pizza também entrega Marmitex.

\subsection{Bares, lanchonetes e restaurantes}
\begin{itemize}
\item  \textbf{Açaizeiro Brasil:} Serve um açaí muito bom e vários tipos de comidas mais leves, como lanches naturais, crepes e saladas, além de vários sucos. O preço não é caro e a comida é boa. Endereço: Avenida Santa Isabel, 518. Telefone: (19) 3365-6555. Site: \url{http://www.portalbaraogeraldo.com.br/anunciantes/acaizeiro-brasil/}
\end{itemize}

\begin{itemize}
\item  \textbf{Alkobar:} Comida árabe e pizzas no início da Santa Isabel. Bom atendimento, comida legal e preço razoável. Telefone: 3289-2755.
\end{itemize}

\begin{itemize}
\item  \textbf{Aulus VideoBar \& Restaurant:} A comida lá é muito boa, só que é muito caro também (especialmente no final de semana), exceto pelo marmitex. Um ambiente diferente, com bicicletas e ferroramas no teto, por exemplo.
\end{itemize}

\begin{itemize}
\item  \textbf{Bagdá Café -- Bar \& Esfiharia:} Esfihas boas, mas um pouco caras. Entregam em Barão (cardápio no site), mas em horários de pico costumam demorar um pouco. A música ambiente inclui música ao vivo e ritmos variados, desde a MPB ao Blues. Endereço: Av. Santa Isabel, 246. Telefones: (19)3289-0541 e (19)3289-1842. Site: \url{http://www.carlinoamaral.com.br/bagda/}
\end{itemize}

\begin{itemize}
\item  \textbf{Bar do Coxinha:} Famoso pela coxinha (realmente boa), vale a pena ir lá, mas é relativamente caro. Localiza-se perto da avenida Santa Isabel, na rua da Sapore Pizza.
\end{itemize}

\begin{itemize}
\item  \textbf{Bar do Jair:} Outro lugar famoso pela coxinha: só que esta é de carne seca. Fica na Rua Eduardo Modesto, 212, relativamente perto da Moradia.
\end{itemize}

\begin{itemize}
\item  \textbf{Barão da picanha:} Churrascaria rodízio localizada na avenida Albino José Barbosa de Oliveira, logo na entrada de Barão.
\end{itemize}

\begin{itemize}
\item  \textbf{Batataria Suiça:} Do lado do Ponto Final, serve batatas recheadas bem diferentes. É um pouco caro, mas vale a pena conferir. Uma dica é que às terças-feiras você compra uma batata, mas recebe duas. Endereço: Estrada da Rhodia -- Praça José Geraldi, a 50m do posto Esso. Telefone: (19) 3201-1174. Site: \url{http://www.battataria.com.br}
\end{itemize}

\begin{itemize}
\item  \textbf{Boi Falô:} O restaurante é uma rancho, com comida típica do interior. É excelente, mas um pouco caro (cerca de R\$30,00 por pessoa), um lugar perfeito para levar seus pais quando eles vêm te visitar (e pagam o almoço!). Abre apenas nos almoços de sábado e domingo. Endereço: Rua do Sol, 600. Telefone: (19) 3289-6671. Site: \url{http://www.boifalo.com.br/}
\end{itemize}

\begin{itemize}
\item  \textbf{Cachaçaria Água Doce:} Localizada na avenida 1, é um lugar frequentado por pessoas mais velhas, ótimo para comida e bebida (pinga, especialmente), mas é bem caro.
\end{itemize}

\begin{itemize}
\item  \textbf{Casa São Jorge:} Música ao vivo todas as noites, com boa variedade. Localiza-se na rua Santa Isabel, mais ou menos perto da moradia.
\end{itemize}

\begin{itemize}
\item  \textbf{Empório Nono:} Caro, tem um chopp muito bem tirado e os melhores petiscos de Campinas. Localiza-se na avenida Albino José Barbosa de Oliveira, quase em frente ao terminal.
\end{itemize}

\begin{itemize}
\item  \textbf{Estância Grill:} Logo na entrada de Barão. Tem rodízios de carne e de pizza à noite. Endereço: Avenida Albino José Barbosa Oliveira, 271. Telefone: (19) 3289-8697/3289-6055/3289-1511
\end{itemize}

\begin{itemize}
\item  \textbf{Fernando's:} No centro de Barão, perto do Banespa, serve cerveja e lanches baratos e muito bons principalmente porque vêm acompanhados de uma porção pequena de fritas! Um lugar simples mas muito limpo e agradável principalmente em relação ao atendimento. Fecha as 23h se segunda a quinta e sábado, tem música ao vivo na sexta e por enquando ainda não abre nos domingos.
\end{itemize}

\begin{itemize}
\item  \textbf{Fran's Café:} Cafeteria. Vende lanches, cafés, doces, salgados e bebidas (quentes ou geladas). Fazem também cafés da manhã. Mas é um pouco caro. Localizado na avenida Albino José Barbosa de Oliveira, 1600.
\end{itemize}

\begin{itemize}
\item  \textbf{Greg Burguers:} Uma lanchenete muito boa, mas também muito cara. Uma das especialidades lá é o milk-shake (realmente muito bom). Fica na estrada da Rhodia (na esquina da Paneteria Di Capri). Endereço: Rua Maria Tereza Dias da Silva 664, Barão Geraldo, 13083-820 Campinas. Telefone: (19) 3289-6400. Site: \url{http://www.gregburgers.com.br}
\end{itemize}

\begin{itemize}
\item  \textbf{Ilha do Barão:} Lá eles fazem entrega de marmita de graça, a 5,00, e a comida é boa. Também tem cerveja barata. Fica na avenida 2, perto do Texaco e do Mc Donalds.
\end{itemize}

\begin{itemize}
\item  \textbf{La Salamandra:} Restaurante mexicano, localizado ao lado do Makis Place. Comida boa e preço compatível, ele tem uma barraquinha na feirinha do CB, às quartas. Endereço: Avenida Albino José Barbosa de Oliveira, 998. Telefone: (19) 3289-2011/(19) 9277-4340/(19)3365-0354. Site: \url{http://www.portalbaraogeraldo.com.br/anunciantes/la-salamandra-culinaria-mexicana-/}
\end{itemize}

\begin{itemize}
\item  \textbf{Makis Place:} Temakeria próxima ao terminal. Endereço: Avenida Albino José Barbosa de Oliveira, 976. Telefone: (19) 3367-3077. Site: \url{http://www.makis.com.br/}
\end{itemize}

\begin{itemize}
\item  \textbf{Ponto Final:} Já comentado acima, fica aberto até altas horas. Mas à noite serve cerveja também a um bom preço. Localiza-se na estrada da Rhodia (continuação da avenida Albino José de Oliveira). Endereço: Av. Albino J. B. Oliveira, 2287. Telefone: (19)3288-0204.
\end{itemize}

\begin{itemize}
\item  \textbf{Quintal do Neto:} No alto da avenida 1, perto do balão de entrada em Barão Geraldo, tem cerveja a preços razoáveis, salgados (coxinha e quibe) grandes, e mesas de sinuca (de ficha e por hora).
\end{itemize}

\begin{itemize}
\item  \textbf{Rudá:} Localizado na Santa Isabel, é um bar novo, com música ambiente.
\end{itemize}

\begin{itemize}
\item  \textbf{Santa Fé Pizza Bar:} Local um pouco caro, mas normalmente com música ao vivo de ótima qualidade. A melhor pizza de Barão (também cara mas mais barata que a Pizzaria Vila Ré). Próximo ao Pão de Açúcar.
\end{itemize}

\begin{itemize}
\item  \textbf{Seu Pimenta:} Muito caro, mais caro até que o Santa Fé e o Nono. Bom atendimento, porções de boteco, e às vezes música ao vivo. Em frente ao Zé Espaço e Bar.
\end{itemize}

\begin{itemize}
\item  \textbf{Solar dos Pampas:} Fazem um esquema no aniversário das pessoas que sai por R\$ 18,00 com rodízio, cerveja, refrigerante, buffet, sorvete e pinga a vontade. Ao lado do Universo Massas. Endereço: Av. Romeu Tórtima, 165. Telefones: (19)3289-1484 e (19)3289-7869.
\end{itemize}

\begin{itemize}
\item  \textbf{Star Clean:} É o bar mais próximo à UNICAMP, e por isso está sempre cheio. Principal ponto de encontro depois da aula e tem um bom preço.
\end{itemize}

\begin{itemize}
\item  \textbf{Star Trash:} Do lado do Star Clean, conhecido como Star Trash, Raimundo Raimundão, ou ainda, Star Trailer, é um trailer que serve cerveja e lanches baratos.
\end{itemize}

\begin{itemize}
\item  \textbf{Subway:} Lanchonete. Vende dos mais variados tipos de lanches. Lanches muito bons, mas caros. Localiza-se no Tilli Center (avenida Albino José Barbosa de Oliveira, 1556, esquina com a avenida 2). Do lado do Subway tem um caixa 24 horas que trabalha com os principais bancos.
\end{itemize}

\begin{itemize}
\item  \textbf{Temakeria:} Lugar relativamente novo, meio caro. Vende só temaki e bebidas. O horário de funcionamento é bastante conveniente. Endereço: Av. Romeu Tórtima, 1259 (relativamente próximo à Unicamp, um pouco pra cima do Bardana). Telefone: (19) 3289-0802. Horário de funcionamento: domingo a terça das 11h30 às 0h, quarta a sábado das 11h30 às 6h. Site: \url{http://www.tmkr.com.br/}
\end{itemize}

\begin{itemize}
\item  \textbf{Universo Massas:} Rodízio de massas perto do terminal. Bom e não é caro. De domingo à noite é o horário mais barato e dá pra encher bem o bucho de massa. Depois de ir até lá, você não vai querer saber de comer massas por um bom tempo. Endereço: Avenida Albino José Barbosa de Oliveira, 576. Telefone: (19) 3289-5369.
\end{itemize}

\begin{itemize}
\item  \textbf{Vila Ré -- Pizza:} Pizzaria próxima do terminal e do supermercado Dalben. Tem alguns sabores diferentes, as pizzas são boas e o preço não é alto. Possui serviço de entrega das 18h às 23h. Endereço: Avenida Albino José Barbosa de Oliveira, 658. Telefone: (19) 3289-0319
\end{itemize}

\begin{itemize}
\item  \textbf{Zé Espaço e Bar:} Barato, mas bem pequeno. Cerveja com desconto antes das 21h. Localiza-se também na avenida Albino José de Oliveira, bem em frente ao Pão de Açúcar.
\end{itemize}

\section{Diversão}
\subsection{Espaços culturais}
\begin{itemize}
\item  \textbf{Casa do Lago:} Dentro da UNICAMP, mantém uma programação quase diária de cinema e exposições. Este ano estão viabilizando a instalação de café e revistaria. A página da casa do lago é: \url{http://www.preac.rei.unicamp.br/casadolago/}.
\end{itemize}

\begin{itemize}
\item  \textbf{Semente:} Fica no fim da avenida Santa Isabel, depois da moradia. Sempre tem apresentações artísticas, como teatros e espetáculos musicais.
\end{itemize}

\subsection{Cinemas}
\begin{itemize}
\item  \textbf{Kinoplex (15 salas).}
\begin{itemize}
\item  Endereço: Shopping D. Pedro (Rodovia Dom Pedro I, Km 137 -- Jd. Sta. Genebra).
\item  Telefone: (19) 3131-2800.
\item  Site: \url{http://www.kinoplex.com.br/}
\end{itemize}
\end{itemize}

\begin{itemize}
\item  \textbf{Cinemark Iguatemi (8 salas).}
\begin{itemize}
\item  Endereço: Shopping Center Iguatemi (Avenida Iguatemi, 777 -- Vila Brandina).
\item  Telefone: (19) 3251-1122.
\item  Site: \url{http://www.cinemark.com.br/}
\end{itemize}
\end{itemize}

\begin{itemize}
\item  \textbf{Cine Jaraguá (2 salas).}
\begin{itemize}
\item  Endereço: Shopping Prado (Bairro Parque Prado).
\item  Telefone: (19) 3243-6537.
\item  Site: \url{http://www.cinejaragua.com.br/}
\end{itemize}
\end{itemize}

\begin{itemize}
\item  \textbf{Box Cinemas (10 salas).}
\begin{itemize}
\item  Endereço: Campinas Shopping (Rua Jacy T. de Camargo, 940 -- Jardim do Lago).
\item  Telefone: (19) 4005-1717.
\item  Site: \url{http://www.boxcinemas.com.br/complexo_local.asp?id_complexo=13}
\end{itemize}
\end{itemize}

\begin{itemize}
\item  \textbf{Cine Galleria (7 salas).}
\begin{itemize}
\item  Endereço: Galleria Shopping (Rod. Dom Pedro I, Km, 131,5 -- Jd. Nilópolis).
\item  Telefones: (19) 3207-0982.
\item  Site: \url{http://www.galleria.com.br/page/cinema.asp}
\end{itemize}
\end{itemize}

\begin{itemize}
\item  \textbf{Cine Moviecom Unimart (4 salas).}
\begin{itemize}
\item  Endereço: Shopping Unimart (Avenida John Boyd Dunlop, 350 -- Chácara da República).
\item  Telefone: (19) 3243-5206.
\item  Site: \url{http://www.moviecom.com.br/indexp.php?id=CAM}
\end{itemize}
\end{itemize}

\begin{itemize}
\item  \textbf{Sala Cine Paradiso.}
\begin{itemize}
\item  Endereço: Galeria Barão Velha (Rua Barão de Jaguara, 936 -- Centro).
\item  Telefone: (19) 3234-4741.
\end{itemize}
\end{itemize}

\begin{itemize}
\item  \textbf{Cine Evolução/MIS (1 sala)}
\begin{itemize}
\item  Endereço: Rua Regente Feijó, 1087 -- Centro
\item  Telefone: (19) 3232-9959
\end{itemize}
\end{itemize}

\subsection{Teatros}
\begin{itemize}
\item  \textbf{Lume Teatro}
\begin{itemize}
\item  Endereço:  Rua Carlos Diniz Leitão, 150 Vila Santa Isabel -- Barão Geraldo
\item  Telefone: (19) 3289-9869
\item  Site: \url{http://www.lumeteatro.com.br/}
\end{itemize}
\end{itemize}

\begin{itemize}
\item  \textbf{Teatro Interno Luiz Otávio Burnier}
\begin{itemize}
\item  Endereço: Centro de Convivência Cultural (Praça Imprensa Fluminense s/nº -- Cambuí)
\item  Telefone: (19) 3252-5977
\end{itemize}
\end{itemize}

\begin{itemize}
\item  \textbf{Teatro de Arena}
\begin{itemize}
\item  Endereço: Centro de Convivência Cultural (Praça Imprensa Fluminense s/nº -- Cambuí)
\item  Telefone: (19) 3252-5977
\end{itemize}
\end{itemize}

\begin{itemize}
\item  \textbf{Teatro Carlos Maia}
\begin{itemize}
\item  Endereço: Rua Cel. Quirino, 2 -- Bosque dos Jequitibás
\item  Telefone: (19) 3231-8795
\end{itemize}
\end{itemize}

\begin{itemize}
\item  \textbf{Teatro José de Castro Mendes}
\begin{itemize}
\item  Endereço: Praça Corrêa de Lemos, s/nº -- Vila Industrial
\item  Telefone: (19) 3272-9359
\end{itemize}
\end{itemize}

\begin{itemize}
\item  \textbf{Auditório Beethoven (Concha Acústica)}
\begin{itemize}
\item  Endereço: Avenida Heitor Penteado, s/nº -- Portão 2 -- Lagoa do Taquaral
\item  Telefone: (19) 3256-9959
\end{itemize}
\end{itemize}

\begin{itemize}
\item  \textbf{Teatro de Arte e Ofício}
\begin{itemize}
\item  Endereço: Rua Conselheiro Antônio Prado, 529 -- Vila Nova
\item  Telefone: (19) 3241-7217
\end{itemize}
\end{itemize}

\begin{itemize}
\item  \textbf{Teatro Dom Nery (Externato São João)}
\begin{itemize}
\item  Endereço: Rua José de Alencar, 360  Centro
\item  Telefone: (19) 3231-2644
\end{itemize}
\end{itemize}

\begin{itemize}
\item  \textbf{Teatro Teresa Aguiar (Conservatório)}
\begin{itemize}
\item  Endereço: Rua José de Alencar, 701 -- Centro
\item  Telefone: (19) 3232-9345
\end{itemize}
\end{itemize}

\begin{itemize}
\item  \textbf{Teatro da Vila Padre Anchieta}
\begin{itemize}
\item  Endereço: Avenida Cardeal Dom Agnelo Rossi, s/nº -- Vila Padre Anchieta
\item  Telefone: (19) 3781-0382
\end{itemize}
\end{itemize}

\begin{itemize}
\item  \textbf{Centro Cultural Evolução}
\begin{itemize}
\item  Endereço: Rua Regente Feijó, 1087 -- Centro
\item  Telefone: (19) 3232-9959
\end{itemize}
\end{itemize}

\begin{itemize}
\item  \textbf{Teatro TIM}
\begin{itemize}
\item  Endereço: Shopping D. Pedro, Entrada das flores
\item  Telefones: (19) 3756-9890 e (19) 3756-9891
\end{itemize}
\end{itemize}

\subsection{Boates e baladas}
\begin{itemize}
\item  \textbf{Cooperativa Brasil:} Para quem gosta de um bom forró, sempre com shows diversos. A galera gosta muito da quarta-universitária. Maiores informações, acesse o site: \url{http://www.cooperativabrasil.com.br/}.
\end{itemize}

\begin{itemize}
\item  \textbf{Campinas Hall:} Muitas das festas mais legais da UNICAMP acontecem lá (como a Festa Brega e a Festa do Contrário), perto da PUCC. É bem grande.
\end{itemize}

\begin{itemize}
\item  \textbf{Delta Blues Bar:} O melhor do Blues \& Rock 'n Roll de Campinas -- \url{http://www.deltabluesbar.com.br/}.
\end{itemize}

\begin{itemize}
\item  \textbf{Barril da Máfia:} \url{http://www.barrildamafia.com.br/}.
\end{itemize}

\begin{itemize}
\item  \textbf{Swingers:} Tem bebidas caras e cada noite tem um tema, mas é melhor às quartas e quintas-feiras. Só homens maiores de 21 e mulheres com mais de 18 anos podem entrar. \url{http://www.grupohappynews.com.br/novo/swingers_campinas/index.php}
\end{itemize}

\begin{itemize}
\item  \textbf{Golden:} Assim como a Swingers fica no Shopping D. Pedro. Balada cara, frequentada geralmente por gente mais velha. \url{http://www.goldstreetbar.com.br/}
\end{itemize}

\begin{itemize}
\item  \textbf{Camaleão Show:} Antigo Club Diesel. Primeira casa noturna de micareta e pagode de Campinas. \url{http://www.camaleaoshow.com.br/}
\end{itemize}

\begin{itemize}
\item  \textbf{Kraft:} Localizada próxima ao Taquaral (na Avenida Imperatriz Leopoldina), toca musica psi a noite toda e fica aberta até quase o amanhecer. Mulher entra de graça até a meia-noite. \url{http://www.clubekraft.com/}
\end{itemize}

\begin{itemize}
\item  \textbf{Cambuí:} Neste bairro existem diversos barzinhos, a maioria é temático, alguns são um pouco caros e cobram covert. É um ótima escolha para quem tiver carro pois fica um pouco longe de Barão.
\end{itemize}

\subsection{Shopping Centers}
\begin{itemize}
\item  \textbf{Shopping Parque D. Pedro:} Foi considerado o maior shopping da América Latina até pouco tempo atrás. Localiza-se na Rodovia Dom Pedro, km 137 (razoavelmente próximo a UNICAMP). O ônibus 3.38 sai do terminal Barão vai para lá e para o Iguatemi. Página do shopping: \url{http://www.parquedpedro.com.br/}.
\end{itemize}

\begin{itemize}
\item  \textbf{Shopping Iguatemi:} Shopping normal, o mais antigo e o segundo maior de Campinas. Localiza-se na Avenida Iguatemi, 777. O 3.38 demora uns 40 minutos para chegar lá. Frequentado por pela galera mais nova e pelo pessoal com um pouco mais de dinheiro. Página do shopping: \url{http://www.iguatemicampinas.com.br/}.
\end{itemize}

\begin{itemize}
\item  \textbf{Shopping Jaraguá:} Shopping pequeno e que possui duas unidades: Uma delas fica na Rua Conceiçao (Jaraguá Conceição); e a outra unidade fica na avenida Brasil (Jaraguá Brasil). Nesse segundo há duas salas de cinema que passam filmes "cult". O ônibus 3.30, no sentido terminal central -- UNICAMP para razoavelmente próximo ao Jaraguá Conceição, um ponto antes do ponto da prefeitura; e os ônibus 3.30, 3.31, 3.32 e 3.33 passam em frente ao Jaraguá Brasil.
\end{itemize}

\begin{itemize}
\item  \textbf{Campinas Shopping:} Longe a dar com pau, mas as lojas não são muito caras. Localiza-se às margens das rodovias Anhanguera e Santos Dumont. Provavelmente você nunca irá lá. Página do shopping: \url{http://www.campinasshopping.com.br/}.
\end{itemize}

\begin{itemize}
\item  \textbf{Shopping Prado:} Shopping bem pequeno, as lojas não são muito caras, mas muito longe. Localiza-se na Avenida Washington Luís, 2480. Provavelmente você também nunca irá lá.
\end{itemize}

\begin{itemize}
\item  \textbf{Galleria Shopping:} Muito bonito, mas lojas muito caras. Também localizado na Rodovia Dom Pedro, mas no km 131,5. O ônibus 3.00 sai do terminal de Barão Geraldo e passa lá. Página do shopping: \url{http://www.galleria.com.br/}.
\end{itemize}

\begin{itemize}
\item  \textbf{Shopping Unimart:} Shopping pequeno, as lojas não são muito caras. Localiza-se na Avenida John Boyd Dunlop, 350. O ônibus 1.34 sai do terminal de Barão Geraldo e passa próximo. Página do shopping: \url{http://www.unimart.com.br/}.
\end{itemize}

\subsection{Lan House}
\begin{itemize}
\item  \textbf{Lion Lan:} Localizada perto da Praça do Coco e da moradia, com 40 PCs. Cobram R\$3,00 a hora.
\end{itemize}

\section{Transporte}
\subsection{Voltar para casa}
Para os calouros de fora de Campinas, além de escolher a nova morada
é importante recolher informações sobre como realizar o trajeto entre sua cidade
e Campinas.

A forma usual é ir de ônibus, mas tenha em mente que a rodoviária é longe e os
trajetos de ônibus até lá são demorados. O endereço da rodoviária é Rua Barão de
Parnaíba, 690. Há duas linhas que passam por lá: o 3.32 (que passa dentro da
UNICAMP) para dentro da rodoviária, mas demora mais pra chegar que o 3.31, que
sai do terminal e para do lado de fora da rodoviária. Em horários de pico,
o trajeto pode demorar quase uma hora, então cuidado para não perder o horário
do ônibus para sua cidade.

Os que vêm de mais longe certamente farão uso do aeroporto de Viracopos, cujo
telefone é 3725-5000. Para chegar ao aeroporto existe a linha 193 que sai da
rodoviária e vai para o aeroporto, e também faz o trajeto de volta. Porém ir com
ônibus circular pode ser um transtorno quando estiver com mala grande. Uma outra
alternativa é a Caprioli que também faz o trajeto da rodoviária para
o aeroporto. A passagem da Caprioli custa em torno de R\$8,00 e os horários
podem ser conferidos no site \url{http://www.caprioli.com.br}.

Para quem for usar os aeroportos da Grande São Paulo (Congonhas e Guarulhos),
também existe o translado da Caprioli. A tarifa sai em torno de R\$30,00
a R\$35,00 e os horários podem ser conferidos no site da Caprioli.

\subsubsection{Caronas}
Uma forma barata e divertida de viajar, além de minimizar o tempo e dinheiro
gastos na viagem, é juntando alguns estudantes no mesmo carro e dividir as
despesas.

O site UniCaronas (antigo Caronas Unicamp -- \url{http://www.unicaronas.com.br})
foi desenvolvido por dois engenheiros de computação da UNICAMP, da turma 2004
(Guilherme Souza e Matheus Marosti), com o intuito de facilitar o deslocamento
dos alunos entre cidades. Em quatro anos de atividade, reuniu mais de 10000
usuários, ajudando a reduzir os custos de viagens e contribuindo para o aumento
do círculo de amizades dos participantes.

Atualmente o site possui carona para diversas cidades de vários estados, mas
novas cidades podem ser inseridas à medida que aumentar a demanda por elas.

Para garantir a segurança dos usuários o site exige no cadastro que
o interessado possua um email da seguintes instituições:

\begin{itemize}
\item  ESPM.
\item  FAMEMA.
\item  FGV.
\item  ITA.
\item  Mackenzie.
\item  PUCCAMP.
\item  UEL.
\item  UEM.
\item  UFABC.
\item  UFBA.
\item  UFES.
\item  UFJF.
\item  UFLA.
\item  UFMG.
\item  UFPR.
\item  UFRGS.
\item  UFSCAR.
\item  UFSJ.
\item  UFTM.
\item  UFU.
\item  UFV.
\item  UNB.
\item  UNESP.
\item  UNICAMP.
\item  UNIFAL.
\item  UNIFEI.
\item  UNIFESP.
\item  USP.
\end{itemize}

Ou então seja convidado por alguém que já usa o site. Assim como as cidades,
novas instituições são adicionadas a medida que aumentar a demanda.

Existe também um sistema de comentários, para que se possa alertar os demais
usuários sobre uma carona anterior: Se o motorista corria demais, se o motorista
ou caronista chegou ao local combinado no horário, se o carro era confortável ou
parecia uma lotação e outros indicativos relevantes recomendando ou não
a carona. No entanto, o site serve apenas para colocar os motoristas
e caronistas em contato, não assumindo responsabilidades sobre nenhuma das
partes.

Além disso o site também mantem uma página
(\url{http://unicaronas.uservoice.com/forums/36297-geral}) aonde os usuários
podem fazer sugestões.

\subsection{Carro}
Para aqueles que tenham seu próprio veículo, é bom saber que a UNICAMP tem
poucas vagas próximas aos locais de aulas. E o número de fiscais de trânsito tem
aumentado muito dentro do Campus.

\subsubsection{Autoescolas}
Se você ainda não tem CNH e pretende obtê-la em Campinas, existem duas opções de
autoescola em Barão Geraldo:

\begin{itemize}
\item  \textbf{Auto Escola Avenida.}
\begin{itemize}
\item  Endereço: Avenida Albino José Barbosa De Oliveira, 658.
\item  Telefone: (19) 3288-0588.
\end{itemize}
\end{itemize}

\begin{itemize}
\item  \textbf{Auto Escola Advanced.}
\begin{itemize}
\item  Endereço: Avenida Santa Isabel, 80.
\item  Telefone: (19) 3289-3022.
\end{itemize}
\end{itemize}

\subsubsection{Recorrer de Multas}
Documentos:
\begin{itemize}
\item  Notificação e Fotocopia.
\item  CNH e Fotocopia.
\item  Documento do Carro e Fotocopia.
\item  Formulário: \url{http://www.emdec.com.br/multas/formulario_unificado.pdf}.
\item  e anexos, se desejar.
\end{itemize}

E levá-los a:
\begin{itemize}
\item  EMDEC.
\begin{itemize}
\item  Horário: De segunda a sexta-feira, das 8h as 17h.
\item  Endereço: Rua Dr. Salles Oliveira, 1028 -- Vila Industrial -- CEP 13035-270.
\end{itemize}
\item  Poupatempo (Centro).
\begin{itemize}
\item  Horário: Das 8h às 18h, de segunda a sexta-feira; e aos sábados, das 7h às 13h.
\item  Endereço: Avenida Francisco Glicério, 935.
\end{itemize}
\item  Poupatempo (Campinas Shopping)
\begin{itemize}
\item  Horário: Das 9h às 19h, de segunda a sexta-feira; e aos sábados, das 8h às 14h.
\item  Endereço: Rua Jacy Teixeira de Camargo, 940.
\end{itemize}
\end{itemize}

Fonte: \url{http://www.emdec.com.br/multas/multas_recursos_info_gerais.php}.

\subsection{Ônibus}
Se não tem condução própria, ou carona, pode utilizar o transporte coletivo de
Campinas. Os ônibus em Campinas são identificados por um número de três dígitos,
uma cor (azul claro, azul escuro, vermelho ou verde), uma figura geométrica
(círculo, quadrado, triângulo ou tetragrama, para que os portadores de
daltonismo possam identificar os ônibus) e um nome. A tarifa (uma das mais caras
do Brasil) foi reajustada no começo de 2011 e é R\$ 2,85. Apesar de existir
passe de estudante, universitários não tem direito ao desconto, apenas
estudantes de ensino fundamental e médio.

Existe quatorze linhas de ônibus que ligam o centro, alguns distritos, bairros
e terminais de Campinas ao distrito de Barão Geraldo. Algumas linhas desembarcam
os passageiros no terminal de Barão Geraldo, de onde seguem em outros ônibus
para a universidade. Já outras linhas vão direto para o campus. As trocas de
ônibus dentro do Terminal de Barão Geraldo são gratuitas, o que não ocorre em
outros terminais (no Terminal Mercado, por exemplo).

Em 2008 foi construída a nova e moderna rodoviária de Campinas, o terminal
multimodal Ramos de Azevedo. Além de transporte interestadual e municipal,
o espaço agrega um terminal metropolitano com o intuito de ligar as cidades da
Região Metropilitana de Campinas. Algumas linhas de ônibus internos de Campinas
também passam neste Terminal, como a linha 3.32.

Veja as linhas de ônibus disponíveis abaixo:

\begin{itemize}
\item  \textbf{1.34 -- Terminal Barão Geraldo (Inclusivo):} Sai do terminal do Ouro Verde, passa próximo ao Shopping Unimart, às avenidas Luís Smânio, Brasil, Theodureto de Almeida Camargo, Albino José Barbosa de Oliveira e vai para o terminal de Barão.
\end{itemize}

\begin{itemize}
\item  \textbf{2.10 -- Terminal Campo Grande / Shop. Dom Pedro / Terminal Barão Geraldo:} Sai do Terminal Campo Grande, passa pelo Shopping Dom Pedro, Terminal Barão Geraldo, avenida 1, rua Roxo Moreira (em frente à Reitoria), de novo pelo Shopping Dom Pedro, voltando para o Terminal Campo Grande. Saindo do terminal de Barão, essa linha passa na rua Roxo Moreira e próximo à area de biológicas e da saúde (Hospital das Clínicas e Faculdade de Ciências Médicas). Passa apenas na região da reitoria e hospital.
\end{itemize}

\begin{itemize}
\item  \textbf{2.66 -- Terminal Padre Anchieta / Hospital das Clínicas:} Sai do Terminal Padre Anchieta, passa pelo Makro, Shopping e rodovia D. Pedro, em frente ao Terminal Barão Geraldo (mas não entra), avenida 2, UNICAMP, PUC, Rodovia D. Pedro I voltando para o Terminal Padre Anchieta.
\end{itemize}

\begin{itemize}
\item  \textbf{2.69 -- Terminal Padre Anchieta / Terminal Barão Geraldo:} Assim como o 2.66, sai do Terminal Padre Anchieta, mas faz um trajeto diferente do 2.66, indo até o terminal de Barão. O intervalo entre os ônibus é de 90 minutos em dias úteis, e de 100 minutos nos sábados, domingos e feriados.
\end{itemize}

\begin{itemize}
\item  \textbf{3.00 -- Sousas / Terminal Barão Geraldo:} Sai de Sousas (um dos distritos de Campinas, assim como Barão Geraldo), passa pelo Shopping Galleria, pelo terminal do Shopping Dom Pedro, rodovias Heitor Penteado e Dom Pedro, tapetão e vai em direção ao terminal de Barão Geraldo. O problema é que o intervalo entre os ônibus é de 45 minutos.
\end{itemize}

\begin{itemize}
\item  \textbf{3.21 -- Centro Médico/Bosque das Palmeiras:} Sai do Terminal Barão e leva à Cidade Universitária II, passando pela avenida 2 e em frente ao Centro Médico.
\end{itemize}

\begin{itemize}
\item  \textbf{3.28 -- Guará:} Assim como o 3.21 leva à Cidade Universitária II, mas vai pela estrada da Rhodia até a Cidade Universitária e segue até o Guará.
\end{itemize}

\begin{itemize}
\item  \textbf{3.29 -- Terminal Barão Geraldo / Cidade Judiciária:} Sai do terminal de Barão, passa pela avenida 2, pela UNICAMP, por algumas ruas e avenidas do bairro fazenda Santa Cândida e vai até a estação da Cidade Judiciária. O intervalo entre ônibus pode variar de 27 a 40 minutos nos dias úteis, e é de 40 minutos aos sábados, domingos e feriados.
\end{itemize}

\begin{itemize}
\item  \textbf{3.30 -- UNICAMP / Hospital das Clínicas:} Do Terminal Central de Campinas à UNICAMP, passando pela rótula e avenidas Moraes Salles, Orosimbo Maia, Anchieta (prefeitura), Brasil e tapetão. Funciona das 6 às 19h30, a cada 15 minutos em média (depende do horário), de segunda a sexta-feira (não funciona nos sábados, domingos e feriados). Para ir do centro até a UNICAMP e da UNICAMP até o centro, essa linha é mais rápida que o 3.32, só que quase sempre os ônibus dessa linha estão cheios.
\end{itemize}

\begin{itemize}
\item  \textbf{3.31 -- Terminal Barão Geraldo / Rodoviária:} Sai do terminal Barão Geraldo e passa na rodoviária. É a opção mais rápida saindo de barão, levando cerca de 30 min (sem trânsito) em comparação com os mais de 40 minutos do 3.32. Passa também pelo Cambui.
\end{itemize}

\begin{itemize}
\item  \textbf{3.32 -- Terminal Barão Geraldo / Hospital das Clínicas / Rodoviária (Inclusivo):} Do Terminal Metropolitano até a UNICAMP e depois para o terminal Barão Geraldo, passando pela rótula, pela rodoviária nova e pelas avenidas Campos Sales, Anchieta (prefeitura), Orozimbo Maia, Brasil, Theodureto de Almeida Camargo e outras. Funciona das 6 às 23 horas, a cada 15 minutos em média, todos os dias. Essa linha vai diretamente para a UNICAMP, porém demora muito para ir do terminal metropolitano/centro até a UNICAMP e da UNICAMP até o centro/terminal metropolitano e quase sempre os ônibus dessa linha estão muito cheios, apesar de que, ao chegar ou sair da UNICAMP eles estão com poucos passageiros. No sentido terminal de Barão Geraldo -- terminal metropolitano, esse ônibus para no ponto localizado ao lado do IC-1.
\end{itemize}

\begin{itemize}
\item  \textbf{3.33 -- Terminal Barão Geraldo / Circular Rótula:} Do Terminal Barão Geraldo ao centro de Campinas, passando pela rótula, Orosimbo Maia, Anchieta (prefeitura) e pelo Terminal Central. Não passa pela UNICAMP, então para chegar à UNICAMP, deve pegar a linha 3.29, 3.32 ou 3.87. Funciona das 5h30 às 23h30, a cada 10 minutos, todos os dias.
\end{itemize}

\begin{itemize}
\item  \textbf{3.37 -- Hospital das Clínicas:} Do Terminal Barão Geraldo ao HC, passando pela UNICAMP. Funciona das 5h30 às 23h30, a cada 15 minutos, de segunda a sexta-feira. Passa pelo IC aos domingos.
\end{itemize}

\begin{itemize}
\item  \textbf{3.38 -- Terminal Barão Geraldo / Shopping D. Pedro / Shopping Iguatemi:} Linha que passa pelos dois principais shoppings da cidade. Funciona das 5:50 às 23:15, de segunda à sábado, a cada 30 minutos e em domingos e feriados funciona das 9:00 às 21:40, a cada 40 minutos.
\end{itemize}

E se pintar qualquer dúvida é só entrar no site da EMDEC
(\url{http://www.emdec.com.br/}) ou da TRANSURC
(\url{http://www.transurc.com.br/}) para ver os horários e a trajetória de todas
as linhas de Campinas.

\subsubsection{Bilhete Único}
O bilhete único foi implantado com o objetivo de facilitar o transporte daqueles
que se utilizam de ônibus. No prazo de 1 hora e meia (de segunda a sábado) e de
2 horas (nos domingos e feriados), o usuário pode utilizar até três ônibus
pagando apenas uma passagem. Para quem usa mais de três ônibus e quer aumentar
o número de integrações, tem de ir à sede da TRANSURC, localizado na rua Onze de
Agosto, 757, Centro.

Para adquiri-lo, dirija-se ao Terminal Barão Geraldo, Central, Ouro Verde, Campo
Grande ou Mercado, munido de seu RG e CPF. Você preencherá um cadastro
e retornará após alguns dias para retirar seu cartão. Para a primeira recarga,
exige-se o pagamento de duas tarifas (o que atualmente fica em R\$ 5,70).

Para recarregar o cartão, além dos terminais, também tem diversos
estabelecimentos comerciais credenciados a fazer a recarga do cartão, que podem
ser vistos na página
\url{http://www.transurc.com.br/2007/Site/Informacoes/RedeCredenciada.aspx}.

\section{Outras necessidades}
\subsection{Supermercados}
No centro de Barão há três supermercados: o Super Barão (conhecido também como
Super Ladrão ou Super Carão), o Pão-de-Açúcar e o Dalben. Ambos Super Barão
e Pão-de-Açũcar são muito caros, mas o Barão tem boas promoções de segunda
e terça-feiras. Cada um é melhor/mais barato para alguma coisa. Só indo bastante
em cada um é que você pega o jeito. O Dalben é novo, abriu em 2007, é maior
e geralmente tem coisas mais baratas que o Pão-de-Açúcar. Às segundas e terças
(no Dalben e no SuperBarão) e às terças (no Pão de Açúcar) há preço promocional
de frutas e verduras. O Dalben localiza-se no alto da Avenida 1, na entrada de
Barão Geraldo. O Barão funciona de segunda a sábado das 7h às 22h e domingos das
8h às 20h e o Pão de Açúcar funciona de domingo a quinta-feiradas 7h às 23h,
sextas e sábados das 7h às 24h. Na moradia, há outro Super Barão e o Benatti,
e no Real Parque outra loja do Super Barão.

Há também lojas de conveniência (como as AM/PM em frente à UNICAMP e no Rio das
Pedras, o StarMart no Texaco do fim da avenida 2 e a do Shell em frente ao
terminal, a mais barata das 4) que embora mais caras que os supermercados, podem
funcionar em horários que esses não e/ou serem mais perto de onde você mora.

Em frente ao terminal Barão há o Recanto Bela Fruta, que não é exatamente um
supermercado, mas é o melhor lugar de Barão para comprar frutas, verduras
e legumes. Vende também algumas coisas de supermercado como pão, leite, iogurte,
mupy etc. Neste mesmo esquema funciona o varejão Oba, que fica na avenida Santa
Isabel, perto da pizzaria Sapore. Na estrada da Rhodia, depois da Padaria Di
Capri, há a frutaria Rio das Pedras, que tem preços no nível do Pão de Açúcar
e do Super Barão, mas é mais próximo pra quem vive na Cidade Universitária 2,
por exemplo.

Se você tiver carro, ou conhecer alguém que tenha, junte um pessoal e faça suas
compras no Tenda Atacado ou no Atacadão. Os dois são muito mais baratos que
qualquer supermercado e tem MUITA coisa. Ah, e nem tudo é atacado, vende umas
coisas a varejo lá também. Para chegar aos dois, saindo de Barão pelo Tapetão,
pegue a D. Pedro sentido Anhanguera. Assim que você entrar na D. Pedro, você já
vai enxergar o Atacadão, à direita. Para ir ao Tenda siga mais um pouco na D.
Pedro e saia logo depois do CEASA à direita.

Outras opções para quem tem carro são: O Carrefour, localizado na Rodovia D.
Pedro e o Wal-Mart, que fica no Shopping Parque D. Pedro.

\subsection{Utilidades em geral}
\begin{itemize}
\item  \textbf{Ki-Água}
\begin{itemize}
\item  Telefone: (19) 3289-4659
\item  Endereço: Rua Eduardo Modesto, 240
\end{itemize}
\end{itemize}

\begin{itemize}
\item  \textbf{Circuito das Águas}
\begin{itemize}
\item  Telefone: (19) 3289-4930
\item  Endereço: Rua Júlia Leite de Barros, 182
\end{itemize}
\end{itemize}

\begin{itemize}
\item  \textbf{Água Mineral Serrana}
\begin{itemize}
\item  Telefone: (19) 3289-4602
\item  Endereço: Rua Albino José Barbosa de Oliveira, 658
\end{itemize}
\end{itemize}

\begin{itemize}
\item  \textbf{Real Gás}
\begin{itemize}
\item  Telefone: (19) 3289-1786
\item  Endereço: Rua Eduardo Pereira de Almeida, 570
\end{itemize}
\end{itemize}

\begin{itemize}
\item  \textbf{Genebra Gás}
\begin{itemize}
\item  Telefone: (19) 3289-3622
\item  Endereço: Avenida Santa Isabel
\end{itemize}
\end{itemize}

\begin{itemize}
\item  \textbf{Drogaria Vitória}
\begin{itemize}
\item  Telefone: (19) 3289-9926
\item  Endereço: Avenida Ruberley Bueretto da Silva, 1015
\end{itemize}
\end{itemize}

\begin{itemize}
\item  \textbf{Drogaria Nova Barão}
\begin{itemize}
\item  Telefone: (19) 3289-6191
\end{itemize}
\end{itemize}

\begin{itemize}
\item  \textbf{Rede Farmaxima}
\begin{itemize}
\item  Telefone: (19) 3289-2824 / (19)3289-4054
\item  Endereço: Avenida Dr. Romeu Tórtima, 255
\end{itemize}
\end{itemize}

\begin{itemize}
\item  \textbf{Drogaria Sidarta}
\begin{itemize}
\item  Telefone: (19) 3289-1999
\item  Endereço: Avenida Prof. Atílio Martini, 190
\end{itemize}
\end{itemize}

\begin{itemize}
\item  \textbf{New Laundry Lavanderia}
\begin{itemize}
\item  Telefone: (19) 3289-3922
\item  Endereço: Rua Francisca Resende Merciai, 54
\end{itemize}
\end{itemize}

\begin{itemize}
\item  \textbf{Desentupimento 24 horas}
\begin{itemize}
\item  Telefone: (19) 3242-1249
\end{itemize}
\end{itemize}

\begin{itemize}
\item  \textbf{Carpintaria São Jorge de Barão}
\begin{itemize}
\item  Telefone: (19) 3289-3399
\item  Endereço: Avenida Santa Isabel, 1882
\end{itemize}
\end{itemize}

\begin{itemize}
\item  \textbf{CPFL (companhia de luz)}
\begin{itemize}
\item  Telefone: 0800-010-1010
\end{itemize}
\end{itemize}

\begin{itemize}
\item  \textbf{Sanasa (companhia de água)}
\begin{itemize}
\item  Telefone: 0800-772-1195
\end{itemize}
\end{itemize}

\begin{itemize}
\item  \textbf{Luigi Cabelereiro}
\begin{itemize}
\item  Telefone: (19) 3288-0198
\end{itemize}
\end{itemize}

\begin{itemize}
\item  \textbf{João Cabelereiros}
\begin{itemize}
\item  Telefone: (19) 3289-3084
\item  Endereço: Rua Orácio Leonardi, 92
\end{itemize}
\end{itemize}

\begin{itemize}
\item  \textbf{Armazém da Foto}
\begin{itemize}
\item  Telefone: (19) 3289-0975
\end{itemize}
\end{itemize}

\begin{itemize}
\item  \textbf{Empresa de Correios e Telégrafos (Correios)}
\begin{itemize}
\item  Endereço 1: Avenida Santa Isabel, 218 -- Centro de Barão.
\item  Telefone 1: (19) 3288-0244.
\item  Endereço 2: Rua Carlos Gomes, 241 -- Campus da UNICAMP (próximo ao ginásio e Instituto de Artes).
\item  Telefone 2: (19) 3289-7288 e (19) 3521-7642.
\end{itemize}
\end{itemize}

\subsection{Bancos}
\begin{itemize}
\item  \textbf{Banco do Brasil:} Com a compra da Nossa Caixa pelo BB, há 3 agências: Uma localizada perto da Reitoria, outra no centro de Barão, perto do Santander, e outra no centro de Barão perto de onde era o Super Barão. Há caixa eletrônico onde antigamente era a agência do BB perto dos Correios.
\end{itemize}

\begin{itemize}
\item  \textbf{Bradesco:} Agência no centro de Barão, do lado do Banco do Brasil e em frente do Santander. Próxima ao Terminal Barão.
\end{itemize}

\begin{itemize}
\item  \textbf{Caixa Econômica Federal:} Localizada na frente do Pão de Açúcar.
\end{itemize}

\begin{itemize}
\item  \textbf{Citibank:} Localizado no alto da avenida 2, próximo ao posto Texaco.
\end{itemize}

\begin{itemize}
\item  \textbf{Itaú:} Possui uma agência na Unicamp, próximo à Reitoria e a ex agência do Banco Real (agora agência do Santander), e outra na avenida Santa Isabel, do lado da sorveteria. Existe também o Itaú Personalité, próximo ao Pão de Açuar e ao McDonalds.
\end{itemize}

\begin{itemize}
\item  \textbf{Santander:} Com a compra do Banco Real pelo Santander, há quatro agêncais em Barão: Duas localizadas ao lado da Reitoria (uma delas era a agência do Real), outra na praça do Ciclo Básico e a outra próxima ao Terminal Barão. Há também caixas eletrônicos, ligados à rede 24 horas, espalhados pela UNICAMP: Na entrada do IMECC, na entrada da FEF, na entrada do IE, na Reitoria, ao lado do xerox do IFGW e no Ciclo Básico.
\end{itemize}

\begin{itemize}
\item  \textbf{Unibanco:} Uma agência na rua de onde era o Super Barão, próximo ao BB e ao Santander do centro.
\end{itemize}

\subsection{Sebos e livrarias}
Em Barão há três sebos: O Curupira, o Cronópio e o Galpão. Geralmente você não
encontra muita coisa boa de computação neles, mas não custa procurar. O Curupira
fica na rua do Terminal, bem em frente a ele. Não é difícil achar. O Galpão fica
perto do Terminal. Saindo do terminal pela avenida marginal à Albino de
Oliveira, vire a primeira à esquerda e a segunda à direita. Funciona de segunda
à sexta até às 18h. O Cronópio fica na mesma rua do Galpão, só que bem longe,
próximo à padaria Fiori. Seguindo a Santa Isabel, vindo do Centro de Barão, vire
à esquerda na esquina que tem uma pizzaria (antes da Fiori). Vire na primeira
à esquerda e você está no Cronópio (Lá também é um restaurante barato
e gostoso).

Na UNICAMP há a livraria Toledo, que fica na Faculdade de Educação (Pedago), tem
pouca coisa de Computação lá mas alguma coisa de Matemática. Também há
a livraria da UNICAMP (no IEL), tem preços bons mas não tem quase nada de
Exatas. Já a Livraria da Química tem livros de exatas, e geralmente eles
conseguem importar a um preço bom. Outras livrarias são a Fnac (no Shopping D.
Pedro), a Saraiva e a Cultura (no Shopping Iguatemi) e a Siciliano (no Shopping
Galeria).

Mais uma dica: Antes de comprar um livro, veja se você vai usar muito ele, se
não tem bastante na biblioteca, se algum veterano legal não pode te emprestar ou
te vender, ou se não rola tirar xerox. Você compra livro só se você precisar
e/ou quiser muito. Caso vá comprar, atente para livrarias virtuais, que podem
ter preços muito menores (pesquise sempre no Buscapé
(\url{http://www.buscape.com.br/})) e emitem boletos para você pagar no banco.
Em alguns casos, a livraria da Editora (localizada no piso térreo da BC) pode
trazê-lo por um preço menor ainda -- consulte sempre!

Uma última dica: Não compre o livro de G.A. (MA141). Estude pelo Stewart II
(livro de Cálculo II), pelo Steinbruch (ou algo assim) ou pelo livro do Paulo
Boulos. Os três são muito melhores que ele. Se você quiser os exercícios para
estudar para os testes pegue na internet ou tire xerox. Use os outros livros
para estudar.

\begin{itemize}
\item  \textbf{Sebo Curupira}
\begin{itemize}
\item  Endereço: Avenida Albino José B. de Oliveira, 980
\item  Telefone: (19) 3289-7522
\end{itemize}
\end{itemize}

\begin{itemize}
\item  \textbf{Sebo Galpão}
\begin{itemize}
\item  Endereço: Rua Francisco Barros Filho, 16
\item  Telefone: (19) 3289-2044
\end{itemize}
\end{itemize}

\begin{itemize}
\item  \textbf{Sebo Valise de Cronópio}
\begin{itemize}
\item  Endereço: Rua Francisco de Barros Filho, 426
\item  Telefone: (19) 3289-0028
\end{itemize}
\end{itemize}

\subsection{Bicicletarias}
\begin{itemize}
\item  \textbf{Bicicletaria Barão}
\begin{itemize}
\item  Endereço: Avenida Santa Isabel, 446.
\item  Telefone: (19) 3249-0449.
\end{itemize}
\end{itemize}

\begin{itemize}
\item  \textbf{Rei do Pedal}
\begin{itemize}
\item  Endereço: Avenida Santa Isabel, 74.
\item  Fone: (19) 3289-9258.
\end{itemize}
\end{itemize}

\begin{itemize}
\item  \textbf{Via Bike}
\begin{itemize}
\item  Endereço: Avenida Albino José Barbosa de Oliveira, 1074.
\item  Telefone: (19) 3289-9888.
\end{itemize}
\end{itemize}

\subsection{Escolas de idiomas}
\begin{itemize}
\item  \textbf{CCAA}
\begin{itemize}
\item  Endereço: Rua Professor Luciano Venere Decourt, 290.
\item  Telefone: (19) 3249-0202.
\item  E-mail: barao [@] ccaa [.] com [.] br
\item  Site: \url{http://www.ccaa.com.br/barao}
\end{itemize}
\end{itemize}

\begin{itemize}
\item  \textbf{CCBEUC -- Centro Cultural Brasil Estados Unidos -- Campinas}
\begin{itemize}
\item  Endereço: Avenida Romeu Tórtima, 531.
\item  Telefone: (19) 3249-0275.
\item  E-mail: admbg [@] ccbeuc [.] com [.] br
\item  Site: \url{http://www.ccbeuc.com.br/}
\end{itemize}
\end{itemize}

\begin{itemize}
\item  \textbf{CNA}
\begin{itemize}
\item  Endereço: Avenida Dr. Romeu Tortima, 553.
\item  Telefone: (19) 3289-4700.
\item  Fax: (19) 3289-4700.
\item  E-mail: baraogeraldo [@] cna [.] com [.] br
\item  Site: \url{http://www.cna.com.br/baraogeraldo}
\end{itemize}
\end{itemize}

\begin{itemize}
\item  \textbf{HAVAD}
\begin{itemize}
\item  Endereço: Avenida Romeu Tórtima, 522.
\item  Telefone: (19) 3288-0012.
\item  Fax: (19) 3249-0488.
\item  Email: havad [@] havad [.] com [.] br
\item  Site: \url{http://www.havad.com.br/}
\end{itemize}
\end{itemize}

\begin{itemize}
\item  \textbf{In Touch}
\begin{itemize}
\item  Endereço: Rua Antônio Augusto de Almeida, 517 -- Cidade Universitária.
\item  Telefone: (19) 3289-3481.
\item  E-mail: secretaria [@] intouch [.] art [.] br
\item  Site: \url{http://www.intouch.art.br/}
\end{itemize}
\end{itemize}

\begin{itemize}
\item  \textbf{INOVA -- Escola de Inglês}
\begin{itemize}
\item  Endereço: Avenida Romeu Tórtima, 391.
\item  Telefone: (19) 3288-0071.
\item  E-mail: fale [@] inovalinguas [.] com [.] br
\item  Site: \url{http://www.inovalinguas.com.br}
\end{itemize}
\end{itemize}

\begin{itemize}
\item  \textbf{Real Time}
\begin{itemize}
\item  Endereço: Rua Agostinho Páttaro, 47 (esquina com a do praça do coco).
\item  Telefone: (19) 3289-6240.
\item  E-mail: barao [@] rtidiomas [.] com [.] br
\item  Site: \url{http://www.realtimeenglish.com.br/}
\end{itemize}
\end{itemize}

\begin{itemize}
\item  \textbf{Wizard}
\begin{itemize}
\item  Endereço: Rua João Batista Antonioli, 90.
\item  Telefone: (19) 3289-6199.
\end{itemize}
\end{itemize}

\begin{itemize}
\item  \textbf{Yazigi}
\begin{itemize}
\item  Endereço: Avenida Dr. Romeu Tórtima, 500.
\item  Telefone: (19) 3249-2375.
\item  Site: \url{http://www.yazigi.com.br/}
\end{itemize}
\end{itemize}

\subsection{Igrejas}
\begin{itemize}
\item  \textbf{IBCU -- Igreja Batista da Cidade Universitária}
\begin{itemize}
\item  Endereço: Rua Tenente Alberto Mendes Jr., 5.
\item  Telefone: (19) 3289-4501.
\item  Site: \url{http://www.ibcu.org.br}
\item  E-mail: jovens [@] ibcu [.] org [.] br
\end{itemize}
\end{itemize}

\begin{itemize}
\item  \textbf{IBBG -- Igreja Batista de Barão Geraldo}
\begin{itemize}
\item  Endereço: Rua Luiz Vicentim, 284.
\item  Telefone: (19) 3289-1793.
\end{itemize}
\end{itemize}

\begin{itemize}
\item  \textbf{IPBG -- Igreja Presbiteriana de Barão Geraldo}
\begin{itemize}
\item  Endereço: Rua Francisco Andreo Aledo, 141 (próximo à praça do coco e moradia).
\item  Telefone: (19) 3289-3239.
\end{itemize}
\end{itemize}

\begin{itemize}
\item  \textbf{Igreja do Nazareno Betânia}
\begin{itemize}
\item  Endereço: Rua Manoel Antunes Novo, 98.
\item  Telefone: (19) 3289-7379.
\end{itemize}
\end{itemize}

\begin{itemize}
\item  \textbf{Assembleia de Deus -- Ministério de Belém}
\begin{itemize}
\item  Endereço: Rua Júlia Leite de Barros, 54 (próximo à moradia da UNICAMP).
\item  Telefone: (19) 3249-0035.
\item  Site: \url{http://www.adbarao.com.br}
\end{itemize}
\end{itemize}

\begin{itemize}
\item  \textbf{Paróquia Santa Isabel}
\begin{itemize}
\item  Endereço: Rua Benedito Alves Aranha, 226.
\item  Telefones: (19) 3289-1101 e (19) 3289-2323
\item  Site: \url{http://www.paroquiasantaisabel.org.br}
\end{itemize}
\end{itemize}

\begin{itemize}
\item  \textbf{Comunidade do Estudante Universitário}
\begin{itemize}
\item  Endereço: Rua Dr. Ruberlei Boaretto da Silva, 785.
\item  E-mail: ceu [.] campinas [@] gmail [.] com
\end{itemize}
\end{itemize}

\subsection{Postos de combustível}
\begin{itemize}
\item  \textbf{Auto Posto Campineira}
\begin{itemize}
\item  Endereço: Avenida Albino Jose Barbosa de Oliveira, 1480.
\item  Telefone: (19) 3289-5991.
\end{itemize}
\end{itemize}

\begin{itemize}
\item  \textbf{Auto Posto Barbieri de Barão Geraldo}
\begin{itemize}
\item  Endereço: Avenida Albino Jose Barbosa de Oliveira, 1001 (próximo ao terminal).
\item  Telefone: (19) 3289-1917 e (19) 9768-6713.
\end{itemize}
\end{itemize}

\begin{itemize}
\item  \textbf{Centro Automotivo Cidade Universitária LTDA}
\begin{itemize}
\item  Endereço: Avenida Doutor Romeu Tortima, 1541.
\item  Telefones: (19) 3289-8457, (19) 3289-9934 e (19) 3289-9199.
\end{itemize}
\end{itemize}

\begin{itemize}
\item  \textbf{Transo Combustíveis LTDA}
\begin{itemize}
\item  Endereço: Avenida Santa Izabel, 1030 (próximo à moradia).
\item  Telefone: (19) 3289-1012.
\end{itemize}
\end{itemize}

\begin{itemize}
\item  \textbf{Esso Auto Posto Futuro}
\begin{itemize}
\item  Endereço: Avenida Albino Jose Barbosa Oliveira, 360 (logo na entrada do distrito de Barão).
\item  Telefone: (19) 3289-4332.
\end{itemize}
\end{itemize}

\begin{itemize}
\item  \textbf{Posto Vô João}
\begin{itemize}
\item  Endereço: Avenida Albino José Barbosa de Oliveira, 2151.
\item  Telefones: (19) 3289-3388 e (19) 3289-6594.
\item  Site: \url{http://www.vojoao.com/}
\end{itemize}
\end{itemize}

OBS: Dentro do campus há um posto de combustível (próximo à descida da FEAGRI),
mas atende apenas veículos oficiais.

\section{SIGLAS malucas}
Nessa sessão serão dadas algumas explicações sobre algumas siglas, códigos
e outros termos muito usados dentro da Unicamp. Aí vão eles:

\begin{itemize}
\item  \textbf{CCG (Comissão Central de Graduação):} Órgão colegiado da Unicamp, é encarregada da orientação, supervisão e revisão periódica do ensino na Universidade. Cabe recurso à CCG de quaisquer decisões das Unidades afetando o ensino.
\end{itemize}

\begin{itemize}
\item  \textbf{CCPG (Comissão Central de Pós-Graduação):} Órgão colegiado da Unicamp, é encarregada da orientação, supervisão e revisão periódica da pós-graduação na Universidade. Cabe recurso à CCPG de quaisquer decisões das Unidades afetando o ensino.
\end{itemize}

\begin{itemize}
\item  \textbf{Consu (Conselho Universitário):} O Conselho Universitário é o órgão máximo da Universidade, para você entender, seria como o Parlamento Inglês. Assim como o Parlamento manda mais que a rainha, o Consu manda mais que o reitor (embora o reitor faça parte dele e influencie fortemente suas decisões). Existe representação discente no Consu, eleita por um processo realizado pela Coordenadoria Geral da Universidade. Até 2003 esse processo era realizado pelos próprios estudantes e há uma longa briga para reavê-lo (tanto que em 2005 o DCE realizou eleições paralelas para o Consu e a CCG).
\end{itemize}

\begin{itemize}
\item  \textbf{Congregação:} É o órgão colegiado do Instituto ou Faculdade. Cabe recurso à Congregação da Unidade de Ensino de quaisquer decisões dos Departamentos e das Coordenações de Curso.
\end{itemize}

\begin{itemize}
\item  \textbf{Departamento:} É administrado por um professor-chefe e um Conselho Departamental, é a menor unidade administrativa, didática e científica da Universidade, sendo responsável pelo desenvolvimento dos programas de ensino, pesquisa e extensão dos serviços à comunidade. Todo instituto e faculdade da universidade possui o seu conjunto de departamentos, conhecidos através de siglas.
\end{itemize}

\begin{itemize}
\item  \textbf{CI (Conselho Interedepartamental):} Este é um "braço" da congregação, responsável por tratar de assuntos menores, como despesas e atribuições de sala. Fazem parte deste órgão, além de um representante discente, o diretor do instituto, os coordenadores e os chefes de departamentos. Não existe na FEEC, sendo que seu papel é desempenhado pelo diretor da faculdade.
\end{itemize}

\begin{itemize}
\item  \textbf{CDI (Comissão Diretora de Informática):} Outro braço da congregação, responsável por tratar de assuntos relacionados aos ambientes computacionais, deliberando sobre a atualização de infraestrutura, a organização da rede, endereços de internet e similares.
\end{itemize}

\begin{itemize}
\item  \textbf{CG (Coordenadoria/Comissão de Graduação):} É o órgão da unidade responsável pelos seus cursos de graduação. Sempre que houver algum problema ou deficiência no curso, é este órgão que vocês devem procurar. Cada curso tem um coordenador (que faz parte da CG), sendo que, atualmente, o professor Sandro Rigo é o coordenador da Ciência e os professores Ricardo Torres (IC) e o José Mario (FEEC).
\end{itemize}

\begin{itemize}
\item  \textbf{CPG (Coordenadoria/Comissão de Pós-Graduação):} Este é o órgão responsável pela pós-graduação no Instituto, coordenando as disciplinas oferecidas e as matrículas na pós. O coordenador atual é o professor Alexandre Falcão. Os alunos tem direito a voz e voto nos colegiados (instâncias decisórias compostas por várias pessoas) da Unicamp, tendo representação discente em número correspondente a um quinto (1/5) dos membros. Geralmente, os representantes discentes são eleitos pelos estudantes ou indicados pelos centros acadêmicos. O exercício da representação estudantil e atividades decorrentes não exonera o aluno da freqüência nas atividades escolares, com exceção da participação em reuniões em órgãos colegiados, nos horários em que estes se reúnem para deliberar.
\end{itemize}

\begin{itemize}
\item  \textbf{DCE (Diretório Central de Estudantes):} É a entidade de representação dos estudantes de graduação da Unicamp, competindo-lhe ainda designar representantes estudantis para os órgãos colegiados da Universidade.
\end{itemize}

\begin{itemize}
\item  \textbf{DAC (Diretoria Acadêmica):} É o órgão executivo e informativo, incumbido do registro e controle das atividades discentes da UNICAMP. Cuida das matrículas, alteração de matrícula, emissão de documentos e relatórios, como o histórico escolar, realiza reserva de salas, entre outras atividades.
\end{itemize}

\begin{itemize}
\item  \textbf{SAE (Serviço de Apoio ao Estudante):} É encarregado da execução de programas de assistência desenvolvidas pela Universidade, por iniciativa própria ou mediante convênios firmados com entidades especializadas.
\end{itemize}

\begin{itemize}
\item  \textbf{Crédito:} Unidade elementar de horas-aula de qualquer curso da UNICAMP. Um crédito equivale a uma hora-aula semanal, ou a 15 horas-aula semestrais.
\end{itemize}

\begin{itemize}
\item  \textbf{Período letivo:} É um nome complicado para se referir ao semestre.
\end{itemize}

\begin{itemize}
\item  \textbf{Currículo pleno:} É o conjunto de disciplinas do curso que o aluno tem que cursar.
\end{itemize}

\begin{itemize}
\item  \textbf{CR (Coeficiente de Rendimento):} Valor entre 0 e 1 que é a média ponderada das notas obtidas em todas as disciplinas até o momento. É calculada usando como pesos o número de créditos de cada disciplina.
\end{itemize}

\begin{itemize}
\item  \textbf{CP (Coeficiente de Progressão):} É a porcentagem do curso que você já cumpriu. Por exemplo, se você tem CP = 0,6123 significa que você cumpriu 61,23\% do curso. Você se forma quando o seu CP for 1 (100\% do curso completo). É importante saber o CP quando for fazer algum estágio, ou um TCC (Trabalho de Conclusão de Curso), ou quando for cursar disciplinas que tenham como pré-requisito AA4xy.
\end{itemize}

\begin{itemize}
\item \begin{itemize}
\item  \textbf{CPF (Coeficiente de Progressão Futuro):} Além do CP, também tem o CPF, que além do nome de um documento é o CP que você terá no fim do semestre caso passe em todas as disciplinas.
\end{itemize}
\end{itemize}

\begin{itemize}
\item \begin{itemize}
\item  \textbf{CPE (Coeficiente de Progressão Exigido):} Além do CP e do CPF há o CPE. O CPE foi instituído a partir de 2005 e é usado para fins de cancelamento, ou não, de matrícula. Para que o aluno possa continuar a fazer o curso, ele precisa ter um CP maior ou igual ao CPE daquele semestre. Tanto o CP, como o CPE e o CPF existem somente nos cursos de graduação.
\end{itemize}
\end{itemize}

\begin{itemize}
\item  \textbf{Pré-requisito:} Matéria(s) que precisa(m) ter sido cursada(s) para que se possa fazer outra(s) matéria(s). Existem dois tipos de pré-requisitos: Os pré-requsitos totais, mais comuns, do qual é exigido tanto a aprovação por nota como por frequência e os pré-requisitos parciais, mais raros, do qual o aluno não precisa ter sido aprovado por nota, mas tem que ter tido aprovação por frequência e nota final maior ou igual a 3,0. Os pré-requisitos parciais são identificados com um asterisco na frente do código da disciplina (não confundir com um apontador).
\end{itemize}

\begin{itemize}
\item  \textbf{AA4xy:} Um tipo de pré-requisito, raríssimo. Não se trata de nenhuma disciplina. Para fazer disciplinas com esse pré-requisito, o aluno tem que tem um CP maior ou igual a 0,xy.
\end{itemize}

\begin{itemize}
\item  \textbf{AA200:} Outro tipo de pré-requisito existente, não tão raro, porém mais presente em disciplinas eletivas. Também não se trata de nenhuma disciplina. É apenas uma autorização da coordenadoria do curso. Se sobrar vagas para a disciplina e a coordenadoria do curso for com a sua cara você faz a disciplina.
\end{itemize}

\begin{itemize}
\item  \textbf{PB (Prédio Básico)} Também conhecido como Ciclo Básico II, é um prédio com várias salas de aula, que fica em frente ao Bandejão, e serve várias unidades que não possuem espaço físico suficiente para comportar seus alunos. No segundo andar ficam as salas de aula (PB01 a PB12) e no terceiro andar ficam os auditórios (PB13 a PB18).
\end{itemize}

\begin{itemize}
\item  \textbf{CB (Ciclo Básico I):} Tem finalidade idêntica ao PB, só que é muito melhor equipado, tem uma acústica muito melhor e tem um ar condicionado capaz de matar esquimó de frio. Fica na mesma praça que o PB, só que no outro extremo. À esquerda da entrada ficam as salas ímpares e à direita ficam as salas pares. No primeiro andar ficam os auditórios (CB01 a CB06) que possuem 140 e 160 lugares e no segundo andar ficam as salas de aula (CB07 a CB18) que possuem 60 e 80 lugares.  O CB e o PB são os lugares onde você vai ter a maioria das suas aulas (especialmente nos dois primeiros anos de curso).
\end{itemize}

\subsection{Siglas de salas de aula}
A tabela abaixo contem algumas siglas de salas de aula que aparecem nos cadernos
de horários, disponibilizados pelas coordenadorias dos cursos e pela DAC.

\begin{tabularx}{\linewidth}{|Y|Y|Y|}\hline

\multicolumn{}{|l|}{ \textbf{Siglas e locais das Salas de Aula no horário}}\tabularnewline \hline

 \textbf{Sigla}  &  \textbf{Local}  &  \textbf{Referência}\tabularnewline \hline

 CB  &  Ciclo Básico I  &  Praça Central, atrás do Banespa, em frente da cantina da Física.\tabularnewline \hline

 CC  &  Instituto de Computação (IC)  &  Ao lado do Departamento de Artes Cênicas (IC-1); ao lado do IE (IC-2) e atrás do IE (IC-3).\tabularnewline \hline

 CI  &  Centro de Estudo de Línguas (CEL)  &  Atrás do IFCH.\tabularnewline \hline

 CL  &  Instituto de Estudos da Linguagem (IEL)  &  Em frente a praça central e ao lado do IFCH.\tabularnewline \hline

 EB  &  Engenharia Básica  &  Atrás da Praça da Paz e próximo à FEEC.\tabularnewline \hline

 EM  &  Faculdade de Engenharia Mecânica (FEM)  &  Atrás do IQ.\tabularnewline \hline

 FA  &  Faculdade de Engenharia de Alimentos (FEA)  &  Em frente ao IQ e à Praça da Paz.\tabularnewline \hline

 FE  &  Faculdade de Engenharia Elétrica e de Computação (FEEC)  &  Em frente à Praça da Paz.\tabularnewline \hline

 IB  &  Instituto de Biologia (IB)  &  Entre o IQ e o Serviço Social do SAE.\tabularnewline \hline

 IE  &  Instituto de Economia (IE)  &  Atrás do IMECC.\tabularnewline \hline

 IF  &  Instituto de Física (IFGW)  &  Em frente ao Ciclo Básico, e a Química.\tabularnewline \hline

 IH  &  Instituto de Filosofia e Ciências Humanas (IFCH)  &  Entre o IMECC e o IEL.\tabularnewline \hline

 IM  &  Instituto de Matemática, Estatística e Computação Científica (IMECC)  &  Em frente a Praça Central.\tabularnewline \hline

 IQ  &  Instituto de Química (IQ)  &  Entre o IB e o IFGW.\tabularnewline \hline

 LE  &  Laboratórios de informática da FEEC  &  Em frente a Praça da Paz e ao lado das salas de aula da FEEC.\tabularnewline \hline

 LF  &  Laboratório de Física  &  Em frente à cantina do IMECC.\tabularnewline \hline

 LQ  &  Laboratório de Química  &  Em frente à biblioteca do IQ.\tabularnewline \hline

 PB  &  Ciclo Básico II, Prédio Básico  &  Praça Central, em frente ao Bandejão.\tabularnewline \hline

\end{tabularx}

\section{Cancelamento, trancamento e desistência. Há alguma diferença?}
Existe várias coisas diferentes que o aluno pode fazer com a sua matrícula.
Aliás, o próprio processo de matrícula (em que o aluno pode escolher, e alterar
(\url{http://www.dac.unicamp.br/portal/grad/regimento/capitulo_iii/secao_iv/index.html}),
as disciplinas que ele quer fazer durante o semestre) já é algo diferente. Essas
coisas são a desistência, cancelamento e trancamento de matrícula.

Embora praticamente todos os alunos da UNICAMP usem esses três termos
"indiscriminadamente", como se fossem sinônimos, para a DAC, esses três termos
tem significados bastante distintos. Aí vai o que cada termo significa:

\subsection{GDE}
Bixo,você já entrou na universidade e agora vai começar a parte mais difícil:
sair da faculdade com um diploma na mão.  A Unicamp é muito diferente da sua
escolinha onde a tia Gertrudes entregava o seu horário impresso coloridinho pra
você colar na capa do seu fichário. Na Unicamp você vai ter que se virar, e vai
descobrir que pra montar o seu horário você precisa perder horas folheando
manuais de aluno, recomendações de horário, descobrindo matérias que tem
dependências e equivalentes... ou você pode usar o GDE.  o GDE
(\url{http://www.gde.ir}) é uma ferramenta que simplifica (MUITO) o seu
planejamento dentro da Unicamp, te dando de bandeja tudo que você vai precisar
pra enfrentar o dia-a-dia na Unicamp.  Ah, tudo isso dentro de uma rede social,
com chat e tudo, onde você pode ver o seu horário, o de seus amigos, o cardápio
do bandejão, avaliações sobre seus professores, calendário... tá fácil hein?  Ou
você pode começar por aqui:
\url{http://www.dac.unicamp.br/portal/grad/regimento/regimento_completo/}. Você
que sabe.

\begin{itemize}
\item  Desistência de matrícula em disciplinas (\url{http://www.dac.unicamp.br/portal/grad/regimento/capitulo_iii/secao_v/index.html}): Processo que é chamado pelos alunos de "trancamento", ou de "trancamento de disciplina", termo aliás que não existe. Consiste de desistir de uma determinada disciplina. O aluno não mais cursa essa disciplina no semestre, tendo de cursá-la em algum semestre posterior. Só pode desistir uma vez da disciplina e pode-se pedir desistência até que se tenha passado 1/2 do semestre.
\item  Cancelamento de matrícula (\url{http://www.dac.unicamp.br/portal/grad/regimento/capitulo_iii/secao_vii/index.html}): Processo em que o aluno se desliga da UNICAMP, por motivo de jubilação, por ter faltado às duas primeiras semanas do ano de ingresso, por ter sido reprovado em todas as disciplinas do primeiro ou do segundo semstre de ingresso, por ter sido expulso, por ter sido aprovado em outra universidade pública (não é permitido fazer mais do que um curso de universidade pública simultaneamente), ou por vontade própria do aluno.
\item  Trancamento de matrícula (\url{http://www.dac.unicamp.br/portal/grad/regimento/capitulo_iii/secao_vi/index.html}): Processo em que o aluno não cursa qualquer disciplina da UNICAMP durante o semestre. Quando, simplesmente, se fala "trancamento", refere-se a esse processo. O aluno tem direito a fazer até dois trancamentos de matrícula, em semestres seguidos ou não e o aluno não pode trancar nenhum dos dois semestres do ano de ingresso. Desistência de todas as disciplinas configura-se como trancamento. O trancamento é pedido na DAC, e pode ser pedido até que se tenha transcorrido 2/3 do semestre (geralmente de dezembro até fim de maio para trancamento de primeiro semestre; e de julho até fim de outubro para trancamento de segundo semestre). Para cada trancamento, o prazo máximo de integralização é postergado.
\end{itemize}

\section{Lugares para estudar}
Há vários lugares para estudar na UNICAMP. Você pode escolher o que você achar
melhor:

\begin{itemize}
\item  \textbf{Biblioteca Central (BC):} A BC tem três andares. O primeiro é onde tem os livros gerais e onde a galera estuda. Geralmente é barulhento em épocas de provas, mas é bom porque sempre tem lugar para estudar e fecha às 22h. Se você não se importa com barulho, ou até acha que você faz bastante, esse é o lugar da BC para você estudar. O segundo andar é onde está a BAE, a Biblioteca da Área de Engenharia. Um pouco mais silenciosa que a BC nas mesas externas, esse andar tem salas para estudo em grupo, bastante silenciosas, mas que sempre estão ocupadas em época de provas, e mesas individuais escondidas entre os periódicos. O terceiro andar é para silence freaks. Morbidamente silencioso, desértico (muita gente desconhece a existência desse andar), esse é o lugar mais silencioso da BC para estudar. Tem umas salinhas de estudo individual e duas mesas para estudo em grupo. O problema é que fecha às 17h, mas o pôr-do-sol de lá de cima também é ma-ra-vi-lho-so.
\end{itemize}

\begin{itemize}
\item  \textbf{Arcádia (ou mesinhas do IEL):} A Arcádia é algumas mesas ao ar livre no IEL (Instituto de Estudos da Linguagem). Em horários de aula é silencioso, é um ambiente muito agradável e por ser ao ar livre, não fecha. Tem dois problemas: O grande fluxo de mulheres no local pode facilmente distraí-lo, principalmente se você as conhecer, e à noite enche de insetos (além da iluminação não ser das melhores). Às vezes, venta bastante e é ruim para estudar com folhas avulsas. Mas ainda assim é um ótimo local para estudar.
\end{itemize}

\begin{itemize}
\item  \textbf{Biblioteca do IFGW:} A biblioteca do IFGW (Instituto de Física Gleb Wataghin) é ótima para dias de calor, por ser super gelada (ar-condicionado mega-super-power!). Tem vantagem sobre as outras bibliotecas pelo fato das salas de estudo serem fora da biblioteca e por isso você não precisa deixar o seu material para entrar na sala de estudos. O problema é que tem poucos lugares e só duas mesas para estudo em grupo. O resto são baias individuais.
\end{itemize}

\begin{itemize}
\item  \textbf{BIMECC:} A biblioteca do IMECC tem poucos lugares, poucas mesas para estudo individual, os locais de estudo ficam dentro da biblioteca (você precisa guardar sua bolsa para entrar), não é muito gelada e o ambiente não é agradável, mas nela e na BAE é que você encontrará a maioria dos livros relacionados a computação.
\end{itemize}

\begin{itemize}
\item  \textbf{Outras bibliotecas:} Aventure-se por outras bibliotecas, como a da Economia, a da Pedago e a da Biologia e as conheça. Para aqueles que gostam (ou são obrigados) a estudar aos fins de semana a BC e as bibliotecas da Educação, da Economia, da Química, da Medicina, do IEL e da Geociências abrem aos sábados. Para saber os horários de funcionamento das bibliotecas, entre no site do SBU (\url{http://www.sbu.unicamp.br/index.php?link=30}).
\end{itemize}

\begin{itemize}
\item  \textbf{Bitolódromos:} Existem dois bitolódromos na UNICAMP: O do IC e o da FEEC (coincidência interessante, né?). O do IC-3 é uma mesa grande com algumas cadeiras no antigo saguão de entrada. O da FEEC fica no fundo do prédio principal (qualquer veterano sabe onde é o bitolódromo, não tenha vergonha de perguntar). O da FEEC é maior, você se distrai menos porque não estão todos os seus colegas (mas várias outras pessoas estão) entrando e saindo de lá (embora os pica-fios sejam bastante barulhentos), e sempre você encontra gente que possa te ajudar. O do IC serve para quando você já estiver lá e com preguiça de ir à Elétrica, porque o da FEEC é muito melhor.
\end{itemize}

\begin{itemize}
\item  \textbf{Sala 316:} Outro alento para as madrugadas de estudo é a sala 316 do IC-3, que fica aberta sempre, ou então, aberta facilmente com a chave em posse do guardinha. É uma sala com carteiras legais, lousa e ar condicionado, aliás, é muito boa para estudo em grupo (NABVS IMINENTVS) por causa da lousa.
\end{itemize}

\begin{itemize}
\item  \textbf{Sua casa:} Se você mora em uma república com pessoas da sua turma, vá fundo e estude em casa. Se você mora sozinho ou com caras de outros cursos/anos, mas se concentra bem em casa, também o faça. Caso contrário, estude na UNICAMP. É muito fácil se distrair em casa. Você vai à geladeira, mexe no computador, lê outra coisa, deita na cama e dorme, entre outras coisas. Prefira estudar na UNICAMP. Outra coisa, não seja egoísta, quando tiver oportunidade de estudar em grupo, prefira essa alternativa. Lembre-se que você não está mais no "cursinho", tente sempre pegar as dicas que a galera te dá, principalmente dos seus veteranos.
\end{itemize}

\section{Melhores banheiros}
Uma das maiores necessidades do ser humano pode ser potencializada se for
realizada num banheiro decente. Portanto, é muito importante que você saiba onde
ir. Alguns dos melhores banheiros da UNICAMP são:

\begin{itemize}
\item  \textbf{IC-3:} Geralmente estão limpos e utilizáveis. Mas fedem! Sempre com papel higiênico, é uma boa pedida na hora do apuro. Exceto nos finais de semana.
\end{itemize}

\begin{itemize}
\item  \textbf{IC-2:} Quase sempre estão limpos e utilizáveis e tem um odor melhor que os do IC-3. Só precisa tomar cuidado pois, as vezes, falta papel higiênico.
\end{itemize}

\begin{itemize}
\item  \textbf{FEEC:} Possui excelentes banheiros escondidos por lá. Procure bem!
\end{itemize}

\begin{itemize}
\item  \textbf{PB:} Os banheiros do segundo e do terceiro andar do Pavilhão Básico também são bons (especialmente os do terceiro andar, por quase não serem usados). Só tome cuidado, porque às vezes não tem papel higiênico.
\end{itemize}

\begin{itemize}
\item  \textbf{FE:} A faculdade de educação tem poucos banheiros masculinos, mas estão entre os melhores da UNICAMP pelo pouco uso.
\end{itemize}

\begin{itemize}
\item  \textbf{CB:} Estes banheiros ficam escondidos próximo às escadas do CB (no térreo). Se você tiver sorte de chegar bem após a limpeza, o banheiro estará em excelentes condições. Porém, na maior parte do tempo ele fica bem sujinho.
\end{itemize}

\begin{itemize}
\item  \textbf{DEQ:} Departamento de Eletrônica Quântica, no IFGW. Dizem que ninguém os usa.
\end{itemize}

\begin{itemize}
\item  \textbf{DRCC:} Departamento de Raios Cósmicos e Cronologia, no IFGW. Um dos melhores banheiros existentes na UNICAMP (senão o melhor). Assim como os banheiros do DEQ, dizem que ninguém os usa.
\end{itemize}

\begin{itemize}
\item  \textbf{DFA:} Departamento de Física Aplicada, no IFGW. Os dois andares do departamento tem banheiros bons e utilizáveis, mas algumas vezes falta papel higiênico.
\end{itemize}

\begin{itemize}
\item  \textbf{IMECC:} Todos os três departamentos (andares) do IMECC tem banheiros bons e utilizáveis. Mas vez ou outra falta papel higiênico.
\end{itemize}

\section{Serviços da Unicamp}
\subsection{Atendimento médico e odontológico -- CECOM}
A CSS é responsável pelo planejamento e execução de programas de saúde voltados
à comunidade universitária da UNICAMP -- alunos, funcionários e docentes.
É responsável também pelo atendimento à saúde oral desta comunidade, incluindo
também os filhos menores de servidores que estejam devidamente matriculados nas
creches e escolas dos campi.

Em português, isso quer dizer que é um "plano de saúde" da UNICAMP. Demora um
pouco (embora o pronto-socorro do CECOM seja bem mais rápido que o do HC), tem
burocracia, mas funciona. Você pode marcar consultas médicas e fazer exames.
O CECOM é localizado próximo ao HC. Para ir, é melhor pegar o Circular pois
é beeeeeem longe. De circular interno, peça para descer no CECOM. É o ponto
final ou o penúltimo dos circulares.

Caso você tenha Unimed, o Centro Médico, que fica perto da UNICAMP, atende pela
Unimed. É mais rápido que o atendimento da UNICAMP (CECOM ou SUS).

Para marcar consultas com Dentista, vá ao CECOM e peça onde que é. Isso é mais
fácil que você tentar entender lendo aqui. Basicamente é embaixo do CECOM, muito
fácil de chegar se alguém apontar com e dedo e dizer "ali". Funciona muito bem,
o atendimento é ótimo. A única burocracia é assistir uma palestra sobre doenças
da boca e escovação antes de poder marcar atendimento. Mas se você estiver com
dores eles te atendem na hora sem marcar consulta nem assistir palestra.

Para saber mais sobre o CECOM, vá ao site deles
(\url{http://www.unicamp.br/css/}).

\subsection{Hemocentro}
Essa é para quem é (ou para quem quer ser) doador de sangue. O centro de
hematologia e hemoterapia (hemocentro) é o órgão da UNICAMP responsável pela
coleta e doação de sangue. Qualquer pessoa pode aparecer no hemocentro para
fazer a doação de sangue. Basta estar com o RG e seguir um conjunto de normas
para a doação de sangue. Para saber mais sobre o processo, é só visitar a página
do hemocentro (\url{http://www.hemocentro.unicamp.br}).

O hemocentro também faz o projeto doador universitário, que consiste de unidades
móveis (ônibus) que param em alguns pontos do campus. Essas unidades fazem
a coleta do sangue. Alunos, professores e funcionários podem ir até essas
unidades móveis para fazer a doação. Para se informar melhor, é só visitar
a página do projeto
(\url{http://www.cecom.unicamp.br/doador_universitario/doador_universitario.html}).

O hemocentro fica localizado acima do HC (próxima ao CECOM). Portanto, se quiser
se deslocar até lá, como está escrito acima, é melhor usar o circular interno.

\subsection{Circular Interno/Moradia}
O circular interno é um ônibus que fica dando voltas na UNICAMP. Há duas linhas,
uma girando no sentido horário e outra no anti-horário. Só funciona até às 18h,
e a freqüência maior é no horário de almoço. Bom para cobrir boas distâncias
como bandejão->IC, IC->FEEC, qualquer lugar->CECOM, qualquer lugar->FEAGRI,
FEAGRI->qualquer lugar. Quase todos os pontos tem os itinerários e os horários.
Ele não costuma atrasar nem adiantar mais que 5 minutos, exceto no período de
férias.

O circular noturno é um ônibus que dá uma volta na UNICAMP partindo do balão da
avenida 1, passando pela BC, IQ, FEEC, IC e voltando para o balão da avenida 1.
Os horários e o itinerário desse ônibus não estão nos pontos como os do Circular
1 e 2. Ele funciona das 19:00 às 23:00, a cada 1 hora.

O circular moradia é um ônibus que dá uma volta na UNICAMP partindo da BC e vai
para a Moradia e faz o caminho contrário. Antigamente ele pegava/deixava gente
em qualquer lugar do itinerário. Atualmente o ônibus só pega/deixa gente na
UNICAMP e na moradia. Os horários mais lotados do circular moradia são 8 da
manhã (sentido moradia->UNICAMP), horário de almoço (ambos os sentidos), horário
de jantar (ambos os sentidos) e o último horário (UNICAMP->Moradia). Se puder
evitar esses horários, faça-o.

Provavelmente você irá receber todos os horários dos circulares, mas se caso
pintar alguma duvida os links estão abaixo:

\begin{itemize}
\item  \textbf{Circular da moradia:}
\begin{itemize}
\item  Horários -- \url{http://www.prefeitura.unicamp.br/prefe/site-novo/Horarios%20circular%20moradia.pdf}
\end{itemize}
\end{itemize}

\begin{itemize}
\item  \textbf{Circular Interno:}
\begin{itemize}
\item  Horários -- \url{http://www.prefeitura.unicamp.br/prefe/site-novo/circular.pdf}
\item  Trajetos -- \url{http://www.prefeitura.unicamp.br/prefe/site-novo/circulartrajeto.pdf}
\end{itemize}
\end{itemize}

\subsection{SAE -- Serviço de Apoio ao Estudante}
O SAE (Serviço de Apoio ao Estudante), principal órgão de apoio ao estudante na
UNICAMP, atua em várias frentes de assistência estudantil. Esta se dá por meio
do gerenciamento de bolsas-auxílio, assistência social e orientações
educacional, jurídica e psicológica, além de apoio a projetos acadêmicos
e sociais e programa de intercâmbio de estudantes no exterior. O SAE também
é responsável pela gestão de estágios na Universidade.

\begin{itemize}
\item  Localização: Prédio do Ciclo Básico, 3º piso (responsável pelo gerenciamento de convênios, estágios, bolsa-pesquisa e bolsa-empresa); e em frente a DAC (serviço social, responsável pelo gerenciamento de bolsas-auxílio).
\begin{itemize}
\item  Horário: Segunda a sexta, das 08h30 às 20h, no período letivo.
\item  Contato: sae [@] unicamp [.] br
\item  Site: \url{http://www.sae.unicamp.br/}
\end{itemize}
\end{itemize}

\subsection{Bolsas-Auxílio}
\subsubsection{Moradia}
Criado em 1989, o Conjunto Residencial Universitário da UNICAMP, tem por
finalidade garantir estadia gratuita e de qualidade para os estudantes que
passam por dificuldades sócio-econômicas.

Para saber mais sobre o processo seletivo entre no site da Moradia da UNICAMP
(\url{http://www.prg.unicamp.br/moradia/index.html}).

\subsubsection{Bolsa Alimentação e Transporte}
Objetivo: Colaborar com o estudante de graduação e pós-graduação em dificuldade
sócio-econômica, nos itens alimentação e transporte.

Critérios para a seleção: Análise do questionário sócio-econômico devidamente
preenchido e documentado, e entrevista com a assistente social.

\subsubsection{Bolsa Trabalho}
Objetivo: Colaborar com o estudante de graduação em dificuldade sócio-econômica,
cuja família não tenha condições de mantê-lo na Universidade. Em contrapartida
o bolsista trabalhará 15 horas semanais em unidades da UNICAMP, ou em grupos de
pesquisas, ou em grupos sociais, em horário compatível com o horário escolar.
E com a orientação de um professor.

Critérios para a seleção: Análise do questionário sócio-econômico devidamente
preenchido e documentado, e entrevista com a assistente social.

Alunos da pós não podem se candidatar à bolsa trabalho.

\subsubsection{Bolsa Emergência}
Objetivo: Atender os estudantes de graduação regularmente matriculados que
estejam passando por dificuldades econômicas emergenciais. Em contrapartida
o bolsista trabalhará 40 horas em unidades da UNICAMP, compatível com o horário
escolar.

Procedimento: o candidato deverá enviar uma carta à Coordenação do SAE
solicitando o benefício e informando sobre os motivos. Preencher e entregar
devidamente documentado o questionário sócio-econômico e marcar entrevista com
a assistente social.

Critérios para seleção: Análise da solicitação, do questionário sócio-econômico
devidamente preenchido, documentado e entrevista com a assistente social.

Alunos da pós não podem se candidatar à bolsa emergência.

\subsubsection{Bolsa PAPI}
Busca incentivar a participação de alunos de graduação e de pós-graduação nas
mais diversas atividades da Unicamp, tais como no auxílio a eventos. Neste
programa, há a solicitação de estudantes por parte de alguma Unidade ou órgão da
UNICAMP, que poderá indicar o nome do aluno ou deixar a critério do SAE para
fazê-lo.

\subsection{Assistência Jurídica}
Objetivo: Orientar os alunos nacionais ou estrangeiros de graduação ou
pós-graduação, na resolução de suas questões pessoais de cunho jurídico que
envolvam os ramos do Direito, principalmente os seguintes:

\begin{itemize}
\item  \textbf{Direito Civil:} contratos em geral; contratos de locação de imóvel/escritura de compromisso de venda e compra; acidentes de trânsito; reparação de dano; ação revisional de aluguel; separação judicial; divórcio; pensão alimentícia, etc.
\item  \textbf{Direito Penal:} violência contra a pessoa; lesões corporais; furto; roubo etc{\dots}
\item  \textbf{Direito do Trabalho:} caracterização de relação de emprego para os não registrados e direitos trabalhistas em geral (Normas da CLT).
\end{itemize}

Procedimento: O aluno deve dirigir-se pessoalmente ao SAE para obter os
esclarecimentos desejados.

Delitos de consumo, encaminhar ao SEDECON ou ao PROCON.

Com relação aos contratos de locação alertamos para algumas dicas importantes:

\begin{itemize}
\item  Jamais pagar qualquer valor antecipadamente (ex: "taxas de reserva de imóvel", "taxa de contrato", aluguel antecipado etc), ou fornecer títulos em garantia: (cheque pré-datado, nota promissória).
\item  Obter o maior número de informações possíveis relativas: ao imóvel, proprietário(s), imobiliária, procuradores;
\item  Condições e valores para pagamento (aluguel e outros encargos ex.: Condomínio, IPTU, seguros etc, para uma real visualização do valor total a ser despendido).
\item  Sempre solicitar uma Minuta do contrato locação.
\item  Não assinar contrato ou qualquer outro documento antes de apresentá-lo para análise por um dos advogados da O.J. SAE.
\end{itemize}

\subsection{Orientação psicológica}
Objetivo: Prestar atendimento psicológico ao aluno de graduação, pós-graduação
e especial.

O Serviço de Orientação Psicológica do SAE funciona em associação com o Serviço
de Atendimento Psicológico e Psiquiátrico ao Estudante (SAPPE) do Departamento
de Psicologia Médica e Psiquiatria da FCM, desde a sua implantação, em 1987.

Funcionamento e informações:

\begin{itemize}
\item  Existem horários para consultas imediatas (ideais para quem está prestes a expulsar um amigo da república etc);
\item  O tratamento de longo prazo é obtido mediante uma palestra explicativa, um horário de atendimento individual e espera em uma lista de candidatos -- horários disponíveis e motivos especiais podem facilitar seu ingresso, mas lembre-se: não é porque aparentemente não parece importante o motivo pelo qual você procura ajuda que isso não deva ser levado a sério. Em alguns casos, apenas quatro seções são suficientes para a pessoa sair do tratamento (lembrando que ele pode ser interrompido a qualquer momento);
\item  Há possibilidade de tratamento em grupo terapêutico, psicoterapia individual, de família e de casal com uma das psicólogas da equipe. O grupo terapêutico é formado levando-se em conta que não devem haver pessoas muito próximas em sua formação, como condição de que o paciente tenha liberdade para falar de seus problemas com menos receio.
\end{itemize}

\subsection{Cadastro de veículos}
Para você poder entrar e sair da UNICAMP sem ter de parar receber e entregar
o papel de identificação do veículo você pode cadastrar seu carro junto
à Prefeitura do Campus. O cadastramento do veículo deverá ser feito diretamente
na Central de Informações (próximo ao Balão da avenida 1) de segunda
a sexta-feira das 09h às 17h.

Documentos necessários:

\begin{itemize}
\item  Identidade funcional (para funcionários/docentes), identidade estudantil para alunos, carta de apresentação da unidade para estagiários;
\item  Documento do veículo;
\item  Preenchimento de declaração.
\end{itemize}
Há possibilidade de se cadastrar até três veículos, com um custo de R\$ 5,00
para o segundo e de R\$ 10,00 para o terceiro veículo.

\section{Agora eu tenho um email da Unicamp}
Ao ingressar num curso de Computação, você recebe 2 contas de e-mail: Do IC e da
DAC. Estes são os principais meios de comunicação da universidade com você,
portanto fique esperto e não deixe de ler regularmente!

Depois de muita luta, o IC foi um dos últimos institutos a ter um webmail
(acredite se quiser!), para que você possa acessar sua conta de qualquer lugar.
O endereço é \url{http://webmail.students.ic.unicamp.br/} e lá você precisa do
seu login (que é o mesmo login do IC, raXXXXXX) e uma senha, que será a mesma
senha do sistema Linux do IC (que vocês receberão do IC em uma das primeiras
semanas de aula). Caso tenha dúvidas, de uma passada na Secretaria de Graduação,
que fica no IC-2 (prédio ao lado das Artes Cênicas e para cima da Economia)
e pergunte!

A DAC também dispõe um webmail para o aluno. O login é a primeira letra do nome
do aluno, seguido dos dígitos do RA. O email da DAC também é útil. É nele que os
alunos são avisados sobre o período de matrícula, recebem os seus pedidos de
matrículas provisórias, avisos de desistências e trancamentos, além de informar
os alunos a respeito de eventos que acontecem na UNICAMP, como feiras,
palestras, festivais, eleições. O melhor a se fazer com essa conta
é direcioná-la para uma outra conta de email da UNICAMP (tem de ser da UNICAMP),
pois você nunca vai lembrar de acessá-lo (instruções para redirecionamento de
emails são dadas logo abaixo). Na própria página da DAC
(\url{http://www.dac.unicamp.br/}) você diretamente cria a sua senha e já pode
acessar o webmail dali mesmo.

Para a galera da Engenharia, tem também o e-mail da FEEC, que muitos nem ficam
sabendo que existe! Só um semestre ou até um ano depois vão até o SIFEEC retirar
seu login e senha! Ou então ficam sabendo só no segundo ou terceiro ano de curso
e aí acumula mais de 700 e-mails não lidos. Assim como o do IC, é muito
utilizado para divulgação de eventos, oportunidades de estágios e de iniciação
científica. Para utilizá-lo, você deve ir até a FEEC e procurar o SIFEEC, que
é o local responsável por isso. Fica no segundo andar do prédio de laboratórios
da FEEC (um com escadas amarelas). O endereço do webmail da FEEC
é \url{http://www.fee.unicamp.br/webmail}.

Você também pode redirecionar os e-mails que receber nas contas do IC, FEEC
e DAC para qualquer outra conta (no Gmail por exemplo). Para isso, siga as dicas
abaixo:

\subsection{Redirecionamento de email da DAC}
Para efetuar o redirecionamento do e-mail institucional para outro e-mail, siga
os passos abaixo:

\begin{itemize}
\item  Acesse Serviços On-line;
\item  Alunos;
\item  Acesso aos Serviços Acadêmicos;
\item  Informe seu RA e sua senha atual (Observação: Não há necessidade de preencher o 3º campo, exceto se sua senha expirou);
\item  Clique no botão confirmar;
\item  Redirecionamento de e-mail.
\end{itemize}

\subsection{Redirecionamento de email da FEEC}
\begin{itemize}
\item  Logue no Linux
\item  Abra um terminal
\item  Certifique-se que está na home
\item  Digite o comando: echo 'seu\_email@gmail.com' > .forward (logicamente substituindo 'seu\_email' pelo seu e-mail).
\end{itemize}

\subsection{Redirecionamento de email do IC}
\begin{itemize}
\item  Acesse \url{https://webmail.students.ic.unicamp.br/}
\item  Faça o login
\item  Filtros
\item  Adicionar nova regra
\item  Na primeira caixa, selecione ALL
\item  No menu abaixo, clique em "Redirecionar para o seguinte endereço de e-mail"
\item  Defina o endereço pra onde você quer redirecionar
\item  Clique em "Adicionar nova regra"
\end{itemize}

Qualquer duvida, novamente, procure um veterano! Esta é sua melhor opção!

Finalizando, não deixe de estar sempre informado sobre os acontecimentos ou
divulgações da UNICAMP, da AAACEC, do CACo e da CONPEC, essas três entidades
compostas por alunos de Computação.

\section{Iniciação Científica}
A Iniciação Científica é um tempo para o aluno de graduação (você, no caso) ter
uma experiência acadêmica mais séria, sentir um pouco como é o clima de
pesquisa. Interessou? O que fazer? Calma, você mal entrou na Universidade.
Geralmente o que se faz é conversar com o professor da área com a qual você se
identifica mais (criptografia, teoria da computação, processamento de imagens,
inteligência artificial, etc), e ver se ele está desenvolvendo algum projeto
interessante naquela área, ou propor para ele alguma idéia sua original mesmo.
Depois você começa a estudar para redigir um projeto e encaminhar para alguma
instituição de fomento à pesquisa (CNPq ou FAPESP), pedindo uma bolsa de
Iniciação Científica. A FAPESP paga R\$424,80 e aceita pedidos de bolsa em
qualquer período do ano. O CNPq paga aproximadamente R\$300,00, e o período para
inscrição de projetos é geralmente em junho e novembro. No primeiro semestre
geralmente é bem mais difícil achar algum professor da área que você se
interessa, aliás, é bem difícil saber a área com a qual você se identifica, pois
você mal começou o curso e não conhece muito do que se estuda em Computação,
muito menos os professores. Mas tenha paciência, agora parece tudo muito
complicado e complexo, mas com o tempo, as coisas vão ficando mais simples. Se
você realmente tiver uma sede insaciável de conhecer o meio da pesquisa, procure
o seu professor de MC102, ele pode te orientar a respeito.

Outra coisa interessante a respeito da Iniciação Científica é que, se você
conseguir bolsa (o que não é muito difícil), pode pegar a disciplina MC040
e posteriormente MC041 (2 semestres), cada uma com 6 créditos, e geralmente se
tira notas altas nessas disciplinas. Ou seja, são 12 créditos praticamente "de
bandeja" para aumentar o seu CR, ou ajudar a recuperá-lo, caso esteja no fundo
do poço. Note bem as aspas. Os trabalhos de Iniciação Científica geralmente
consomem muito tempo de estudo e dedicação, não vá pensando que é moleza não.

Na FEEC você pode conseguir as matérias de Iniciação (EA002 até EA005) mesmo sem
bolsa, mas você ainda assim vai precisar de um orientador. Lá a Iniciação
Científica também substitui o estágio, mas não tem equivalência com as do IC.

\section{Intercâmbio}
Como assim? Acabei de sair da minha cidade natal, do aconchego do meu lar e você
vem me falar de sair do país? Calma, calminha{\dots} Não é para agora, mas é bom
ir se preparando {\dots} Falando sério, uma introdução sobre intercâmbios.
A Unicamp hoje, junto com a USP, é uma das universidades brasileiras que tem
maior prestígio fora do país e o Cori (Cooperação Internacional), IC e FEEC tem
vários acordos bilaterais de intercâmbio. Então para você que quer dar um salto
em algum idioma, conhecer outras culturas e sentir na pele a aventura de ser
estrangeiro, comece a se preparar desde já.

A primeira coisa é escolher uma língua. Pois, mesmo o inglês sendo a língua que
se fala no mundo, não é 100\% da população do país de destino que fala o inglês
corretamente, então pense em começar a estudar a língua do país em que pretende
ir no seu intercâmbio. Hoje a maior parte dos computeiros vai para França,
Alemanha e países hispanofônicos, logo pensar em fazer francês, alemão ou
espanhol é uma boa pedida e com chances de facilitar sua aceitação do pedido de
intercâmbio. Existem também bolsas para estudos na Itália, então aprender
italiano pode ser uma boa alternativa. Ter um bom nível de inglês também ajuda
no processo seletivo e, claro, inglês não faz mal a ninguém (ainda mais, a nós
do ramo de tecnologia)!

Para onde ir? Grande questão! Vou citar o caso dos EC03: de 90 alunos, 10 alunos
foram estudar/estagiar fora! Destes, 6 foram para França, 1 para Portugal,
1 para Itália, 1 para Eslovênia e 1 para Argentina.

A França hoje recebe grande numero de estudantes de computação, dado acordos que
a FEEC tem com os INSA's e com as écoles Centrales e também graças a bolsas de
estudo oferecidas pela Capes e pelo governo francês. Mas não significa que
a França seja seu destino certo, no site do CORI existem oportunidades para ir
para, além dos países já citados, Japão, América Latina, Alemanha, Espanha, etc.
Muitos com boas bolsas de estudo ou com incentivos que valem a pena caso você
tenha um pouco de grana para se sustentar no inicio. Visite sempre o site da
CORI e participe das reuniões que ela faz, ficar ligado é a chave para conseguir
encontrar uma boa oportunidade. Existem outras opções de intercâmbio como
o AIESEC que promove um intercâmbio para estágios no exterior, se seu interesse
é mais profissional, procure se informar.

Mas vale a pena? Poxa, vou atrasar meu curso, ficarei deslocado de turma, vou
ficar em um país estranho, para quê? Acho que não vale a pena{\dots}

{\dots}Ledo engano, ledo engano{\dots}

Vamos começar pelos motivos profissionais : ter no currículo que você fala uma
lingua estrangeira fluentemente devido a sua imersão no país é algo muito
valorizado pelas empresas, além do fato que o pessoal do RH vai ver que você tem
capacidade de se virar sozinho, uma vez que não é tão óbvio sair do país
e recomeçar sua vida fora. Você não atrasará tanto seu curso, pois a Unicamp
conta com um sistema de equivalências de matérias e se você escolher bem pode
fazer matérias que serão convalidadas na Unicamp. Agora, o que realmente
é importante! Você está na faculdade, está na hora de deixar o colo da mamãe
e partir para o mundo{\dots} a experiência de conhecer outras culturas, criar
laços de amizade (ou mesmo algo mais) internacionais, viajar por terras
desconhecidas não tem preço!!! Pense que é no seu tempo de facul que terá
oportunidade de fazer uma aventura destas, não desperdice{\dots}

Se você abriu um sorriso e pensa que está preparado para sair do país{\dots}
comece a estudar, garoto! Não que, ter uma boa nota seja a única forma de
conseguir uma vaga em uma bolsa de estudos, mas com certeza é a mais fácil.
Busque "atirar em todas as frentes", mantenha seu CR num bom nível, busque
conhecer organismos como AIESEC e procure grupos de trabalho (no IC existem
vários) que podem levar alunos ao exterior. Boa sorte!

Se você realmente se interessou, ai vão uns links com mais informações:

\begin{itemize}
\item  AIESEC: \url{http://www.aiesec.org/brazil/}
\item  CORI: \url{http://www.cori.unicamp.br/}
\end{itemize}

Fique atento aos e-mails que você receberá do IC e da FEEC. Muitos deles são sobre programas de intercâmbio.

\section{Eletivas e Proficiência}
A UNICAMP oferece a seus alunos a oportunidade de personalizar seu currículo de
acordo com seu interesse por meio das disciplinas eletivas. Ao contrário das
disciplinas do núcleo comum, você pode escolher a matéria que vai cursar. Alguns
créditos podem ser cumpridos com qualquer disciplina oferecida pela
universidade, outros estão restritos a um determinado conjunto (para mais
detalhes, consulte seu catálogo no
\href{http://www.dac.unicamp.br/portal/grad/catalogos/index.html}{site da DAC},
na página destinada à Graduação).

Mas não se esqueça de que com um grande poder vem uma grande responsabilidade!
A UNICAMP lhe dá liberdade para escolher o melhor jeito de se preparar para seu
futuro, e espera que você saiba o que fazer com ela. Você pode socializar com
outros cursos, aprender uma língua estrangeira, assistir a seminários ou obter
um certificado de estudos na FEEC ou no IC.

Muitas pessoas, especialmente nos cursos de computação, fazem proficiência em
línguas, em alguns casos para evitar o jubilamento, outros para não ter que
passar mais um semestre na faculdade. Apesar de registrar a familiaridade do
aluno com uma outra língua em sua integralização, essa prática não enriquece
a graduação de nenhum estudante que faça tal escolha. Além disso, muitos bixos
arrependem-se de terem feito a prova e perdido preferência na hora de pegar uma
disciplina interessante, podendo até mesmo não conseguir se matricular.

Converse com seus amigos e veteranos para descobrir o melhor jeito de usufruir
dessa liberdade que poucas universidades oferecem! Dificilmente você não
encontrará algo com o qual se identifica ou que não ensine lições interessantes.

\section{Cuidado com CR e Reprovação}
Durante o curso você vai ouvir que se preocupar com seu CR é bobagem, que
estudar para tirar nota não leva a lugar nenhum, que depois de formado não é seu
CR que te colocará no mercado de trabalho, etc. Cuidado, a sua única opção de
vida após formado não é trabalhar numa empresa, e preste atenção "após formado"
não significa durante o curso.

Durante o curso você vai ter a possibilidade de participar de várias atividades
acadêmicas e algumas delas vão exigir bom aproveitamento acadêmico do aluno, por
exemplo, para se candidatar a monitor de uma disciplina é exigido do aluno um CR
>= 0,7. Isto significa que sua média de notas das disciplinas cursadas até
aquele momento deve ser maior ou igual a 7,0. Para pleitear uma bolsa de
iniciação científica, onde há concorrência entre alunos do país todo, também
será exigido bom aproveitamento, assim como para uma bolsa de mestrado. Caso
você não saiba, mestrado faz parte da pós-graduação, ou seja, o seu CR vai te
influenciar até após formado.

Cuidado também com a reprovação, há instituições, como a FAPESP (Fundação de
Amparo à Pesquisa do Estado de São Paulo) que é a maior fomentadora de pesquisas
do estado de São Paulo e que paga os maiores valores de bolsas do país, que te
exclui de qualquer disputa só por ter uma reprovação no seu histórico escolar da
graduação. Não te exclui oficialmente, mas como é muito concorrido por ser
a melhor pagadora, se seu nome for para o final da lista por causa da reprovação
pode considerar-se excluído da disputa.

Parece óbvio, quem estuda tira boas notas, mas até você aprender a estudar como
a universidade exige pode demorar um pouco e há pessoas que nunca aprendem.

Há certos períodos (semestres) quando você já estiver mais avançado no curso em
que poderá sentir-se à vontade para desistir de uma disciplina em que esteja
matriculado, deixando para completá-la posteriormente. Quando você fizer isso há
a possibilidade de desistir da disciplina, se desmatriculando oficialmente dela.
Mas há pessoas que simplesmente deixam de cursar a disciplina, reprovando por
nota e falta e ficando com uma nota baixa em seu histórico. Cuidado com isso,
pode ser frustrante para você no futuro. Por isso se for desistir de cursar uma
disciplina após matriculado, sempre peça a desistência e tente não reprovar.

Lembre-se de que 4 ou 5 anos é muito tempo, você pode mudar de ideia a qualquer
momento sobre o que pretende fazer no futuro.

\section{Grupos e Entidades da Unicamp}
\subsection{GPSL}
O Grupo Pró Software Livre (GPSL) é aberto para alunos de todos os institutos
e para a comunidade. Seu objetivo é de promover o uso e o desenvolvimento do
software livre.

A grande maioria das instituições educacionais utilizam apenas software
proprietário e a UNICAMP não é exceção. São poucos os institutos que usam
software livre e é muito fácil se formar sem sequer saber da existência ou ter
qualquer experiência com algo diferente do software proprietário vigente.

O GPSL é um espaço para refletir sobre este monopólio do software proprietário
imposto sobre nós através de tantas instâncias, inclusive nossa própria
universidade, a ética (ou falta de ética) por trás disso e seus efeitos sociais.
É também um espaço para se articular e pensar em formas de ação, que podem
envolver desde a educação e conscientização dos usuários de computadores até
o próprio desenvolvimento de software livre. O GPSL também pensa sobre
a questões de mercado e econômicas relacionadas ao software livre.

Como atividades práticas organizamos os Seminários de Software Livre, pequenos
encontros onde se discute desde de questões técnicas até as questões filosóficas
relacionadas ao software livre. De vez em quando também organizam cursos.

Para entrar no nossa lista de discussão mande um e-mail para gpsl-unicamp [@]
googlegroups [.] com, ou então se cadastre na mesma através do site:
\url{http://groups.google.com/group/gpsl-unicamp}.

\subsection{Gamux}
O Gamux (Grupo de Pesquisa e Desenvolvimento de Jogos da Unicamp), composto por
computeiros e alunos do IA (Instituto de Artes), é uma ótima oportunidade para
quem tiver curiosidade de saber como são feitos os jogos eletrônicos. E isso
tanto de PC quanto console, já que o grupo dispõe de um Xbox para
desenvolvimento.

Mas não espere já chegar fazendo um MMORPG 3D com mais raças que um livro de
Tolkien. Paciência, padawan, você provavelmente começará com algo bem simples,
como um pong, e descobrirá que não é tão simples assim{\dots}

Fique atento, eles costumam organizar um ciclo de palestras semanais sobre como
fazer jogos em XNA (um framework da Microsoft, em C\#, que roda tanto no PC
quanto no Xbox. Sim, bixos sem saber orientação a objetos conseguem acompanhar).

Mas não precisa esperar até lá, apareça no LMS (Laboratório da Microsoft, logo
depois do guarda no IC, é o "beco" na direita) e/ou entre no fórum/lista de
e-mails.

O grupo possui um grande conhecimento em XNA, porque é o mais usado nos inúmeros
concursos que participam (e ganham =), mas por também ser de pesquisa, não se
restringe a isso. Se você quiser estudar allegro, SDL, Ogre, Game Maker, até
MUGEN (lol), tenha certeza que alguém poderá te ajudar (nem que seja pra te dar
uns links e dizer RTFM).

\begin{itemize}
\item  Site: \url{http://bacuri.lms.ic.unicamp.br}
\item  Fórum: \url{http://forum.gamux.com.br/}
\item  Lista de discussão: \url{http://groups.google.com/group/gamedev_unicamp}
\end{itemize}

\subsection{Curso Exato}
O Curso Exato tem como principal objetivo auxiliar no aprendizado das
disciplinas de matemática, física e química de forma direta e dinâmica, ou seja,
tentar preencher com algum conhecimento as mentes ocas dos estudantes de ensino
médio de Campinas. Tem como professores alunos de graduação Unicamp, orientados
por docentes da instituição. O curso possui ampla presença dos alunos de
Engenharia de Computação, que são membros ativos do projeto. Possui caráter
comunitário, sendo totalmente gratuito. Conta com o apoio da pró-reitoria de
extensão e assuntos comunitários (PREAC) e do curso pré-vestibular Cooperativa
do Saber.\textbackslash \textbackslash  Você, bixo, que pensa em fazer algo de
útil da sua vida arregace as mangas e procure a gente que será muito bem-vindo!

\begin{itemize}
\item  Site: \url{http://www.cursoexato.com.br}.
\item  Email pra contato: curso [.] exato [@] gmail [.] com
\end{itemize}

\subsection{DCE}
Criado em 1978, o DCE (Diretório Central dos Estudantes) é a entidade que
representa todos os estudantes de graduação da universidade, articulando
e organizando o Movimento Estudantil (ME).

Enquanto entidade representativa, cabe ao DCE representar o conjunto dos
estudantes em todos os espaços dentro e fora da universidade, diante das mais
diversas entidades (reitoria, sindicatos, DCEs de outras universidades, Centros
Acadêmicos, associações etc.) e Movimentos Sociais.

Enquanto articulador do ME, cabe ao DCE organizar os estudantes na luta por uma
educação superior realmente pública, gratuita e de qualidade. Para tanto,
é papel fundamental do DCE propor, juntamente com os Centros Acadêmicos,
discussões políticas que extrapolem os nossos currículos e o nosso dia-a-dia.
Além disso, o DCE deve propor ações que vão ao encontro das reivindicações
estudantis, de forma que elas sejam levadas e cobradas da reitoria ou até mesmo
do governo.

O DCE esteve envolvido em várias conquistas dos estudantes, das quais podemos
destacar algumas lutas históricas: A construção da moradia estudantil; a melhora
de estrutura para cursos noturnos (o que tornou acessível para esse período,
bibliotecas, xerox, laboratórios e secretaria de graduação); a reunião semestral
para avaliação de curso; o não-aumento do preço do bandejão; uma seleção mais
justa para as vagas na moradia; entre diversas outras. Além disso, o DCE teve
participação em importantes lutas sociais que extrapolam o âmbito da UNICAMP,
como a organização do Plebiscito contra a Alca e do Plebiscito contra o Provão;
a luta por mais verbas para a educação no estado de São Paulo; diversas lutas
pela qualidade do ensino e manutenção de direitos dos estudantes em outras
universidades como na UNIP, UNIMEP, FUPPESP etc.

Porém, muitas lutas importantes ainda devem ser travadas por nós estudantes:
Seja por mais bolsas de assistência estudantil, por mais vagas na moradia, por
mais democracia nos espaços de decisão da universidade, por melhorias nos
laboratórios, pela contratação de mais professores etc.

Todo ano (mais precisamente no mês de novembro), há eleições para definir qual
a chapa que comandará a entidade no ano seguinte, assim como há eleições para
representação discente no Consu e na CCG. É muito importante a participação dos
alunos nessas eleições.

Ainda esse ano, teremos pela frente uma importante discussão a ser feita, que
atinge diretamente os estudantes universitários, que é o projeto de Reforma
Universitária apresentado pelo governo Lula. Resumidamente, podemos dizer que
esse projeto traz um conjunto de medidas que reforçam a privatização das
universidades públicas e passa a encarar a educação superior não mais como um
direito e sim como uma mercadoria. Essa pauta esteve na ordem do dia do DCE em
2007, começando pelo tema da Calourada Integrada (organizada pelo DCE e Centros
Acadêmicos), que abordou o conjunto de reformas proposto pelo governo, em
especial a Universitária (leia mais sobre a Reforma Universitária no site do
DCE).

Enfim, são muitas as discussões e as lutas que nos aguardam esse ano. Não deixe
de participar da Calourada Integrada (a programação estará espalhada pela
universidade), do DCE (que fica ao lado do bandejão) e do CACo. Afinal, o ME
deve ser construído por todos nós. Participe!!!

\begin{itemize}
\item  Telefones: (19) 3521-7910 / 3521-7042.
\item  E-mail: dceunicamp [@] gmail [.] com.
\item  Site: \url{http://www.dceunicamp.org.br/}.
\end{itemize}

\subsection{Rádio Muda}
Você provavelmente nunca viu nada do tipo na sua vida. Uma rádio na qual
qualquer ser humano pode fazer o seu programa tranqüilamente, sem burocracias
(tendo espaço na grade de horários, lógico).

Funciona mais ou menos assim: Você vai na Reunião de Grade, que geralmente é no
começo do semestre (procure se informar!) em alguma sala do IFCH, vê se tem
algum horário livre na grade de programação, dá o seu nome e e-mail, ou
telefone, e o nome do seu programa. Se tiver mais gente querendo o mesmo
horário, não sei direito como é feita a seleção, quais os critérios. Essas
reuniões costumam lotar muito. Enfim, se tudo der certo, você paga R\$10,00 por
uma hora semanal de programa, durante o semestre todo. O preço é simbólico, só
para ajudar o pessoal da rádio (entenda-se os programadores, incluindo você)
a pagar as contas, consertar equipamentos, enfim, manter a rádio no ar. Depois
é só fazer a cópia da chave com um carinha da livraria do IFCH, ir no programa
de alguém para aprender a mexer nos equipamentos e mandar bala!

A Rádio Muda fica embaixo da caixa d'água (carinhosamente apelidada de Pau do
Zeferino) que fica perto do Teatro de Arena, bem em frente à BC (Biblioteca
Central).

Se você só quiser ouvir a muda, 105,7 MHz no seu rádio (em Barão Geraldo ou
Paulínia) ou pela Internet, através do site \url{http://muda.radiolivre.org/}.

\subsection{Competições de programação}
Curte programar? Nunca programou mas está gostando de MC102? Já brincou de
"Olimpíada" naquelas provinhas com direito a medalha? Vá fundo! Para os bixos
com menos de 20 anos existe a Olimpíada Brasileira de Informática no primeiro
semestre. Anualmente, muitos alunos do IC (em ambos os cursos) ganham premiação
nessa competição. O site dela é \url{http://olimpiada.ic.unicamp.br/}. Esteja
sempre visitando-o.

Ela é uma familiarização para a principal -- a Maratona de Programação (não, não
corremos 40 km programando), que consiste de problemas mais difíceis e é feita
em equipes de 3 alunos. A UNICAMP possui uma grande tradição nessa competição,
tendo ido a 4 dos últimos 6 mundiais, e sendo campeã da regional brasileira em
2000, 2002 e 2004. Já houve campeões da ciência, do mestrado, das duas
modalidades da engenharia, além de bixos medalhistas. Para maiores informações,
procure pelo site da organização nacional: \url{http://maratona.ime.usp.br/}.
O treinamento com o técnico Alberto Miranda (miranda [@] ic [.] unicamp [.] br)
ocorre às sextas-feiras à tarde no IC-3.

\subsection{Projetos de Extensão Comunitários}
\subsubsection{SUBA}
O SUBA (Sociedade e Universidade em Busca de Alternativas) é um movimento
político surgido em 2003, fruto de uma ocupação de mais de 40 dias ao lado do
bandejão, e que hoje congrega diversos projetos de extensão em debates e ações
acerca do modelo de universidade existente e da possibilidade de transformá-lo.
O SUBA tem uma atuação marcante no questionamento das estruturas da
universidade, se contrapondo à mercantilização da educação impulsionada pelos
cursos pagos, e propõe a construção de uma universidade e de uma sociedade que
vise a superação das desigualdades.

Para conhecer melhor o que é o SUBA informe-se com o pessoal do CACo ou do DCE,
ou vá diretamente ao espaço conquistado que fica ao lado da Praça da Paz.
Conheça também os projetos de extensão comunitátirios (veja o tópico "Grupos
e Entidades da UNICAMP"). Segue uma reportagem publicada no CMI (Centro de Mídia
Independente)
\url{http://www.midiaindependente.org/pt/blue/2003/10/266379.shtml}:

\textit{Estudantes ocupam espaço privatizado na UNICAMP}

\textit{Estudantes da UNICAMP e movimentos sociais ocuparam, na segunda-feira, dia 20 de outubro, um espaço da universidade desocupado há mais de um ano que tinha sido antes cedido para uma lanchonete Subway e fechado por problemas de higiene e dívidas. Os estudantes e movimentos sociais se reuniram num movimento chamado SUBA (Sociedade e Universidade em Busca de Alternativas) e pretendem dar um uso público e social para o lugar.}

\textit{A reitoria, por sua vez, estuda alugar o espaço para restaurantes, bancos e outras empresas. O espaço está ocupado e uma ampla grade de atividades sociais e culturais foi estabelecida. A programação é aberta e toda a sociedade é convidada a participar das atividades.}

\textit{A ocupação visa a apoiar os debates e a realização de projetos de extensão universitária, atualmente sem o apoio institucional da universidade, além de dar maior visibilidade ao processo de privatização da UNICAMP, que vem privilegiando empresas nacionais e transnacionais em detrimento de um ensino público e de qualidade.}

\textit{Apoio: Grupo Aberto de Extensão Universitária; Projetos de Extensão Universitária; Bateria Pública; BioArt; Campanha Nacional contra a ALCA -- Comitê UNICAMP; Cinematographo; Comissão de Cultura, Esporte e Lazer do DCE; Diversidade -- Grupo Pela Livre Expressão da Sexualidade Humana; GPSL -- Grupo Pró Software Livre; Grupo de Flautistas; Espaço Cultural da Mogiana/Guanabara; Núcleo pela Reforma Agrária "Carlos Marighella"; Pastoral Universitária; Ponte Pra Lua (Pirofagia); DCE UNICAMP -- Diretório Central dos Estudantes; CAB, CACH, CACT, CACo, CAEA, CAECO, CAEF, CALL, CAIA, CAP e outros centros acadêmicos da UNICAMP.}

\subsubsection{Cine Clube D'Amora}
Exibe e promove debates de obras fundamentais do cinema na Moradia, com
o objetivo de integração e intercâmbio entre estudantes e moradores dos bairros
vizinhos à Moradia. Que o Cine D'Amora se torne referência para ambos de
cultura, lazer e reflexão!

\subsubsection{Cio da Terra}
Trabalha na perspectiva de uma educação crítica e superadora das desigualdades
sociais na comunidade do Assentamento Rural de Sumaré II. É direcionado aos
moradores locais que concluíram o ensino médio e que querem e não podem pagar
pela continuidade dos estudos.

\subsubsection{Cursinho da Moradia}
Pré-Vestibular Popular na Moradia Estudantil, busca propiciar a troca ativa de
conhecimento, estimulando o senso crítico e o reconhecimento pelo estudante de
sua condição social de exclusão, tal como a possibilidade de transformar
a sociedade.

\subsubsection{Grupo de Diversidade Sexual}
Grupo de discussão sobre os LGTTBs (Lésbicas, Gays, Travestis, Transexuais
e Bissexuais). Se reúne às Terças-feiras, 17:30, na cantina do DCE.

\subsubsection{Mano a mano}
Trabalha através da arte-educação com crianças e adolescentes em situação de rua
no centro de Campinas. Objetiva que eles desenvolvam sua autoestima, se
reconheçam como sujeitos de direitos e questionem a condição em que vivem,
buscando alternativas para sair da rua.

\begin{itemize}
\item  E-mail: manoamano [@] yahoogrupos [.] com [.] br.
\item  Site: \url{http://move.to/manoamano}.
\end{itemize}

\subsubsection{MAP -- Movimento Abrindo Portas}
Trabalha alfabetização de jovens e adultos excluídos da educação formal em Barão
Geraldo. Objetiva uma alfabetização ampla, da leitura do mundo, contribuindo
para que o educando se reconheça enquanto sujeito do seu aprendizado e da
transformação da sociedade.

\subsubsection{Núcleo Pela Reforma Agrária Carlos Marighella}
O Núcleo desenvolve e participa das atividades político-culturais dentro e fora
da Universidade com o objetivo de discutir as mazelas sociais oriundas do
processo de desenvolvimento da sociedade brasileira e a importância dos
movimentos e organizações populares para a reversão desse quadro, sobretudo no
tocante à questão agrária.

\subsubsection{Projeto Educacional Machado de Assis}
Trata-se de um projeto educacional que funciona no Instituto de Estudos da
Linguagem (IEL). É destinado a estudantes de baixa renda e visa, além do
ingresso destes nas universidades públicas, proporcionar uma formação crítica,
inclusive sobre o próprio caráter excludente do vestibular e da universidade.

\subsubsection{Projeto Educacional Via Popular}
O Via Popular é um projeto de educação popular que, além de abordar conteúdos
programáticos cobrados nos exames de Colégios Técnicos, visa à construção do
senso crítico e o reconhecimento do indivíduo na sociedade e sua capacidade para
transformá-la.

\subsubsection{Sonha Barão}
Junto a disciplina AM018, envolve os alunos da UNICAMP no desenvolvimento de
trabalhos sociais em Barão Geraldo, promovendo, assim, a integração da
universidade com a comunidade e proporcionando a vivência da cidadania
e a formação do homem integral.

\subsubsection{Trilharestórias}
Baseado na contação e leitura de histórias, é um projeto de arte-educação que
atua junto a crianças da periferia de Campinas e na Moradia, com o objetivo de
incentivar a imaginação e expressão criativa, além do gosto por histórias, entre
elas as das comunidades que as crianças fazem parte.

\subsubsection{VEJA -- Vivência Educacional de Jovens e Adultos}
Curso de nível fundamental para moradores de Barão Geraldo. Busca o aprendizado
mútuo entre educadores e educandos, através da troca de saberes e de uma
educação reflexiva, dinâmica e prazerosa, que aborde temas da realidade e do
cotidiano.

\subsubsection{Viveiro Guapuruvu}
Por meio da capacitação dos estudantes no trabalho de agro-floresta e realização
de plantios comunitários, busca a articulação dos estudantes com outros setores
da sociedade. Produz mudas em viveiro no IB para o plantio em mutirões no
Assentamento Rural II de Sumaré.

\subsection{Grupos Religiosos}
\subsubsection{Pastoral Universitária}
Grupo católico que se reúne semanalmente para estudar textos (bíblicos ou não),
livros, documentos, aprofundar a fé e promover a integração e união de seus
participantes. A Pastoral Universitária também organiza grupos de preparação
para Primeira Comunhão e Crisma, além de duas Missas semanais e Grupos de Oração
Universitários (GOUs). As reuniões da PU acontecem às quartas, em dois horários:
As 12h15 e as 18h10. As Missas são realizadas às quintas (12h15) e às terças
(18h10). Os GOUs acontecem nas terças (12h30) e nas quintas (18h). O local
é sempre o mesmo para todas as atividades: A sala PB18.

\subsubsection{ABU -- Aliança Bíblica Universitária}
Grupo evangélico não ligado a nenhuma denominação que organiza várias reuniões
e grupos de discussões, filiado à Aliança Bíblica Universitária do Brasil
(\url{http://www.abub.org.br}). Quaisquer dúvidas, entre no site
\url{http://www.abucampinas.org}, mande um e-mail para contato [@] abucampinas
[.] org ou abucamp\_co [@] yahoogrupos [.] com [.] br ou ainda ligue para (19)
3289-2823.

\section{Conpec}
A CONPEC é a empresa júnior dos cursos de Ciência e Engenharia da Computação da
UNICAMP. Nela você tem a oportunidade de aplicar os conhecimentos teóricos
adquiridos em sua vida acadêmica em uma situação real, de mercado, com clientes,
prazos e soluções reais. É uma chance ainda de aprender sobre coisas que você
nunca veria na faculdade, como Marketing, Finanças, planejamento e liderança,
indispensáveis considerando-se que exige-se cada vez mais do profissional de
computação um perfil empreendedor. Muitos ex-membros da CONPEC usam os
conhecimentos adquiridos na empresa não só como um adicional ao buscar uma vaga
no mercado de trabalho, mas também para montar suas próprias empresas ou em
serviços não ligados diretamente à computação, como consultorias estratégicas.

Dito isto, a CONPEC é uma excelente oportunidade para conhecer os seus colegas
de cursos -- veteranos e bixos, pessoas de outros cursos e até mesmo de fora da
universidade. É ainda uma grande chance para perder a inibição de falar em
público e aperfeiçoar sua capacidade de expor opiniões, além de aprender como
agir em um ambiente profissional.

Para fazer parte da CONPEC fique atento à data da Palestra de Apresentação do
Processo Seletivo, que ocorre duas vezes por ano, sempre no início do semestre.
Para saber mais sobre a empresa visite o site \url{http://www.conpec.com.br/} ou
tire suas dúvidas mandando um email para conpec [@] conpec [.] com [.] br.

\section{Atlética -- AAACEC}
Assim como em outras unidades de ensino da UNICAMP, o Instituto de Computação
(IC) e a Faculdade de Engenharia Elétrica e de Computação (FEEC) tem sua
entidade que promove a prática de eventos desportivos entre os membros da
graduação e da pós-graduação: A Associação Atlética Acadêmica da Ciência
e Engenharia da Computação, mais conhecida como AAACEC, ou simplesmente
\textbf{Atlética}. A AAACEC é uma entidade sem fins lucrativos, que tem uma
diretoria composta por de alunos da computação, eleita a cada ano pelos alunos
dos cursos de Engenharia e Ciência da Computação, e pós-graduandos vinculados ao
IC.

A AAACEC é a entidade responsável pela participação da Computação em competições
esportivas, tanto dentro da UNICAMP (Calouríadas, Interanos, Olimpíadas) como
com faculdades de outras cidades (Intercomp), competições essas que costumam
acontecer uma vez a cada ano. A fim de possibilitar essa participação, a AAACEC,
além de se encarregar das inscrições e organização, promove treinos regulares de
basquete, vôlei, handball e futsal, e disponibiliza o material (bolas, redes,
etc.) para a prática de tais esportes. A AAACEC se encarrega da reserva e/ou
locação de quadras para a realização dos treinos e competições em que isso se
fizer necessário. Os treinos são semanais e oferecidos para as modalidades
masculina e feminina, de forma que qualquer associado da Atlética pode
participar.

\subsection{Como, então, se associar à AAACEC?}
No caso dos recém-ingressantes, o jeito mais simples é comprando o KIT BIXO,
organizado pela própria AAACEC, contendo camiseta, caneca e outros produtos da
Atlética. Com ele, o calouro torna-se sócio e pode usufruir livremente de todos
os jogos e/ou eventos que a AAACEC venha a organizar pelo resto de sua vida.

Quem não comprar o Kit Bixo no primeiro ano, pode se associar mais tarde com
o pagamento de uma taxa de associação. Além do Kit Bixo, a AAACEC tem diversos
produtos que podem ser comprados por qualquer associado durante todo o curso,
como camisetas, blusas, agasalhos, adesivos, chaveiros, mouse-pads, entre
outros.

\subsection{E para quem vai o dinheiro da compra do kit bixo e dos demais produtos?}
A AAACEC promove a integração entre os alunos, e a realização de experiências
sociais e esportivas:

\begin{itemize}
\item  \textbf{A famosa Choppada Comp/Fono} (gratuita para os bixos)
\item  \textbf{Churrascos}
\item  \textbf{Torneios Esportivos:}
\begin{itemize}
\item  \textit{Internos} (Torneio início, Calouríadas, InterAnos, Olimpíadas)
\item  \textit{Externos} (Intercomp, UniSinos)
\end{itemize}
\item  \textbf{Festas} (Muitas delas em união com outros cursos, promovendo ainda maior integração)
\end{itemize}

Para tudo isso, é necessário capital, que é obtido com muito trabalho da
Diretoria com a venda dos Kits Bixo e produtos.

É importante lembrar que a diretoria não tem remuneração alguma: Os diretores
trabalham pelo ideal comum de fazer a Computação crescer, e consequentemente
ganhar experiência de organização de eventos e pessoas, aprendizado de valor
inestimável, tanto na vida profissional como social.

A AAACEC busca fortalecer, acima de tudo, o nome do curso de Computação da
UNICAMP, enaltecendo nossas qualidades dentro e fora da Universidade! Orgulho de
ser Computação UNICAMP!!

Faça parte você também dessa integração! Associe-se à AAACEC!

As diretorias anteriores da AAACEC podem ser vistas em:
\url{http://aaacec.com/diretoria}

\subsection{Contato}
Não hesite em tirar dúvidas ou enviar sugestões!

\begin{itemize}
\item  E-mail: aaacec [@] ic [.] unicamp [.] br.
\item  Site: \url{http://aaacec.com/}.
\end{itemize}

\subsection{Atendimento}
Ainda não há uma definição para este semestre, mas provavelmente o atendimento
será de segunda a quinta das 12h30m às 13h30m e segunda a sexta das 18h-19h

\subsection{Reuniões}
As reuniões estão sendo realizadas todas as terças às 12h no IC-3.

\subsection{Bateria Valorosa}
Criada em 1998 a bateria valorosa é umas das melhores baterias da UNICAMP.

Durante o ano, ela participa de diversos eventos como o UPA, Intercomp, festa
das baterias, apoiando nossos atletas em jogos. Também somos convidados para
tocar para a fonoaudiologia em seus jogos, como o Interfono.

A festa das baterias é um evento organizado pela Bateria Valorosa e no último
ano trouxemos, com a ajuda da Bateria Alcalina (Instituto de Artes), a Bateria
da Nenê de Vila Matilde do carnaval de São Paulo, fazendo dessa edição a melhor
de todos os tempos!

A Bateria realiza ensaios semanais, nos quais estão todos convidados
a participar.

Esperamos que muitos bixos participem, e para isso basta comparecer aos ensaios,
lembrando que não é necessário saber tocar nenhum instrumento. TODOS que
quiserem aprender serão muito bem vindos na bateria!

\section{Centro Acadêmico da Computação}
Vulgo Centro Acadêmico da Computação, é uma entidade autônoma e sem fins
lucrativos, formada por estudantes de graduação dos cursos de Engenharia
e Ciência da Computação, e pós-graduação do IC, com o objetivo principal de
representar esses estudantes no âmbito acadêmico. Todo aluno desses cursos é um
membro do CACo.

Já ouviu falar em movimento estudantil? Então{\dots} É nóis! Muita gente tem
ideias estranhas e erradas sobre centro acadêmico (CA), diretório central dos
estudantes (DCE), movimento estudantil e coisas relacionadas a isso. Muito desse
"preconceito" é criado por alguns fatos e algumas pessoas isoladas, cujas
atitudes às vezes precipitadas acabam sendo tomadas como regra geral. O que
importa é estar aberto. Esqueça o que você acha que sabe sobre movimento
estudantil, sobre centro acadêmico. Tente pelo menos uma vez participar, estar
do lado de dentro, ver as coisas como elas são de verdade, procurar estar
presente nas reuniões, e aí sim, forme a sua opinião e critique à vontade.

Como o CACo representa os estudantes? Por exemplo, fazendo este utilíssimo
Manual do Bixo, de qualidade inigualável, inquestionável e inconteste, que tem
a pretensão de ajudá-lo neste começo de vida universitária. Tudo bem, o que
mais? Ajudamos a melhorar a qualidade dos nossos cursos através do GDA (Grupo
Discente de Avaliação), e fomentamos discussões com palestras e mesas redondas
com convidados especialistas. Realizamos assembleias com um número maior de
alunos para tomar decisões mais polêmicas e delicadas, reivindicamos espaço
físico decente para o lazer e a convivência entre os estudantes e promovemos
eventos de integração e discussão, como o CineCACo e o PipoCACo. Procuramos
facilitar a vida dos estudantes, instalando uma máquina de refrigerante
e guloseimas, e também um orelhão na entrada do IC-3. Soma-se a isso tudo
a reforma curricular da computação que está em discussão (e que conseguimos
graças à luta do centro acadêmico) e também a defesa dos estudantes junto ao IC.
Essas são algumas das atividades realizadas pelo CACo, e que serão explicadas
com maiores detalhes mais adiante.

O CACo é o seu Centro Acadêmico, então procure-nos quando você tiver algum
problema, reclamação ou sugestão para o IC ou a FEEC que o CACo irá te dar todo
o suporte necessário (daremos as informações que você precisar  e que nós
tivermos, é claro  conversaremos com professores, ou até o diretor do Instituto
caso seja preciso), assim como se você tiver alguma sugestão ou reclamação com
relação ao CACo. Portanto não tenha medo de falar conosco!

\begin{itemize}
\item  E-mail: caco [@] ic [.] unicamp [.] br.
\item  Site: \url{http://www.caco.ic.unicamp.br/}.
\item  Reuniões: Você será informado no começo do semestre sobre os horários das reuniões. Participe!
\end{itemize}

Veja abaixo alguns dos projetos que o CACo participa e/ou realiza.

\subsection{Grupo Discente de Avaliação}
O GDA (Grupo Discente de Avaliação) é a uma avaliação das matérias e professores
do curso feita pelos próprios alunos. Atualmente está em sua terceira edição
consecutiva, tendo, cada vez mais, uma maior participação dos alunos. Esse grupo
está preocupado em fazer uma avaliação quantitativa E qualitativa do ensino,
fornecendo dados suficientes para que professores e instituto, juntamente do
centro acadêmico, possam trabalhar por melhorias. Se você gostaria de ajudar
nele, entre em contato com algum membro do CACo. Para responder ao GDA ou ler
suas respostas vá à página: \url{http://www.caco.ic.unicamp.br/gda/}.

\subsection{Reunião de avaliação de curso}
A reunião de avaliação de cursos também é uma conquista dos estudantes da
UNICAMP. Essa reunião acontece uma vez a cada semestre. Geralmente as reuniões
acontecem nos meses de maio (primeiro semestre) e outubro (segundo semestre),
numa terça ou quinta-feira, no horário das 10h às 12h para os cursos
diurnos/integrais e das 19h as 21h para os cursos noturnos. Durante esses
horários, os alunos estão dispensados das aulas.

Nas reuniões de cursos, participam os alunos, os coordenadores de curso e alguns
professores, além de responsáveis pela infraestrutura de informática. Nessas
reuniões são feitas análise sobre o curso, avaliações a respeito de disciplinas,
da infraestrutura de informática e demais discussões (encontrar
deficiências/problemas e propor/sugerir soluções) para que se possa melhorar os
cursos.

Os alunos tem um papel muito importante nessas reuniões, já que são eles os
principais e maiores interessados pelo que pode sair dessas reuniões.

\subsection{Wiki de Matérias}
A Wiki de Matérias é um projeto antigo do CACo, que busca construir um banco de
dados (disponível na internet) sobre as matérias da UNICAMP. A diferença dessa
wiki das ementas e programas constantes no site da DAC é que quem coloca os
conteúdos na wiki são os próprios alunos, que podem, além de dar uma descrição
mais detalhada de cada disciplina, colocar links e explicações para certas
partes do conteúdo. O endereço da wiki de matérias é:
\url{http://www.caco.ic.unicamp.br/wiki/index.php/Wiki_de_Matérias}.

\subsection{Reforma Curricular}
Após alguma luta, os estudantes conseguiram que o IC colocasse em pauta na
Comissão de Graduação a reforma curricular dos cursos de Ciência e Engenharia de
Computação, além da criação de novas matérias e de certificados de estudos. Esta
reforma foi obtida com êxito, e já reflete no seu catálogo, ou seja, nas
matérias que você irá fazer.

Reformas curriculares são muito importantes, por manter o curso mais atualizado,
dentro dos padrões do que o mercado e as novas tecnologias existem. Os novos
catálogos podem ser vistos em:
\url{http://www.dac.unicamp.br/sistemas/catalogos/grad/catalogo2010/index.html}

Mas como o mundo e o mercado estão em constante mudança, reformas nos currículos
torna-se frequentemente necessários. Por isso o CACo está sempre buscando
melhoras nos nossos cursos.

\subsection{Caravana para o FISL}
O FISL (Fórum Internacional de Software Livre) ocorre todo ano em Porto Alegre.
É um dos maiores da categoria, reune milhares de pessoas do mundo todo e conta
com palestras dos mais renomados nomes da computação. E é claro que os
computeiros da Unicamp não podem ficar de fora! O CACo organiza todo ano uma
caravana para o fórum. Com o patrocínio dos institutos nós conseguimos fazer
a caravana mais barata de todo FISL. Mesmo para quem acabou de entrar, o fórum
é uma ótima oportunidade para entrar em contato com o mundo da computação.

\subsection{Pesquisa Salarial}
No último ano, através da colaboração do diretor do IC, o professor Hans
Liesenberg, e da rede social Reunion, promovemos uma pesquisa salarial com
ex-alunos, que ajudou a fornecer um bom panorama da realidade em que se encontra
o profissional formado pela Unicamp na área de computação. O PDF com a pesquisa
está disponível no site do CACo.

\subsection{CineCACo}
O CineCACo é também algo que foi iniciado em 2004 e que promete neste ano
funcionar a todo vapor. São apresentações de filmes, documentários e curtas que
são feitas para mostrar aos alunos que nem só de estudos se faz uma faculdade.

\subsection{PipoCACo}
O PipoCACo é um espaço de discussão do CACo, onde vários computeiros se reúnem
para discutir um tema de interesse (discussões sobre universidade pública,
eleições, reforma curricular e trote, por exemplo), sempre regado a pipoca
e guaraná.

Esse espaço também é utilizado para guiar o CACo nas decisões a serem tomadas,
como, por exemplo, o que os alunos defendereriam frente à Coordenadoria de
Graduação, acerca da reforma curricular.

O PipoCACo e o CineCACo são ótimos momentos para os bixos se conhecerem e também
conhecerem as atividades do CACo de forma mais descontraída.

\subsection{Palestra Azóide/Bzóide}
A Palestra Azóide/Bzóide é um evento aonde professores, alunos e ex-alunos expõe
e discutem as modalidades AA e AB. A palestra vem se mostrando muito importante
para a definição de qual modalidade seguir, principalmente por desmentir grandes
mitos. O principal deles é aquele que diz que AA é somente software, AB
é somente hardware. Lembrando que somente a Engenharia tem a escolha de
modalidades final do 4º semestre.

\subsection{CACo Poker Series}
Semestralmente, o CACo organiza um torneio de poker para os alunos,
possibilitando grande integração da galera. Você será informado da data de
realização do próximo torneio! Fique atento.

\subsection{Listas de e-mails do CACo}
As listas de e-mails do CACo são listas de discussão dos alunos de computação da
UNICAMP que visam a integrar os computeiros de vários anos, fomentar discussões
ou debater ideias e posições do CACo em relação aos assuntos de interesse dos
alunos que representa.

Atualmente, o CACo conta com três listas de discussão com objetivos distintos:

\begin{itemize}
\item  \textbf{Computeiros (computeiros [@] caco [.] ic [.] unicamp [.] br)}: Lista de todos as turmas de computação. Tem por objetivo integrar todos os alunos. É o lugar onde se pode pedir dicas de matérias, opiniões sobre os melhores professores para cada matéria, trocar informações e se confraternizar.
\end{itemize}

\begin{itemize}
\item  \textbf{Gestão (gestao [@] caco [.] ic [.] unicamp [.] br)}: Discussões ligadas estritamente à gestão do Centro Acadêmico.
\end{itemize}

\begin{itemize}
\item  \textbf{Offtopic (offtopic [@] caco [.] ic [.] unicamp [.] br)}: A lendária lista para discussões infinitas sobre qualquer assunto, de política, ciência e religião a Filme do Bátima.
\end{itemize}

Para participar das listas, visite a página das listas do CACo:
\url{http://sardinha.caco.ic.unicamp.br/}.

\subsection{Atendimento}
O CACo disponiliza um horário para que que você possa comprar algum de nossos
produtos, tirar dúvidas ou apenas se comunicar conosco. Para conferir quais
horários e dias da semana são nossos atendimentos, visite o site do CACo.

\subsection{Gestão}
A atual gestão do Centro Acadêmico foi eleita no fim de novembro de 2010 e deve
se prolongar até outubro deste ano. Mas lembre-se o CACo não é formado só por
seus coordenadores. O coletivo de nosso Centro Acadêmico é bem maior. Lembre-se
de que todo estudante de Computação tem o direito de dar pitaco nas discussões
do CACo. Os calouros sempre realizam funções decisivas no Centro Acadêmico.

Para ficar mais integrado ao que ocorre no seu Centro Acadêmico, como as suas
ações, projetos e quais os problemas atuais, você pode se increver na lista da
gestão do CACo através do site das listas de discussão do CACo:
\url{http://sardinha.caco.ic.unicamp.br/}

Portanto, não tenha medo! Tenha vontade! Venha participar e construir o nosso
Centro!

\subsubsection{Chapa "All In" (2010/2011)}
\begin{itemize}
\item  \textbf{Coordenador Geral de Engenharia de Computação}
\begin{itemize}
\item  André Felipe Barros Selva (Selva EC09)
\end{itemize}
\end{itemize}

\begin{itemize}
\item  \textbf{Coordenador Geral de Ciência da Computação}
\begin{itemize}
\item  Alex Bredariol Grilo (Alex CC07)
\end{itemize}
\end{itemize}

\begin{itemize}
\item  \textbf{Coodenador Geral de Pós-Graduação}
\begin{itemize}
\item  Antonio Henrique Berno Zanutto (Tunico EC07)
\end{itemize}
\end{itemize}

\begin{itemize}
\item  \textbf{Coordenadoria Financeira}
\begin{itemize}
\item  Marco Aurélio Diniz Junqueira (Peão EC08)
\item  Humberto Aboud Torres Lobo (Sheldon EC010)
\end{itemize}
\end{itemize}

\begin{itemize}
\item  \textbf{Coordenadoria Administrativa}
\begin{itemize}
\item  Juliano Siloto Assine (Bexiga EC09)
\item  José Américo Nabuco L. F. de Freitas (Jota EC010)
\end{itemize}
\end{itemize}

\begin{itemize}
\item  \textbf{Coordenador de Ensino e Graduação}
\begin{itemize}
\item  Pedro Tabacof (Tabaco EC08)
\item  Raphael de Oliveira Costa Danella (Danella EC08)
\item  Victor Fernando Pompêo Barbosa (Pompêo EC09)
\item  Marco Antônio Lasmar Almada (Marco EC010)
\end{itemize}
\end{itemize}

\begin{itemize}
\item  \textbf{Coordenador de Comunicação}
\begin{itemize}
\item  Gerson de Paulo Carlos (Gerson EC09)
\item  Thiago Rosario Caetano (Thiago EC010)
\end{itemize}
\end{itemize}

\begin{itemize}
\item  \textbf{Coordenador Tecnológico}
\begin{itemize}
\item  Carlos Polachini Zanoveli Junior (Whisky EC09)
\item  Ivan Sichmann Freitas (Ivan EC09)
\item  Rafael Timbó Matos (Rafael CC010)
\end{itemize}
\end{itemize}

\begin{itemize}
\item  \textbf{Coordenador de Eventos e Cultura}
\begin{itemize}
\item  Luiz Gonzaga de Oliveira Neto (Formiga EC09)
\end{itemize}
\end{itemize}

\begin{itemize}
\item  \textbf{Coordenador de Patrimônio}
\begin{itemize}
\item  Leonardo Francisco (Leo EC09)
\item  Pedro Henrique D. de Vasconcelos Affonso (Pedro EC010)
\end{itemize}
\end{itemize}

\begin{itemize}
\item  \textbf{Coordenador de Marketing e Produtos}
\begin{itemize}
\item  Fernando Lucchesi Bastos Jurema (Tomitinha EC09)
\item  Filipe Carvalho Xavier (Feel EC09)
\end{itemize}
\end{itemize}

As gestões anteriores do Caco podem ser vistas em:
\url{http://www.caco.ic.unicamp.br/institucional/gestões}

\section{Jornal da Unicamp}
O jornal da UNICAMP é um jornal de distribuição semanal distribuído dentro da
UNICAMP. Pode-se encontrar diversos exemplares do jornal na sala de computadores
que fica ao lado da DAC, no IC-2 e em diversos pontos espalhado pelo campus.

No jornal são mostradas as diversas pesquisas realizadas pela universidade,
eventos, livros e obras de autoria de professores, aparições da UNICAMP na
imprensa e defesas de mestrado e doutorado.

Na página do jornal (\url{http://www.unicamp.br/ju}) pode-se encontrar as
últimas edições do jornal (em PDF), além de poder fazer a assinatura online do
jornal, distribuído em formato PDF. A assinatura e distribuição do jornal são
gratuitas.

\section{Eventos da Unicamp}
\subsection{Feira do Livro}
Evento iniciado em 2002, há cada dois anos a Editora da UNICAMP realiza a Feira
do Livro, no Ginásio da UNICAMP. Há excelentes opções, de diversas editoras,
principalmente na área de literatura, artes e humanas, com no mínimo 50\% de
desconto.

\subsection{Semanas da UNICAMP}
Alguns cursos da UNICAMP realizam anualmente um evento (chamado de semana) em
que os alunos de graduação tem um contato com o mercado de trabalho, com as
pesquisas, com as tendências e novidades dos cursos e demais assuntos, ramos
e áreas de cada curso. Para isso, participam desse evento ex-alunos
e profissionais, realizam-se palestras e minicursos, são feitas visitas
a empresas e são feitos debates ("mesas redondas").

A computação tem a sua semana. É o Computação e Mercado (C\&M), que é realizado
pela conpec desde 1993.

\subsection{Talento}
Talento é um evento que acontece todo ano, desde 1999, durante um dia inteiro,
no ginásio multidisciplinar e organizado pelo núcleo de empresas juniores da
UNICAMP. Trata-se de uma feira de recrutamento, aonde alunos da UNICAMP e de
outras universidades, e o público em geral tem contato com empresas, seja por
meio de palestras, mesas redondas e estandes, e estas apresentam o seu processo
seletivo. Nesse evento também é feito o cadastro de currículos dos visitantes.

Para saber mais sobre o evento, é só acessar a página do talento
(\url{http://www.talentounicamp.com.br/}). O evento é gratuito.

\subsection{UPA -- UNICAMP de Portas Abertas}
Desde 2003 acontece a UPA (UNICAMP de Portas Abertas). Esse evento é um
aprimoramento de um evento que acontecia há alguns anos atrás, a Universidade
Aberta ao Público.

A UPA é um evento anual, em que, durantes dois dias (geralmente dois dias do mês
de setembro) a UNICAMP é apresentada para estudantes dos ensinos fundamental
e médio de todo o país (muitos desses alunos pré-vestibulandos). A apresentação
da universidade é feita por professores e alunos, que mostram as salas de aula
e as pesquisas realizadas.

No ano de 2007, a UPA recebeu a visita de 50 mil alunos de 950 escolas públicas
e privadas de 10 estados.

Para saber mais sobre o evento, é só acessar a página
\url{http://www.upa.unicamp.br/}.

\section{Pra que que eu estou estudando isso??}
O ensino médio acabou, você finalmente está livre de todas as inutilidades, como
química orgânica e separação silábica de verbos parnasianos, só vai ver coisas
relevantes para a profissão, e{\dots}

Pimba! HZ291. Pode, Arnaldo?

Primeiro, você precisa saber<del>-saber que até mesmo o sanduíche-íche</del> que
a Universidade não é um curso técnico. A ideia não é só te dar capacitação
profissional, mas sim formar pessoas melhores(inb4 blablabla). Para que um
computeiro precisa de contabilidade? Para nada, mas uma pessoa(de exatas pelo
menos) precisa ter uma noção disso.

Outro problema: o que exatamente é "relevante para a sua profissão"?
A computação é uma área muito vasta, e a graduação(o curso que você tá
fazendo{\dots}) é bem generalista, para te dar base para escolher. Por exemplo,
vai ter gente que nunca mais vai usar GA/Algelin, mas quem for para a área de
computação gráfica vai comer matriz no café da manhã. Quem garante que no meio
do curso você não decida ir para essa área? Ou ainda, que no seu emprego não te
joguem um problema desse tipo?

Se você continuar na Universidade, na pós você só terá matérias da sua área, já
que você já sabe o suficiente pra dizer que área é essa. Mas ainda falta muito
chão até lá{\dots}

Pra quem é da engenharia, um problema maior é que, para conseguir o CREA,
existem algumas matérias obrigatórias(embora completamente inúteis, sem
exagero), como ResMat. A Unicamp até pode contrariar essas orientações, até
certo ponto, mas dificilmente os professores concordariam.  (por outro lado,
você poderá construir prédios de até 2 andares. recomendamos fortemente que você
não faça isso)

Para quem é da ciência, o curso não é para formar "programadores". Vocês serão
mais que isso, serão cientistas, e isso envolve ver coisas além de só código.

(mas fique tranquilo que, independente do curso, você terá que sofrer muito
programando <del>MWAHAHA</del>)

Nesse manual não temos muito espaço pra falar em detalhes. Para dar uma noção
mais especifica da utilidade(ou não) de cada matéria, fizemos uma wiki, que você
pode acessar aqui:
\url{http://www.caco.ic.unicamp.br/wiki/index.php/Wiki_de_Mat%C3%A9rias}

\end{document}
