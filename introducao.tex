% Este arquivo .tex será incluído no arquivo .tex principal. Não é preciso
% declarar nenhum cabeçalho

\section{Para pensar}

O objetivo desse tópico não é responder grandes questões, nem doutrinar ninguém,
apenas expor alguns assuntos que achamos importante que você, bixo, analise
neste momento em que você está iniciando sua vida na universidade.

\subsection{Unicamp, uma Universidade Pública?}

À primeira vista, um título como esse pode parecer estranho. "Mas como? Acabei
de entrar na Unicamp, uma universidade estadual e não vou pagar mensalidades
para estudar nela. É óbvio que ela é uma Universidade Pública". Será?

Público, no sentido do dicionário, refere-se a "todos", ao "povo", mas também no
sentido popular quando nos referimos a algo público, logo lembramos de acesso
irrestrito e gratuito, sem distinção alguma a todos os cidadãos e cidadãs, um
lugar que todos tem direitos de usufruir. Mais ainda, público significa que
todos pagam, que é gerido e que foi construído com o dinheiro de todas as
pessoas através dos impostos arrecadados pelos mecanismos do Estado. É isso que
acontece por exemplo com os hospitais, creches e escolas públicas que, embora
possuam atendimento muitas vezes precário, atendem a todos.

Mas e uma universidade, de que forma ela se encaixa dentro do "público"?

Para responder melhor essa questão, devemos analisar a universidade com base em
seu tripé essencial: ensino, pesquisa e extensão. Comecemos por analisar
o acesso ao ensino. Por um lado, o processo seletivo da universidades públicas
paulistas, o vestibular, é uma forma virtualmente imparcial de seleção, que deve
analisar apenas o nível de conhecimento do candidato (que muitas vezes está
ligado ao valor de sua renda), sem levar em conta a religião, o sexo, ou
a classe social do vestibulando. Logo, exceto por pontos extras no vestibular
(para negros, índios e estudantes de escola pública), para ingressar em uma
universidade como a Unicamp não há nenhuma distinção entre os cidadãos. Por
outro lado, só no Estado de São Paulo há cerca de 2 milhões de jovens com idade
para estar na universidade, enquanto isso as três universidades estaduais
paulistas (USP, UNESP e Unicamp) não chegam a oferecer 100 mil vagas de
graduação e pós. Portanto, nem 5\% dos jovens paulistas tem acesso ao ensino
público superior, sendo que a grande maioria dos ingressantes na universidade
pública pertence às classes média ou alta. Existem iniciativas pontuais que
buscam mudar este quadro: Cursinhos comunitários (como a Cooperativa do Saber,
Cursinho do Sindicato, dentre outros) tem obtido um bom nível de aprovação de
seus alunos, porém em cursos com baixa concorrência, sendo que cursos como
medicina e computação continuam praticamente restritos às classe mais altas.

Agora, tendo em mente a pesquisa, podemos dizer que a Unicamp se destaca nesse
ponto: Ela é responsável por 15\% de toda a pesquisa brasileira, desenvolve
vários estudos sobre a sociedade brasileira (tendo publicado recentemente um
atlas social), projetou equipamentos de segurança para carros, está trabalhando
em sistemas computacionais para saúde, entre tantas outras pesquisas voltadas
para o progresso da ciência e tecnologia nacional. Isso, sem contar que
a Unicamp foi responsável por 35\% das patentes registradas na década de 90 aqui
no Brasil. Mas grande parte dessas patentes são vendidas a grandes empresas,
muitas vezes multinacionais. Outro ponto crítico existente são as pesquisas
particulares desenvolvidas na Unicamp (a sala .NET, no IC, é um exemplo disso,
onde as patentes são divididas entre universidade, pesquisador e Microsoft).

Finalmente, falaremos sobre a extensão. Na nossa universidade o principal
projeto de extensão é a administração de hospitais da região, como o HC, que
é o maior hospital público do interior paulista, e atende pessoas de toda
a região e até de fora do Estado. Fora ele, a universidade não desenvolve nenhum
projeto de extensão gratuito que tenha grande destaque. No Instituto de
Computação, por exemplo, os cursos de extensão custam mais de R\$ 2000,00 por
aluno, o que não pode realmente ser chamado de extensão universitária, uma vez
que não está distribuindo à comunidade o conhecimento produzido aqui dentro.
Além disso a Unicamp também restringe o acesso a diversos espaços da
universidade, como por exemplo o controle do acesso noturno ao campus,
a coibição de festas, e a dificuldade de acesso aos espaços da Faculdade de
Educação Física. Outras universidades como a UNESP e a USP Leste tem uma
política bem melhor de extensão. Na Unicamp alguns alunos tem se dedicado
a desenvolver projetos de extensão que busquem a inclusão da comunidade na
universidade, através de cursinhos pré-vestibulares, reforço escolar,
e atividades com a população carente. A Computação historicamente tem pouco
engajamento, sendo que um dos únicos projetos com destaque é o GPSL, que
desenvolve a integração digital através do Software Livre, mas atualmente está
meio parado.

Resgatar o sentido do público tanto conceitual quanto materialmente se faz
sempre necessário. Assim, desde já, participe das discussões do centro acadêmico
para questionar as deficiências e produzir novos caminhos para formarmos com
a colaboração de todos uma universidade cada vez mais pública.

\subsection{Eu, um estudante público?}

Você já parou para pensar o que está começando ocorrer em sua vida? A partir de
agora você estuda em uma universidade pública, ou, como já foi dito, a partir de
agora o povo está pagando para você estudar. E o que você fará com esse
privilégio?

Se você for perguntar, encontrará milhares de maneiras de encarar o fato de ser
um estudante público, e provavelmente algumas das respostas provavelmente
incluiriam um pouco das visões a seguir:

Alguns veem a aprovação como o último passo do desafio de entrar na
universidade, uma conquista pessoal, e, desta maneira, a única pessoa a quem
estes tem de prestar contas sobre o que fizeram de seus estudos na Unicamp
seriam eles mesmos. Outros acreditam que estudar em uma universidade pública
traz uma responsabilidade direta: estudar corretamente. Um aluno da Unicamp
teria de aproveitar a universidade ao máximo, buscando sempre aprender, para que
saia daqui como um profissional competente para auxiliar o progresso tecnológico
da nossa sociedade, de modo a cumprir com o papel que lhe foi atribuído. Também
existem algumas pessoas que acreditam que assim que entramos na universidade nos
tornamos agentes públicos. Sendo assim, além de estudar também seria papel do
aluno interagir constantemente com a comunidade passando a ela os conhecimentos
que a universidade lhe proporcionou, buscando criar um elo
universidade-comunidade.

Afinal, qual dessas visões seria a mais correta sobre o que é um estudante
público? Essa é uma resposta que não daremos aqui (até porque, como foi dito no
início, respostas não fazem parte do objetivo deste tópico), ela é algo
individual. Mas seria bom que você pensasse qual a razão pela qual você está na
Unicamp e qual o objetivo da sociedade quando ela paga para que você tenha essa
oportunidade.

E, bixo, nunca é demais desejar: Que você faça o melhor proveito do seu tempo
aqui na Unicamp!
