% Este arquivo tex vai ser incluído no arquivo tex principal, não pe preciso
% declarar nenhum cabeçalho

\section{Iniciação Científica}

A Iniciação Científica é um tempo para o aluno de graduação (você, no caso) ter
uma experiência acadêmica mais séria, sentir um pouco como é o clima de
pesquisa. Interessou? O que fazer? Calma, você mal entrou na Universidade.
Geralmente o que se faz é conversar com o professor da área com a qual você se
identifica mais (criptografia, teoria da computação, processamento de imagens,
inteligência artificial, etc), e ver se ele está desenvolvendo algum projeto
interessante naquela área, ou propor para ele alguma idéia sua original mesmo.
Depois você começa a estudar para redigir um projeto e encaminhar para alguma
instituição de fomento à pesquisa (CNPq ou FAPESP), pedindo uma bolsa de
Iniciação Científica. A FAPESP paga R\$502,80 e aceita pedidos de bolsa em
qualquer período do ano. O CNPq paga aproximadamente R\$360,00, e o período para
inscrição de projetos é geralmente em junho e novembro. No primeiro semestre
geralmente é bem mais difícil achar algum professor da área que você se
interessa, aliás, é bem difícil saber a área com a qual você se identifica, pois
você mal começou o curso e não conhece muito do que se estuda em Computação,
muito menos os professores. Mas tenha paciência, agora parece tudo muito
complicado e complexo, mas com o tempo, as coisas vão ficando mais simples. Se
você realmente tiver uma sede insaciável de conhecer o meio da pesquisa, procure
o seu professor de MC102, ele pode te orientar a respeito.

Outra coisa interessante a respeito da Iniciação Científica é que, se você
conseguir bolsa (o que não é muito difícil), pode pegar a disciplina MC040
e posteriormente MC041 (2 semestres), cada uma com 6 créditos, e geralmente se
tira notas altas nessas disciplinas. Ou seja, são 12 créditos praticamente "de
bandeja" para aumentar o seu CR, ou ajudar a recuperá-lo, caso esteja no fundo
do poço. Note bem as aspas. Os trabalhos de Iniciação Científica geralmente
consomem muito tempo de estudo e dedicação, não vá pensando que é moleza não.

Na FEEC você pode conseguir as matérias de Iniciação (EA002 até EA005) mesmo sem
bolsa, mas você ainda assim vai precisar de um orientador. Lá a Iniciação
Científica também substitui o estágio, mas não tem equivalência com as do IC.

\section{Intercâmbio}

Como assim? Acabei de sair da minha cidade natal, do aconchego do meu lar e você
vem me falar de sair do país? Calma, calminha{\dots} Não é para agora, mas é bom
ir se preparando {\dots} Falando sério, uma introdução sobre intercâmbios.
A Unicamp hoje, junto com a USP, é uma das universidades brasileiras que tem
maior prestígio fora do país e o Cori (Cooperação Internacional), IC e FEEC tem
vários acordos bilaterais de intercâmbio. Então, para você que quer dar um salto
em algum idioma, conhecer outras culturas e sentir na pele a aventura de ser
estrangeiro, comece a se preparar desde já.

A primeira coisa é escolher uma língua. Pois, mesmo o inglês sendo a língua que
se fala no mundo, não é 100\% da população do país de destino que fala o inglês
corretamente, então pense em começar a estudar a língua do país em que pretende
ir no seu intercâmbio. Hoje, a maior parte dos computeiros vai para França,
Alemanha e países hispanofônicos, logo pensar em fazer francês, alemão ou
espanhol é uma boa pedida e com chances de facilitar sua aceitação do pedido de
intercâmbio. Existem também bolsas para estudos na Itália, então aprender
italiano pode ser uma boa alternativa. Ter um bom nível de inglês também ajuda
no processo seletivo e, claro, inglês não faz mal a ninguém (ainda mais a nós
do ramo de tecnologia)!

Para onde ir? Grande questão! Vou citar o caso dos EC03: de 90 alunos, 10 alunos
foram estudar/estagiar fora! Destes, 6 foram para França, 1 para Portugal,
1 para Itália, 1 para Eslovênia e 1 para Argentina.

A França hoje recebe grande número de estudantes de computação, devido a acordos que
a FEEC tem com os INSA's e com as écoles Centrales e também graças a bolsas de
estudo oferecidas pela Capes e pelo governo francês. Mas não significa que
a França seja seu destino certo: no site do CORI existem oportunidades para ir
para, além dos países já citados, Japão, América Latina, Alemanha, Espanha, etc.
Muitos com boas bolsas de estudo ou com incentivos que valem a pena caso você
tenha um pouco de grana para se sustentar no início. Visite sempre o site da
CORI e participe das reuniões que ela faz, pois ficar ligado é a chave para conseguir
encontrar uma boa oportunidade. Existem outras opções de intercâmbio, como
a AIESEC, que promove um intercâmbio para estágios no exterior. Se seu interesse
é mais profissional, procure se informar.

Mas vale a pena? Poxa, vou atrasar meu curso, ficarei deslocado de turma, vou
ficar em um país estranho, para quê? Acho que não vale a pena{\dots}

{\dots}Ledo engano, ledo engano{\dots}

Vamos começar pelos motivos profissionais : ter no currículo que você fala uma
língua estrangeira fluentemente devido a sua imersão no país é algo muito
valorizado pelas empresas, além do fato que o pessoal do RH vai ver que você tem
capacidade de se virar sozinho, uma vez que não é tão óbvio sair do país
e recomeçar sua vida fora. Você não atrasará tanto seu curso, pois a Unicamp
conta com um sistema de equivalências de matérias e se você escolher bem pode
fazer matérias que serão convalidadas na Unicamp. Agora, o que realmente
é importante: Você está na faculdade, está na hora de deixar o colo da mamãe
e partir para o mundo{\dots} a experiência de conhecer outras culturas, criar
laços de amizade (ou mesmo algo mais) internacionais, viajar por terras
desconhecidas não tem preço!!! Pense que é no seu tempo de facul que terá
oportunidade de fazer uma aventura destas, não desperdice{\dots}

Se você abriu um sorriso e pensa que está preparado para sair do país{\dots}
comece a estudar, garoto! Não que ter uma boa nota seja a única forma de
conseguir uma vaga em uma bolsa de estudos, mas com certeza é a mais fácil.
Busque "atirar em todas as frentes", mantenha seu CR num bom nível, busque
conhecer organismos como AIESEC e procure grupos de trabalho (no IC existem
vários) que podem levar alunos ao exterior. Boa sorte!

Se você realmente se interessou, ai vão uns links com mais informações:

\begin{itemize}
\item  AIESEC: \url{http://www.aiesec.org/brazil/}
\item  CORI: \url{http://www.cori.unicamp.br/}
\end{itemize}

Fique atento aos e-mails que você receberá do IC e da FEEC. Muitos deles são sobre programas de intercâmbio.

\section{Eletivas e Proficiência}

A UNICAMP oferece a seus alunos a oportunidade de personalizar seu currículo de
acordo com seu interesse por meio das disciplinas eletivas. Ao contrário das
disciplinas do núcleo comum, você pode escolher a matéria que vai cursar. Alguns
créditos podem ser cumpridos com qualquer disciplina oferecida pela
universidade, outros estão restritos a um determinado conjunto (para mais
detalhes, consulte seu catálogo em
\url{http://www.dac.unicamp.br/portal/grad/catalogos/index.html},
na página destinada à Graduação).

Mas não se esqueça de que com um grande poder vem uma grande responsabilidade!
A UNICAMP lhe dá liberdade para escolher o melhor jeito de se preparar para seu
futuro, e espera que você saiba o que fazer com ela. Você pode socializar com
outros cursos, aprender uma língua estrangeira, assistir a seminários ou obter
um certificado de estudos na FEEC ou no IC.

Muitas pessoas, especialmente nos cursos de computação, fazem proficiência em
línguas, em alguns casos para evitar o jubilamento, outros para não ter que
passar mais um semestre na faculdade. Apesar de registrar a familiaridade do
aluno com uma outra língua em sua integralização, essa prática não enriquece
a graduação de nenhum estudante que faça tal escolha. Além disso, muitos bixos
arrependem-se de terem feito a prova e perdido preferência na hora de pegar uma
disciplina interessante, podendo até mesmo não conseguir se matricular.

Converse com seus amigos e veteranos para descobrir o melhor jeito de usufruir
dessa liberdade que poucas universidades oferecem! Dificilmente você não
encontrará algo com o qual se identifica ou que não ensine lições interessantes.

\section{Cuidado com CR e Reprovação}

Durante o curso você vai ouvir que se preocupar com seu CR é bobagem, que
estudar para tirar nota não leva a lugar nenhum, que depois de formado não é seu
CR que te colocará no mercado de trabalho, etc. Cuidado, a sua única opção de
vida após formado não é trabalhar numa empresa, e preste atenção, pois "após formado"
não significa durante o curso.

Durante o curso você vai ter a possibilidade de participar de várias atividades
acadêmicas e algumas delas vão exigir bom aproveitamento acadêmico do aluno. Por
exemplo, para se candidatar a monitor de uma disciplina é exigido do aluno um CR
>= 0,7. Isto significa que sua média de notas das disciplinas cursadas até
aquele momento deve ser maior ou igual a 7,0. Para pleitear uma bolsa de
iniciação científica, onde há concorrência entre alunos do país todo, também
será exigido bom aproveitamento, assim como para uma bolsa de mestrado. Caso
você não saiba, mestrado faz parte da pós-graduação, ou seja, o seu CR vai te
influenciar até após formado.

Cuidado também com a reprovação, há instituições, como a FAPESP (Fundação de
Amparo à Pesquisa do Estado de São Paulo), que é a maior fomentadora de pesquisas
do estado de São Paulo e que paga os maiores valores de bolsas do país, que te
excluem de qualquer disputa só por ter uma reprovação no seu histórico escolar da
graduação. Não te exclui oficialmente, mas como é muito concorrido por ser
a melhor pagadora, se seu nome for para o final da lista por causa da reprovação
pode considerar-se excluído da disputa.

Parece óbvio que quem estuda tira boas notas, mas até você aprender a estudar como
a universidade exige pode demorar um pouco e há pessoas que nunca aprendem.

Há certos períodos (semestres) quando você já estiver mais avançado no curso em
que poderá sentir-se à vontade para desistir de uma disciplina em que esteja
matriculado, deixando para completá-la posteriormente. Quando você fizer isso há
a possibilidade de desistir da disciplina, desmatriculando-se oficialmente dela.
Mas há pessoas que simplesmente deixam de cursar a disciplina, reprovando por
nota e falta e ficando com uma nota baixa em seu histórico. Cuidado com isso,
pode ser frustrante para você no futuro. Por isso, se for desistir de cursar uma
disciplina após matriculado, sempre peça a desistência e tente não reprovar.

Lembre-se de que 4 ou 5 anos é muito tempo, você pode mudar de ideia a qualquer
momento sobre o que pretende fazer no futuro.
