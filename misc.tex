% Este arquivo tex vai ser incluído no arquivo tex principal, não pe preciso
% declarar nenhum cabeçalho

\section{Outras necessidades}
\subsection{Supermercados}

No centro de Barão há três supermercados: o Super Barão (conhecido também como
Super Ladrão ou Super Carão), o Pão-de-Açúcar e o Dalben. Ambos Super Barão
e Pão-de-Açũcar são muito caros, mas o Barão tem boas promoções de segunda
e terça-feiras. Cada um é melhor/mais barato para alguma coisa. Só indo bastante
em cada um é que você pega o jeito. O Dalben é novo, abriu em 2007, é maior
e geralmente tem coisas mais baratas que o Pão-de-Açúcar. Às segundas e terças
(no Dalben e no SuperBarão) e às terças (no Pão de Açúcar) há preço promocional
de frutas e verduras. O Dalben localiza-se no alto da Avenida 1, na entrada de
Barão Geraldo. O Barão funciona de segunda a sábado das 7h às 22h e domingos das
8h às 20h e o Pão de Açúcar funciona de domingo a quinta-feiradas 7h às 23h,
sextas e sábados das 7h às 24h. Na moradia, há outro Super Barão e o Benatti,
e no Real Parque outra loja do Super Barão.

Há também lojas de conveniência (como as AM/PM em frente à Unicamp e no Rio das
Pedras, o StarMart no Texaco do fim da avenida 2 e a do Shell em frente ao
terminal, a mais barata das 4) que embora mais caras que os supermercados, podem
funcionar em horários que esses não e/ou serem mais perto de onde você mora.

Em frente ao terminal Barão há o Recanto Bela Fruta, que não é exatamente um
supermercado, mas é o melhor lugar de Barão para comprar frutas, verduras
e legumes. Vende também algumas coisas de supermercado como pão, leite, iogurte,
mupy etc. Neste mesmo esquema funciona o varejão Oba, que fica na avenida Santa
Isabel, perto da pizzaria Sapore. Na estrada da Rhodia, depois da Padaria Di
Capri, há a frutaria Rio das Pedras, que tem preços no nível do Pão de Açúcar
e do Super Barão, mas é mais próximo pra quem vive na Cidade Universitária 2,
por exemplo.

Se você tiver carro, ou conhecer alguém que tenha, junte um pessoal e faça suas
compras no Tenda Atacado ou no Atacadão. Os dois são muito mais baratos que
qualquer supermercado e tem MUITA coisa. Ah, e nem tudo é atacado, vende umas
coisas a varejo lá também. Para chegar aos dois, saindo de Barão pelo Tapetão,
pegue a D. Pedro sentido Anhanguera. Assim que você entrar na D. Pedro, você já
vai enxergar o Atacadão, à direita. Para ir ao Tenda siga mais um pouco na D.
Pedro e saia logo depois do CEASA à direita.

Outras opções para quem tem carro são: O Carrefour, localizado na Rodovia D.
Pedro e o Wal-Mart, que fica no Shopping Parque D. Pedro.

\subsection{Utilidades em geral}

\begin{itemize}
\item  \textbf{Ki-Água}
\begin{itemize}
\item  Telefone: (19) 3289-4659
\item  Endereço: Rua Eduardo Modesto, 240
\end{itemize}
\end{itemize}

\begin{itemize}
\item  \textbf{Circuito das Águas}
\begin{itemize}
\item  Telefone: (19) 3289-4930
\item  Endereço: Rua Júlia Leite de Barros, 182
\end{itemize}
\end{itemize}

\begin{itemize}
\item  \textbf{Água Mineral Serrana}
\begin{itemize}
\item  Telefone: (19) 3289-4602
\item  Endereço: Rua Albino José Barbosa de Oliveira, 658
\end{itemize}
\end{itemize}

\begin{itemize}
\item  \textbf{Real Gás}
\begin{itemize}
\item  Telefone: (19) 3289-1786
\item  Endereço: Rua Eduardo Pereira de Almeida, 570
\end{itemize}
\end{itemize}

\begin{itemize}
\item  \textbf{Genebra Gás}
\begin{itemize}
\item  Telefone: (19) 3289-3622
\item  Endereço: Avenida Santa Isabel
\end{itemize}
\end{itemize}

\begin{itemize}
\item  \textbf{Drogaria Vitória}
\begin{itemize}
\item  Telefone: (19) 3289-9926
\item  Endereço: Avenida Ruberley Bueretto da Silva, 1015
\end{itemize}
\end{itemize}

\begin{itemize}
\item  \textbf{Drogaria Nova Barão}
\begin{itemize}
\item  Telefone: (19) 3289-6191
\end{itemize}
\end{itemize}

\begin{itemize}
\item  \textbf{Rede Farmaxima}
\begin{itemize}
\item  Telefone: (19) 3289-2824 / (19)3289-4054
\item  Endereço: Avenida Dr. Romeu Tórtima, 255
\end{itemize}
\end{itemize}

\begin{itemize}
\item  \textbf{Drogaria Sidarta}
\begin{itemize}
\item  Telefone: (19) 3289-1999
\item  Endereço: Avenida Prof. Atílio Martini, 190
\end{itemize}
\end{itemize}

\begin{itemize}
\item  \textbf{New Laundry Lavanderia}
\begin{itemize}
\item  Telefone: (19) 3289-3922
\item  Endereço: Rua Francisca Resende Merciai, 54
\end{itemize}
\end{itemize}

\begin{itemize}
\item  \textbf{Desentupimento 24 horas}
\begin{itemize}
\item  Telefone: (19) 3242-1249
\end{itemize}
\end{itemize}

\begin{itemize}
\item  \textbf{Carpintaria São Jorge de Barão}
\begin{itemize}
\item  Telefone: (19) 3289-3399
\item  Endereço: Avenida Santa Isabel, 1882
\end{itemize}
\end{itemize}

\begin{itemize}
\item  \textbf{CPFL (companhia de luz)}
\begin{itemize}
\item  Telefone: 0800-010-1010
\end{itemize}
\end{itemize}

\begin{itemize}
\item  \textbf{Sanasa (companhia de água)}
\begin{itemize}
\item  Telefone: 0800-772-1195
\end{itemize}
\end{itemize}

\begin{itemize}
\item  \textbf{Luigi Cabelereiro}
\begin{itemize}
\item  Telefone: (19) 3288-0198
\end{itemize}
\end{itemize}

\begin{itemize}
\item  \textbf{João Cabelereiros}
\begin{itemize}
\item  Telefone: (19) 3289-3084
\item  Endereço: Rua Orácio Leonardi, 92
\end{itemize}
\end{itemize}

\begin{itemize}
\item  \textbf{Armazém da Foto}
\begin{itemize}
\item  Telefone: (19) 3289-0975
\end{itemize}
\end{itemize}

\begin{itemize}
\item  \textbf{Empresa de Correios e Telégrafos (Correios)}
\begin{itemize}
\item  Endereço 1: Avenida Santa Isabel, 218 -- Centro de Barão.
\item  Telefone 1: (19) 3288-0244.
\item  Endereço 2: Rua Carlos Gomes, 241 -- Campus da Unicamp (próximo ao ginásio e Instituto de Artes).
\item  Telefone 2: (19) 3289-7288 e (19) 3521-7642.
\end{itemize}
\end{itemize}

\subsection{Bancos}

\begin{itemize}
\item  \textbf{Banco do Brasil:} Com a compra da Nossa Caixa pelo BB, há 3 agências: Uma localizada perto da Reitoria, outra no centro de Barão, perto do Santander, e outra no centro de Barão perto de onde era o Super Barão. Há caixa eletrônico onde antigamente era a agência do BB perto dos Correios.
\end{itemize}

\begin{itemize}
\item  \textbf{Bradesco:} Agência no centro de Barão, do lado do Banco do Brasil e em frente do Santander. Próxima ao Terminal Barão.
\end{itemize}

\begin{itemize}
\item  \textbf{Caixa Econômica Federal:} Localizada na frente do Pão de Açúcar.
\end{itemize}

\begin{itemize}
\item  \textbf{Citibank:} Localizado no alto da avenida 2, próximo ao posto Texaco.
\end{itemize}

\begin{itemize}
\item  \textbf{Itaú:} Possui uma agência na Unicamp, próximo à Reitoria e a ex agência do Banco Real (agora agência do Santander), e outra na avenida Santa Isabel, do lado da sorveteria. Existe também o Itaú Personalité, próximo ao Pão de Açuar e ao McDonalds. E com a compra do Unibanco pelo Itaú, 
agora existe também uma agência na rua onde era o Super Barão, perto da lotérica.
\end{itemize}

\begin{itemize}
\item  \textbf{Santander:} Com a compra do Banco Real pelo Santander, há quatro agêncais em Barão: Duas localizadas ao lado da Reitoria (uma delas era a agência do Real), outra na praça do Ciclo Básico e a outra próxima ao Terminal Barão. Há também caixas eletrônicos, ligados à rede 24 horas, espalhados pela Unicamp: Na entrada do IMECC, na entrada da FEF, na entrada do IE, na Reitoria, ao lado do xerox do IFGW e no Ciclo Básico.
\end{itemize}

\subsection{Sebos e livrarias}

Em Barão há três sebos: O Curupira, o Cronópio e o Galpão. Geralmente você não
encontra muita coisa boa de computação neles, mas não custa procurar. O Curupira
fica na rua do Terminal, bem em frente a ele. Não é difícil achar. O Galpão fica
perto do Terminal. Saindo do terminal pela avenida marginal à Albino de
Oliveira, vire a primeira à esquerda e a segunda à direita. Funciona de segunda
à sexta até às 18h. O Cronópio fica na mesma rua do Galpão, só que bem longe,
próximo à padaria Fiori. Seguindo a Santa Isabel, vindo do Centro de Barão, vire
à esquerda na esquina que tem uma pizzaria (antes da Fiori). Vire na primeira
à esquerda e você está no Cronópio (Lá também é um restaurante barato
e gostoso).

Na Unicamp há a livraria Toledo, que fica na Faculdade de Educação (Pedago), tem
pouca coisa de Computação lá mas alguma coisa de Matemática. Também há
a livraria da Unicamp (no IEL), tem preços bons mas não tem quase nada de
Exatas. Já a Livraria da Química tem livros de exatas, e geralmente eles
conseguem importar a um preço bom. Outras livrarias são a Fnac (no Shopping D.
Pedro), a Saraiva e a Cultura (no Shopping Iguatemi) e a Siciliano (no Shopping
Galeria).

Mais uma dica: Antes de comprar um livro, veja se você vai usar muito ele, se
não tem bastante na biblioteca, se algum veterano legal não pode te emprestar ou
te vender, ou se não rola tirar xerox. Você compra livro só se você precisar
e/ou quiser muito. Caso vá comprar, atente para livrarias virtuais, que podem
ter preços muito menores (pesquise sempre no Buscapé
(\url{http://www.buscape.com.br/})) e emitem boletos para você pagar no banco.
Em alguns casos, a livraria da Editora (localizada no piso térreo da BC) pode
trazê-lo por um preço menor ainda -- consulte sempre!

Uma última dica: Não compre o livro de G.A. (MA141). Estude pelo Stewart II
(livro de Cálculo II), pelo Steinbruch ou pelo livro do Paulo
Boulos. Os três são muito melhores que ele. Se você quiser os exercícios para
estudar para os testes pegue na internet ou tire xerox. Use os outros livros
para estudar.

\begin{itemize}
\item  \textbf{Sebo Curupira}
\begin{itemize}
\item  Endereço: Avenida Albino José B. de Oliveira, 980
\item  Telefone: (19) 3289-7522
\end{itemize}
\end{itemize}

\begin{itemize}
\item  \textbf{Sebo Galpão}
\begin{itemize}
\item  Endereço: Rua Francisco Barros Filho, 16
\item  Telefone: (19) 3289-2044
\end{itemize}
\end{itemize}

\begin{itemize}
\item  \textbf{Sebo Valise de Cronópio}
\begin{itemize}
\item  Endereço: Rua Francisco de Barros Filho, 426
\item  Telefone: (19) 3289-0028
\end{itemize}
\end{itemize}

\subsection{Bicicletarias}

\begin{itemize}
\item  \textbf{Bicicletaria Barão}
\begin{itemize}
\item  Endereço: Avenida Santa Isabel, 446.
\item  Telefone: (19) 3249-0449.
\end{itemize}
\end{itemize}

\begin{itemize}
\item  \textbf{Rei do Pedal}
\begin{itemize}
\item  Endereço: Avenida Santa Isabel, 74.
\item  Fone: (19) 3289-9258.
\end{itemize}
\end{itemize}

\begin{itemize}
\item  \textbf{Via Bike}
\begin{itemize}
\item  Endereço: Avenida Albino José Barbosa de Oliveira, 1074.
\item  Telefone: (19) 3289-9888.
\end{itemize}
\end{itemize}

\subsection{Escolas de idiomas}

\begin{itemize}
\item  \textbf{CCAA}
\begin{itemize}
\item  Endereço: Rua Professor Luciano Venere Decourt, 290.
\item  Telefone: (19) 3249-0202.
\item  E-mail: barao [@] ccaa [.] com [.] br
\item  Site: \url{http://www.ccaa.com.br/barao}
\end{itemize}
\end{itemize}

\begin{itemize}
\item  \textbf{CCBEUC -- Centro Cultural Brasil Estados Unidos -- Campinas}
\begin{itemize}
\item  Endereço: Avenida Romeu Tórtima, 531.
\item  Telefone: (19) 3249-0275.
\item  E-mail: admbg [@] ccbeuc [.] com [.] br
\item  Site: \url{http://www.ccbeuc.com.br/}
\end{itemize}
\end{itemize}

\begin{itemize}
\item  \textbf{CNA}
\begin{itemize}
\item  Endereço: Avenida Dr. Romeu Tortima, 553.
\item  Telefone: (19) 3289-4700.
\item  Fax: (19) 3289-4700.
\item  E-mail: baraogeraldo [@] cna [.] com [.] br
\item  Site: \url{http://www.cna.com.br/baraogeraldo}
\end{itemize}
\end{itemize}

\begin{itemize}
\item  \textbf{HAVAD}
\begin{itemize}
\item  Endereço: Avenida Romeu Tórtima, 522.
\item  Telefone: (19) 3288-0012.
\item  Fax: (19) 3249-0488.
\item  Email: havad [@] havad [.] com [.] br
\item  Site: \url{http://www.havad.com.br/}
\end{itemize}
\end{itemize}

\begin{itemize}
\item  \textbf{In Touch}
\begin{itemize}
\item  Endereço: Rua Antônio Augusto de Almeida, 517 -- Cidade Universitária.
\item  Telefone: (19) 3289-3481.
\item  E-mail: secretaria [@] intouch [.] art [.] br
\item  Site: \url{http://www.intouch.art.br/}
\end{itemize}
\end{itemize}

\begin{itemize}
\item  \textbf{INOVA -- Escola de Inglês}
\begin{itemize}
\item  Endereço: Avenida Romeu Tórtima, 391.
\item  Telefone: (19) 3288-0071.
\item  E-mail: fale [@] inovalinguas [.] com [.] br
\item  Site: \url{http://www.inovalinguas.com.br}
\end{itemize}
\end{itemize}

\begin{itemize}
\item  \textbf{Real Time}
\begin{itemize}
\item  Endereço: Rua Agostinho Páttaro, 47 (esquina com a do praça do coco).
\item  Telefone: (19) 3289-6240.
\item  E-mail: barao [@] rtidiomas [.] com [.] br
\item  Site: \url{http://www.realtimeenglish.com.br/}
\end{itemize}
\end{itemize}

\begin{itemize}
\item  \textbf{Wizard}
\begin{itemize}
\item  Endereço: Rua João Batista Antonioli, 90.
\item  Telefone: (19) 3289-6199.
\end{itemize}
\end{itemize}

\begin{itemize}
\item  \textbf{Yazigi}
\begin{itemize}
\item  Endereço: Avenida Dr. Romeu Tórtima, 500.
\item  Telefone: (19) 3249-2375.
\item  Site: \url{http://www.yazigi.com.br/}
\end{itemize}
\end{itemize}

\subsection{Igrejas}
\begin{itemize}
\item  \textbf{IBCU -- Igreja Batista da Cidade Universitária}
\begin{itemize}
\item  Endereço: Rua Tenente Alberto Mendes Jr., 5.
\item  Telefone: (19) 3289-4501.
\item  Site: \url{http://www.ibcu.org.br}
\item  E-mail: jovens [@] ibcu [.] org [.] br
\end{itemize}
\end{itemize}

\begin{itemize}
\item  \textbf{IBBG -- Igreja Batista de Barão Geraldo}
\begin{itemize}
\item  Endereço: Rua Luiz Vicentim, 284.
\item  Telefone: (19) 3289-1793.
\end{itemize}
\end{itemize}

\begin{itemize}
\item  \textbf{IPBG -- Igreja Presbiteriana de Barão Geraldo}
\begin{itemize}
\item  Endereço: Rua Francisco Andreo Aledo, 141 (próximo à praça do coco e moradia).
\item  Telefone: (19) 3289-3239.
\end{itemize}
\end{itemize}

\begin{itemize}
\item  \textbf{Igreja do Nazareno Betânia}
\begin{itemize}
\item  Endereço: Rua Manoel Antunes Novo, 98.
\item  Telefone: (19) 3289-7379.
\end{itemize}
\end{itemize}

\begin{itemize}
\item  \textbf{Assembleia de Deus -- Ministério de Belém}
\begin{itemize}
\item  Endereço: Rua Júlia Leite de Barros, 54 (próximo à moradia da Unicamp).
\item  Telefone: (19) 3249-0035.
\item  Site: \url{http://www.adbarao.com.br}
\end{itemize}
\end{itemize}

\begin{itemize}
\item  \textbf{Paróquia Santa Isabel}
\begin{itemize}
\item  Endereço: Rua Benedito Alves Aranha, 226.
\item  Telefones: (19) 3289-1101 e (19) 3289-2323
\item  Site: \url{http://www.paroquiasantaisabel.org.br}
\end{itemize}
\end{itemize}

\begin{itemize}
\item  \textbf{Comunidade do Estudante Universitário}
\begin{itemize}
\item  Endereço: Rua Dr. Ruberlei Boaretto da Silva, 785.
\item  E-mail: ceu [.] campinas [@] gmail [.] com
\end{itemize}
\end{itemize}

\subsection{Postos de combustível}

\begin{itemize}
\item  \textbf{Auto Posto Campineira}
\begin{itemize}
\item  Endereço: Avenida Albino Jose Barbosa de Oliveira, 1480.
\item  Telefone: (19) 3289-5991.
\end{itemize}
\end{itemize}

\begin{itemize}
\item  \textbf{Auto Posto Barbieri de Barão Geraldo}
\begin{itemize}
\item  Endereço: Avenida Albino Jose Barbosa de Oliveira, 1001 (próximo ao terminal).
\item  Telefone: (19) 3289-1917 e (19) 9768-6713.
\end{itemize}
\end{itemize}

\begin{itemize}
\item  \textbf{Centro Automotivo Cidade Universitária LTDA}
\begin{itemize}
\item  Endereço: Avenida Doutor Romeu Tortima, 1541.
\item  Telefones: (19) 3289-8457, (19) 3289-9934 e (19) 3289-9199.
\end{itemize}
\end{itemize}

\begin{itemize}
\item  \textbf{Transo Combustíveis LTDA}
\begin{itemize}
\item  Endereço: Avenida Santa Izabel, 1030 (próximo à moradia).
\item  Telefone: (19) 3289-1012.
\end{itemize}
\end{itemize}

\begin{itemize}
\item  \textbf{Esso Auto Posto Futuro}
\begin{itemize}
\item  Endereço: Avenida Albino Jose Barbosa Oliveira, 360 (logo na entrada do distrito de Barão).
\item  Telefone: (19) 3289-4332.
\end{itemize}
\end{itemize}

\begin{itemize}
\item  \textbf{Posto Vô João}
\begin{itemize}
\item  Endereço: Avenida Albino José Barbosa de Oliveira, 2151.
\item  Telefones: (19) 3289-3388 e (19) 3289-6594.
\item  Site: \url{http://www.vojoao.com/}
\end{itemize}
\end{itemize}

OBS: Dentro do campus há um posto de combustível (próximo à descida da FEAGRI),
mas atende apenas veículos oficiais.

\section{SIGLAS malucas}

Nessa sessão serão dadas algumas explicações sobre algumas siglas, códigos
e outros termos muito usados dentro da Unicamp. Aí vão eles:

\begin{itemize}
\item  \textbf{CCG (Comissão Central de Graduação):} Órgão colegiado da Unicamp, é encarregada da orientação, supervisão e revisão periódica do ensino na Universidade. Cabe recurso à CCG de quaisquer decisões das Unidades afetando o ensino.
\end{itemize}

\begin{itemize}
\item  \textbf{CCPG (Comissão Central de Pós-Graduação):} Órgão colegiado da Unicamp, é encarregada da orientação, supervisão e revisão periódica da pós-graduação na Universidade. Cabe recurso à CCPG de quaisquer decisões das Unidades afetando o ensino.
\end{itemize}

\begin{itemize}
\item  \textbf{Consu (Conselho Universitário):} O Conselho Universitário é o órgão máximo da Universidade, para você entender, seria como o Parlamento Inglês. Assim como o Parlamento manda mais que a rainha, o Consu manda mais que o reitor (embora o reitor faça parte dele e influencie fortemente suas decisões). Existe representação discente no Consu, eleita por um processo realizado pela Coordenadoria Geral da Universidade. Até 2003 esse processo era realizado pelos próprios estudantes e há uma longa briga para reavê-lo (tanto que em 2005 o DCE realizou eleições paralelas para o Consu e a CCG).
\end{itemize}

\begin{itemize}
\item  \textbf{Congregação:} É o órgão colegiado do Instituto ou Faculdade. Cabe recurso à Congregação da Unidade de Ensino de quaisquer decisões dos Departamentos e das Coordenações de Curso.
\end{itemize}

\begin{itemize}
\item  \textbf{Departamento:} É administrado por um professor-chefe e um Conselho Departamental, é a menor unidade administrativa, didática e científica da Universidade, sendo responsável pelo desenvolvimento dos programas de ensino, pesquisa e extensão dos serviços à comunidade. Todo instituto e faculdade da universidade possui o seu conjunto de departamentos, conhecidos através de siglas.
\end{itemize}

\begin{itemize}
\item  \textbf{CI (Conselho Interedepartamental):} Este é um "braço" da congregação, responsável por tratar de assuntos menores, como despesas e atribuições de sala. Fazem parte deste órgão, além de um representante discente, o diretor do instituto, os coordenadores e os chefes de departamentos.
\end{itemize}

\begin{itemize}
\item  \textbf{CDI (Comissão Diretora de Informática):} Outro braço da congregação, responsável por tratar de assuntos relacionados aos ambientes computacionais, deliberando sobre a atualização de infraestrutura, a organização da rede, endereços de internet e similares.
\end{itemize}

\begin{itemize}
\item  \textbf{CG (Coordenadoria/Comissão de Graduação):} É o órgão da unidade responsável pelos seus cursos de graduação. Sempre que houver algum problema ou deficiência no curso, é este órgão que vocês devem procurar. Cada curso tem um coordenador (que faz parte da CG), sendo que, atualmente, o professor Hélio Pedrini é o coordenador da Ciência e os professores Eduardo Xavier (IC) e o Akebo Yamakami (FEEC) são os coordenadores da Engenharia.
\end{itemize}

\begin{itemize}
\item  \textbf{CPG (Coordenadoria/Comissão de Pós-Graduação):} Este é o órgão responsável pela pós-graduação no Instituto, coordenando as disciplinas oferecidas e as matrículas na pós. O coordenador atual é o professor Paulo Lício de Geus. Os alunos tem direito a voz e voto nos colegiados (instâncias decisórias compostas por várias pessoas) da Unicamp, tendo representação discente em número correspondente a um quinto (1/5) dos membros. Geralmente, os representantes discentes são eleitos pelos estudantes ou indicados pelos centros acadêmicos. O exercício da representação estudantil e atividades decorrentes não exonera o aluno da freqüência nas atividades escolares, com exceção da participação em reuniões em órgãos colegiados, nos horários em que estes se reúnem para deliberar.
\end{itemize}

\begin{itemize}
\item  \textbf{DCE (Diretório Central de Estudantes):} É a entidade de representação dos estudantes de graduação da Unicamp, competindo-lhe ainda designar representantes estudantis para os órgãos colegiados da Universidade.
\end{itemize}

\begin{itemize}
\item  \textbf{DAC (Diretoria Acadêmica):} É o órgão executivo e informativo, incumbido do registro e controle das atividades discentes da Unicamp. Cuida das matrículas, alteração de matrícula, emissão de documentos e relatórios, como o histórico escolar, realiza reserva de salas, entre outras atividades.
\end{itemize}

\begin{itemize}
\item  \textbf{SAE (Serviço de Apoio ao Estudante):} É encarregado da execução de programas de assistência desenvolvidas pela Universidade, por iniciativa própria ou mediante convênios firmados com entidades especializadas.
\end{itemize}

\begin{itemize}
\item  \textbf{Crédito:} Unidade elementar de horas-aula de qualquer curso da Unicamp. Um crédito equivale a uma hora-aula semanal, ou a 15 horas-aula semestrais.
\end{itemize}

\begin{itemize}
\item  \textbf{Período letivo:} É um nome complicado para se referir ao semestre.
\end{itemize}

\begin{itemize}
\item  \textbf{Currículo pleno:} É o conjunto de disciplinas do curso que o aluno tem que cursar.
\end{itemize}

\begin{itemize}
\item  \textbf{CR (Coeficiente de Rendimento):} Valor entre 0 e 1 que é a média ponderada das notas obtidas em todas as disciplinas até o momento. É calculada usando como pesos o número de créditos de cada disciplina.
\end{itemize}

\begin{itemize}
\item  \textbf{CP (Coeficiente de Progressão):} É a porcentagem do curso que você já cumpriu. Por exemplo, se você tem CP = 0,6123 significa que você cumpriu 61,23\% do curso. Você se forma quando o seu CP for 1 (100\% do curso completo). É importante saber o CP quando for fazer algum estágio, ou um TCC (Trabalho de Conclusão de Curso), ou quando for cursar disciplinas que tenham como pré-requisito AA4xy.
\end{itemize}

\begin{itemize}
\item \begin{itemize}
\item  \textbf{CPF (Coeficiente de Progressão Futuro):} Além do CP, também tem o CPF, que além do nome de um documento é o CP que você terá no fim do semestre caso passe em todas as disciplinas.
\end{itemize}
\end{itemize}

\begin{itemize}
\item \begin{itemize}
\item  \textbf{CPE (Coeficiente de Progressão Exigido):} Além do CP e do CPF há o CPE. O CPE foi instituído a partir de 2005 e é usado para fins de cancelamento, ou não, de matrícula. Para que o aluno possa continuar a fazer o curso, ele precisa ter um CP maior ou igual ao CPE daquele semestre. Tanto o CP, como o CPE e o CPF existem somente nos cursos de graduação.
\end{itemize}
\end{itemize}

\begin{itemize}
\item  \textbf{Pré-requisito:} Matéria(s) que precisa(m) ter sido cursada(s) para que se possa fazer outra(s) matéria(s). Existem dois tipos de pré-requisitos: Os pré-requsitos totais, mais comuns, do qual é exigido tanto a aprovação por nota como por frequência e os pré-requisitos parciais, mais raros, do qual o aluno não precisa ter sido aprovado por nota, mas tem que ter tido aprovação por frequência e nota final maior ou igual a 3,0. Os pré-requisitos parciais são identificados com um asterisco na frente do código da disciplina (não confundir com um apontador).
\end{itemize}

\begin{itemize}
\item  \textbf{AA4xy:} Um tipo de pré-requisito, raríssimo. Não se trata de nenhuma disciplina. Para fazer disciplinas com esse pré-requisito, o aluno tem que tem um CP maior ou igual a 0,xy.
\end{itemize}

\begin{itemize}
\item  \textbf{AA200:} Outro tipo de pré-requisito existente, não tão raro, porém mais presente em disciplinas eletivas. Também não se trata de nenhuma disciplina. É apenas uma autorização da coordenadoria do curso. Se sobrar vagas para a disciplina e a coordenadoria do curso for com a sua cara você faz a disciplina.
\end{itemize}

\begin{itemize}
\item  \textbf{PB (Prédio Básico)} Também conhecido como Ciclo Básico II, é um prédio com várias salas de aula, que fica em frente ao Bandejão, e serve várias unidades que não possuem espaço físico suficiente para comportar seus alunos. No segundo andar ficam as salas de aula (PB01 a PB12) e no terceiro andar ficam os auditórios (PB13 a PB18).
\end{itemize}

\begin{itemize}
\item  \textbf{CB (Ciclo Básico I):} Tem finalidade idêntica ao PB, só que é muito melhor equipado, tem uma acústica muito melhor e tem um ar condicionado capaz de matar esquimó de frio. Fica na mesma praça que o PB, só que no outro extremo. À esquerda da entrada ficam as salas ímpares e à direita ficam as salas pares. No primeiro andar ficam os auditórios (CB01 a CB06) que possuem 140 e 160 lugares e no segundo andar ficam as salas de aula (CB07 a CB18) que possuem 60 e 80 lugares.  O CB e o PB são os lugares onde você vai ter a maioria das suas aulas (especialmente nos dois primeiros anos de curso).
\end{itemize}

\subsection{Siglas de salas de aula}

A tabela~\ref{tab:institutos} contém algumas siglas de salas de aula que aparecem nos cadernos
de horários, disponibilizados pelas coordenadorias dos cursos e pela DAC.

\begin{center}
\begin{table*}[ht!]
{
\begin{tabular}{|l|p{6cm}|p{8cm}|}\hline

\multicolumn{3}{|c|}{ \textbf{Siglas e locais das Salas de Aula no horário}}\tabularnewline \hline

 \textbf{Sigla}  &  \textbf{Local}  &  \textbf{Referência}\tabularnewline \hline

 CB  &  Ciclo Básico I  &  Praça Central, atrás do Banespa, em frente da cantina da Física.\tabularnewline \hline

 CC  &  Instituto de Computação (IC)  &  Ao lado do Departamento de Artes Cênicas (IC-1); ao lado do IE (IC-2) e atrás do IE (IC-3).\tabularnewline \hline

 CI  &  Centro de Estudo de Línguas (CEL)  &  Atrás do IFCH.\tabularnewline \hline

 CL  &  Instituto de Estudos da Linguagem (IEL)  &  Em frente a praça central e ao lado do IFCH.\tabularnewline \hline

 EB  &  Engenharia Básica  &  Atrás da Praça da Paz e próximo à FEEC.\tabularnewline \hline

 EM  &  Faculdade de Engenharia Mecânica (FEM)  &  Atrás do IQ.\tabularnewline \hline

 FA  &  Faculdade de Engenharia de Alimentos (FEA)  &  Em frente ao IQ e à Praça da Paz.\tabularnewline \hline

 FE  &  Faculdade de Engenharia Elétrica e de Computação (FEEC)  &  Em frente à Praça da Paz.\tabularnewline \hline

 IB  &  Instituto de Biologia (IB)  &  Entre o IQ e o Serviço Social do SAE.\tabularnewline \hline

 IE  &  Instituto de Economia (IE)  &  Atrás do IMECC.\tabularnewline \hline

 IF  &  Instituto de Física (IFGW)  &  Em frente ao Ciclo Básico, e a Química.\tabularnewline \hline

 IH  &  Instituto de Filosofia e Ciências Humanas (IFCH)  &  Entre o IMECC e o IEL.\tabularnewline \hline

 IM  &  Instituto de Matemática, Estatística e Computação Científica (IMECC)  &  Em frente a Praça Central.\tabularnewline \hline

 IQ  &  Instituto de Química (IQ)  &  Entre o IB e o IFGW.\tabularnewline \hline

 LE  &  Laboratórios de informática da FEEC  &  Em frente a Praça da Paz e ao lado das salas de aula da FEEC.\tabularnewline \hline

 LF  &  Laboratório de Física  &  Em frente à cantina do IMECC.\tabularnewline \hline

 LQ  &  Laboratório de Química  &  Em frente à biblioteca do IQ.\tabularnewline \hline

 PB  &  Ciclo Básico II, Prédio Básico  &  Praça Central, em frente ao Bandejão.\tabularnewline \hline

\end{tabular}
}
\hfill{}
\caption{Siglas das salas de aula}
\label{tab:institutos}
\end{table*}
\end{center}

\section{Cancelamento, trancamento e desistência. Há alguma diferença?}

Existe várias coisas diferentes que o aluno pode fazer com a sua matrícula.
Aliás, o próprio processo de matrícula (em que o aluno pode escolher, e alterar
(\url{http://www.dac.unicamp.br/portal/grad/regimento/capitulo_iii/secao_iv/index.html}),
as disciplinas que ele quer fazer durante o semestre) já é algo diferente. Essas
coisas são a desistência, cancelamento e trancamento de matrícula.

Embora praticamente todos os alunos da Unicamp usem esses três termos
"indiscriminadamente", como se fossem sinônimos, para a DAC, esses três termos
tem significados bastante distintos. Aí vai o que cada termo significa:

\begin{itemize}
\item Desistência de matrícula em disciplinas
      (\url{http://www.dac.unicamp.br/portal/grad/regimento/capitulo_iii/secao_v/index.html}):
      Processo que é chamado pelos alunos de "trancamento", ou de "trancamento de
      disciplina", termo aliás que não existe. Consiste de desistir de uma
      determinada disciplina. O aluno não mais cursa essa disciplina no semestre,
      tendo de cursá-la em algum semestre posterior.  Só pode desistir uma vez da
      disciplina e pode-se pedir desistência até que se tenha passado 1/2 do
      semestre.
\item Cancelamento de matrícula
      (\url{http://www.dac.unicamp.br/portal/grad/regimento/capitulo_iii/secao_vii/index.html}):
      Processo em que o aluno se desliga da Unicamp, por motivo de jubilação, por
      ter faltado às duas primeiras semanas do ano de ingresso, por ter sido
      reprovado em todas as disciplinas do primeiro ou do segundo semstre de
      ingresso, por ter sido expulso, por ter sido aprovado em outra universidade
      pública (não é permitido fazer mais do que um curso de universidade pública
      simultaneamente), ou por vontade própria do aluno.
\item Trancamento de matrícula
      (\url{http://www.dac.unicamp.br/portal/grad/regimento/capitulo_iii/secao_vi/index.html}):
      Processo em que o aluno não cursa qualquer disciplina da Unicamp durante
      o semestre. Quando, simplesmente, se fala "trancamento", refere-se a esse
      processo. O aluno tem direito a fazer até dois trancamentos de matrícula, em
      semestres seguidos ou não e o aluno não pode trancar nenhum dos dois
      semestres do ano de ingresso. Desistência de todas as disciplinas
      configura-se como trancamento. O trancamento é pedido na DAC, e pode ser
      pedido até que se tenha transcorrido 2/3 do semestre (geralmente de dezembro
      até fim de maio para trancamento de primeiro semestre; e de julho até fim de
      outubro para trancamento de segundo semestre). Para cada trancamento,
      o prazo máximo de integralização é postergado.
\end{itemize}

\subsection{GDE}

Bixo,você já entrou na universidade e agora vai começar a parte mais difícil:
sair da faculdade com um diploma na mão.  A Unicamp é muito diferente da sua
escolinha onde a tia Gertrudes entregava o seu horário impresso coloridinho pra
você colar na capa do seu fichário. Na Unicamp você vai ter que se virar, e vai
descobrir que pra montar o seu horário você precisa perder horas folheando
manuais de aluno, recomendações de horário, descobrindo matérias que tem
dependências e equivalentes... ou você pode usar o GDE.  o GDE
(\url{http://www.gde.ir}) é uma ferramenta que simplifica (MUITO) o seu
planejamento dentro da Unicamp, te dando de bandeja tudo que você vai precisar
pra enfrentar o dia-a-dia na Unicamp.  Ah, tudo isso dentro de uma rede social,
com chat e tudo, onde você pode ver o seu horário, o de seus amigos, o cardápio
do bandejão, avaliações sobre seus professores, calendário... tá fácil hein?  Ou
você pode começar por aqui:
\url{http://www.dac.unicamp.br/portal/grad/regimento/regimento_completo/}. Você
que sabe.

\section{Lugares para estudar}

Há vários lugares para estudar na Unicamp. Você pode escolher o que você achar
melhor:

\begin{itemize}
\item  \textbf{Biblioteca Central (BC):} A BC tem três andares. O primeiro é onde tem os livros gerais e onde a galera estuda. Geralmente é barulhento em épocas de provas, mas é bom porque sempre tem lugar para estudar e fecha às 22h. Se você não se importa com barulho, ou até acha que você faz bastante, esse é o lugar da BC para você estudar. O segundo andar é onde está a BAE, a Biblioteca da Área de Engenharia. Um pouco mais silenciosa que a BC nas mesas externas, esse andar tem salas para estudo em grupo, bastante silenciosas, mas que sempre estão ocupadas em época de provas, e mesas individuais escondidas entre os periódicos. O terceiro andar é para silence freaks. Morbidamente silencioso, desértico (muita gente desconhece a existência desse andar), esse é o lugar mais silencioso da BC para estudar. Tem umas salinhas de estudo individual e duas mesas para estudo em grupo. O problema é que fecha às 17h, mas o pôr-do-sol de lá de cima também é ma-ra-vi-lho-so.
\end{itemize}

\begin{itemize}
\item  \textbf{Arcádia (ou mesinhas do IEL):} A Arcádia é algumas mesas ao ar livre no IEL (Instituto de Estudos da Linguagem). Em horários de aula é silencioso, é um ambiente muito agradável e por ser ao ar livre, não fecha. Tem dois problemas: O grande fluxo de mulheres no local pode facilmente distraí-lo, principalmente se você as conhecer, e à noite enche de insetos (além da iluminação não ser das melhores). Às vezes, venta bastante e é ruim para estudar com folhas avulsas. Mas ainda assim é um ótimo local para estudar.
\end{itemize}

\begin{itemize}
\item  \textbf{Biblioteca do IFGW:} A biblioteca do IFGW (Instituto de Física Gleb Wataghin) é ótima para dias de calor, por ser super gelada (ar-condicionado mega-super-power!). Tem vantagem sobre as outras bibliotecas pelo fato das salas de estudo serem fora da biblioteca e por isso você não precisa deixar o seu material para entrar na sala de estudos. O problema é que tem poucos lugares e só duas mesas para estudo em grupo. O resto são baias individuais.
\end{itemize}

\begin{itemize}
\item  \textbf{BIMECC:} A biblioteca do IMECC tem poucos lugares, poucas mesas para estudo individual, os locais de estudo ficam dentro da biblioteca (você precisa guardar sua bolsa para entrar), não é muito gelada e o ambiente não é agradável, mas nela e na BAE é que você encontrará a maioria dos livros relacionados a computação.
\end{itemize}

\begin{itemize}
\item  \textbf{Outras bibliotecas:} Aventure-se por outras bibliotecas, como a da Economia, a da Pedago e a da Biologia e as conheça. Para aqueles que gostam (ou são obrigados) a estudar aos fins de semana a BC e as bibliotecas da Educação, da Economia, da Química, da Medicina, do IEL e da Geociências abrem aos sábados. Para saber os horários de funcionamento das bibliotecas, entre no site do SBU (\url{http://www.sbu.unicamp.br/index.php?link=30}).
\end{itemize}

\begin{itemize}
\item  \textbf{Bitolódromos:} Existem dois bitolódromos na Unicamp: O do IC e o da FEEC (coincidência interessante, né?). O do IC-3 é uma mesa grande com algumas cadeiras no antigo saguão de entrada. O da FEEC fica no fundo do prédio principal (qualquer veterano sabe onde é o bitolódromo, não tenha vergonha de perguntar). O da FEEC é maior, você se distrai menos porque não estão todos os seus colegas (mas várias outras pessoas estão) entrando e saindo de lá (embora os pica-fios sejam bastante barulhentos), e sempre você encontra gente que possa te ajudar. O do IC serve para quando você já estiver lá e com preguiça de ir à Elétrica, porque o da FEEC é muito melhor.
\end{itemize}

\begin{itemize}
\item  \textbf{Sala 316:} Outro alento para as madrugadas de estudo é a sala 316 do IC-3, que fica aberta sempre, ou então, aberta facilmente com a chave em posse do guardinha. É uma sala com carteiras legais, lousa e ar condicionado, aliás, é muito boa para estudo em grupo (NABVS IMINENTVS) por causa da lousa.
\end{itemize}

\begin{itemize}
\item  \textbf{Sua casa:} Se você mora em uma república com pessoas da sua turma, vá fundo e estude em casa. Se você mora sozinho ou com caras de outros cursos/anos, mas se concentra bem em casa, também o faça. Caso contrário, estude na Unicamp. É muito fácil se distrair em casa. Você vai à geladeira, mexe no computador, lê outra coisa, deita na cama e dorme, entre outras coisas. Prefira estudar na Unicamp. Outra coisa, não seja egoísta, quando tiver oportunidade de estudar em grupo, prefira essa alternativa. Lembre-se que você não está mais no "cursinho", tente sempre pegar as dicas que a galera te dá, principalmente dos seus veteranos.
\end{itemize}

\section{Melhores banheiros}

Uma das maiores necessidades do ser humano pode ser potencializada se for
realizada num banheiro decente. Portanto, é muito importante que você saiba onde
ir. Alguns dos melhores banheiros da Unicamp são:

\begin{itemize}
\item  \textbf{IC-3:} Geralmente estão limpos e utilizáveis. Mas fedem! Sempre com papel higiênico, é uma boa pedida na hora do apuro. Exceto nos finais de semana.
\end{itemize}

\begin{itemize}
\item  \textbf{IC-2:} Quase sempre estão limpos e utilizáveis e tem um odor melhor que os do IC-3. Só precisa tomar cuidado pois, as vezes, falta papel higiênico.
\end{itemize}

\begin{itemize}
\item  \textbf{FEEC:} Possui excelentes banheiros escondidos por lá. Procure bem!
\end{itemize}

\begin{itemize}
\item  \textbf{PB:} Os banheiros do segundo e do terceiro andar do Pavilhão Básico também são bons (especialmente os do terceiro andar, por quase não serem usados). Só tome cuidado, porque às vezes não tem papel higiênico.
\end{itemize}

\begin{itemize}
\item  \textbf{FE:} A faculdade de educação tem poucos banheiros masculinos, mas estão entre os melhores da Unicamp pelo pouco uso.
\end{itemize}

\begin{itemize}
\item  \textbf{CB:} Estes banheiros ficam escondidos próximo às escadas do CB (no térreo). Se você tiver sorte de chegar bem após a limpeza, o banheiro estará em excelentes condições. Porém, na maior parte do tempo ele fica bem sujinho.
\end{itemize}

\begin{itemize}
\item  \textbf{DEQ:} Departamento de Eletrônica Quântica, no IFGW. Dizem que ninguém os usa.
\end{itemize}

\begin{itemize}
\item  \textbf{DRCC:} Departamento de Raios Cósmicos e Cronologia, no IFGW. Um dos melhores banheiros existentes na Unicamp (senão o melhor). Assim como os banheiros do DEQ, dizem que ninguém os usa.
\end{itemize}

\begin{itemize}
\item  \textbf{DFA:} Departamento de Física Aplicada, no IFGW. Os dois andares do departamento tem banheiros bons e utilizáveis, mas algumas vezes falta papel higiênico.
\end{itemize}

\begin{itemize}
\item  \textbf{IMECC:} Todos os três departamentos (andares) do IMECC tem banheiros bons e utilizáveis. Mas vez ou outra falta papel higiênico.
\end{itemize}
