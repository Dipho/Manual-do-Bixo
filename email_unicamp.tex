% Este arquivo tex vai ser incluído no arquivo tex principal, não pe preciso
% declarar nenhum cabeçalho

\section{Agora eu tenho um email da Unicamp}

Ao ingressar num curso de Computação, você recebe 2 contas de e-mail: Do IC e da
DAC. Estes são os principais meios de comunicação da universidade com você,
portanto fique esperto e não deixe de ler regularmente!

Depois de muita luta, o IC foi um dos últimos institutos a ter um webmail
(acredite se quiser!), para que você possa acessar sua conta de qualquer lugar.
O endereço é \url{http://webmail.students.ic.unicamp.br/} e lá você precisa do
seu login (que é o mesmo login do IC, raXXXXXX) e uma senha, que será a mesma
senha do sistema Linux do IC (que vocês receberão do IC em uma das primeiras
semanas de aula). Caso tenha dúvidas, de uma passada na Secretaria de Graduação,
que fica no IC-2 (prédio ao lado das Artes Cênicas e para cima da Economia)
e pergunte!

Uma dica interessante para a utilizaçao do email do IC, que, muitas vezes, 
as pessoas não sabem, é que quando você recebe a sua senha, vem indicado
no papel um {\it alias} para o seu email. Assim, você poderá utlizá-lo de
uma forma mais amigável, ao invés de ser somente o seu próprio RA, por 
exemplo:
\begin{verbatim}RA129873@students.ic.unicamp.br \end{verbatim} por 
\begin{verbatim}fransisco.silva@students.ic.unicamp.br \end{verbatim}

A DAC também dispõe um webmail para o aluno. O login é a primeira letra do nome
do aluno, seguido dos dígitos do RA. O email da DAC também é útil. É nele que os
alunos são avisados sobre o período de matrícula, recebem os seus pedidos de
matrículas provisórias, avisos de desistências e trancamentos, além de informar
os alunos a respeito de eventos que acontecem na Unicamp, como feiras,
palestras, festivais, eleições. O melhor a se fazer com essa conta
é direcioná-la para uma outra conta de email da Unicamp (tem de ser da Unicamp),
pois você nunca vai lembrar de acessá-lo (instruções para redirecionamento de
emails são dadas logo abaixo). Na própria página da DAC
(\url{http://www.dac.unicamp.br/}) você diretamente cria a sua senha e já pode
acessar o webmail dali mesmo.

Para a galera da Engenharia, tem também o e-mail da FEEC, que muitos nem ficam
sabendo que existe! Só um semestre ou até um ano depois vão até o SIFEEC retirar
seu login e senha! Ou então ficam sabendo só no segundo ou terceiro ano de curso
e aí acumula mais de 700 e-mails não lidos. Assim como o do IC, é muito
utilizado para divulgação de eventos, oportunidades de estágios e de iniciação
científica. Para utilizá-lo, você deve ir até a FEEC e procurar o SIFEEC, que
é o local responsável por isso. Fica no segundo andar do prédio de laboratórios
da FEEC (um com escadas amarelas). O endereço do webmail da FEEC
é \url{http://www.fee.unicamp.br/webmail}.

Você também pode redirecionar os e-mails que receber nas contas do IC, FEEC
e DAC para qualquer outra conta (no Gmail por exemplo). Para isso, siga as dicas
abaixo:

\subsection{Redirecionamento de email da DAC}

Para efetuar o redirecionamento do e-mail institucional para outro e-mail, siga
os passos abaixo:

\begin{itemize}
\item  Acesse Serviços On-line;
\item  Alunos;
\item  Acesso aos Serviços Acadêmicos;
\item  Informe seu RA e sua senha atual (Observação: Não há necessidade de preencher o 3º campo, exceto se sua senha expirou);
\item  Clique no botão confirmar;
\item  Redirecionamento de e-mail.
\end{itemize}

\subsection{Redirecionamento de email da FEEC}

\begin{itemize}
\item  Logue no Linux
\item  Abra um terminal
\item  Certifique-se que está na home
\item  Digite o comando: echo 'seu\_email@gmail.com' > .forward (logicamente substituindo 'seu\_email' pelo seu e-mail).
\end{itemize}

\subsection{Redirecionamento de email do IC}

\begin{itemize}
\item  Acesse \url{https://webmail.students.ic.unicamp.br/}
\item  Faça o login
\item  Filtros
\item  Adicionar nova regra
\item  Na primeira caixa, selecione ALL
\item  No menu abaixo, clique em "Redirecionar para o seguinte endereço de e-mail"
\item  Defina o endereço pra onde você quer redirecionar
\item  Clique em "Adicionar nova regra"
\end{itemize}

Qualquer dúvida, novamente, procure um veterano! Esta é sua melhor opção!

Finalizando, não deixe de estar sempre informado sobre os acontecimentos ou
divulgações da Unicamp, da AAACEC, do CACo e da CONPEC, essas três entidades
compostas por alunos de Computação.
