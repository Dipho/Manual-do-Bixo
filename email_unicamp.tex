% Este arquivo .tex será incluído no arquivo .tex principal. Não é preciso
% declarar nenhum cabeçalho

\section{Agora eu tenho um e-mail da Unicamp}

Ao ingressar num curso de computação, você recebe pelo menos duas contas de e-mail: do IC e da
DAC. Estes são os principais meios de comunicação da Universidade com você,
portanto fique esperto e não deixe de ler regularmente!

Para acessar o webmail do IC, o endereço é \url{webmail.students.ic.unicamp.br}.
O usuário e a senha são os mesmos do sistema Linux do IC, que você receberá nas
primeiras semanas de aula. Caso tenha dúvidas, dê uma passada na Secretaria de Graduação,
que fica no IC-2 (prédio ao lado das Artes Cênicas e para cima da Economia)
e pergunte!

Uma dica interessante, que muitas vezes passa
despercebida, é que no mesmo documento em que você recebe sua senha, vem indicado
um {\it alias} para o seu e-mail. Assim, você poderá utlizá-lo de
uma forma mais amigável, ao invés de ser somente o seu próprio RA, por 
exemplo:
\begin{verbatim}RA129873@students.ic.unicamp.br \end{verbatim} por 
\begin{verbatim}fransisco.silva@students.ic.unicamp.br \end{verbatim}

A DAC também dispõe um webmail para o aluno. O login é a primeira letra do nome
do aluno, seguido dos dígitos do RA. O e-mail da DAC também é útil. É nele que os
alunos são avisados sobre o período de matrícula, recebem os seus pedidos de
matrículas provisórias, avisos de desistências e trancamentos, além de informar
os alunos a respeito de eventos que acontecem na Unicamp, como feiras,
palestras, festivais, eleições. Na própria página da DAC
(\url{www.dac.unicamp.br}) você diretamente cria a sua senha e já pode
acessar o webmail dali mesmo.

Para a galera da Engenharia, ainda há o e-mail da FEEC, que muitos sequer ficam
sabendo que existe! Só um semestre ou até um ano depois vão até o SIFEEC retirar
seu login e senha. Ou então ficam sabendo só no segundo ou terceiro ano de curso
e aí há mais de 700 e-mails não lidos. Assim como o do IC, é muito
utilizado para divulgação de eventos, oportunidades de estágios e de iniciação
científica. Para utilizá-lo, você deve ir até a FEEC e procurar o SIFEEC, que
é o local responsável por isso. Fica no segundo andar do prédio de laboratórios
da FEEC (um com escadas amarelas). O endereço do webmail da FEEC
é \url{webmail.fee.unicamp.br} e a senha é a mesma do Linux da FEEC.

Você também pode redirecionar os e-mails que receber nas contas do IC, FEEC
e DAC para qualquer outra conta (no Gmail por exemplo). Para isso, siga as dicas
abaixo:

\subsection{Redirecionamento de e-mail da DAC}

Para efetuar o redirecionamento do e-mail institucional para outro e-mail, siga
os passos abaixo:

\begin{itemize}
\item  Acesse \url{www.dac.unicamp.br}
\item  Acesse Serviços On-line
\item  Acesse Alunos
\item  Clique em Acesso aos Serviços Acadêmicos
\item  Faça login
\item  Clique em Redirecionamento de Email
\end{itemize}

\subsection{Redirecionamento de e-mail da FEEC}

\begin{itemize}
\item  Acesse \url{webmail.fee.unicamp.br}
\item  Faça login
\item  Acesse Options
\item  Acesse Mail Forwarding
\end{itemize}

\subsection{Redirecionamento de e-mail do IC}

\begin{itemize}
\item  Acesse \url{webmail.students.ic.unicamp.br}
\item  Faça login
\item  Acesse Filtros
\item  Clique em Adicionar uma nova regra
\item  Em Condição, na primeira lista drop-down com a opção Header, escolha a opção All
\item  Em Ação, escolha Redirecionar para o seguinte endereço de e-mail
\end{itemize}

Qualquer dúvida, novamente, procure um veterano!

Finalizando, não deixe de estar sempre informado sobre os acontecimentos ou
divulgações da Unicamp, do CACo, da AAACEC e da Conpec, essas três entidades
compostas por alunos de computação.
