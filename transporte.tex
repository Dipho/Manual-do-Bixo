% Este arquivo .tex será incluído no arquivo .tex principal. Não é preciso
% declarar nenhum cabeçalho

\section{Transporte}
\subsection{Voltar para casa}

Para os calouros de fora de Campinas, além de escolher a nova morada
é importante recolher informações sobre como realizar o trajeto entre sua cidade
e Campinas.

A forma usual é ir de ônibus, mas tenha em mente que a rodoviária é longe e os
trajetos de ônibus até lá são demorados. O endereço da rodoviária é Rua Barão de
Parnaíba, 690. Há duas linhas que passam por lá: o 3.32 (que passa dentro da
Unicamp) para dentro da rodoviária, mas demora mais pra chegar que o 3.31, que
sai do terminal e para do lado de fora da rodoviária. Em horários de pico,
o trajeto pode demorar quase uma hora, então cuidado para não perder o horário
do ônibus para sua cidade.

Os que vêm de mais longe certamente farão uso do aeroporto de Viracopos, cujo
telefone é 3725-5000. Para chegar ao aeroporto existe a linha 193 que sai da
rodoviária e vai para o aeroporto, e também faz o trajeto de volta. Porém ir com
ônibus circular pode ser um transtorno quando estiver com mala grande. Uma outra
alternativa é a Caprioli que também faz o trajeto da rodoviária para
o aeroporto. A passagem da Caprioli custa em torno de R\$8,00 e os horários
podem ser conferidos no site \url{www.caprioli.com.br}.

Para quem for usar os aeroportos da Grande São Paulo (Congonhas e Guarulhos),
também existe o translado da Caprioli. A tarifa sai em torno de R\$30,00
a R\$35,00 e os horários podem ser conferidos no site da Caprioli.

\subsubsection{Caronas}

Uma forma barata e divertida de viajar, além de minimizar o tempo e dinheiro
gastos na viagem, é juntando alguns estudantes no mesmo carro e dividir as
despesas.

O site UniCaronas (antigo Caronas Unicamp -- \url{www.unicaronas.com.br})
foi desenvolvido por dois engenheiros de computação da Unicamp, da turma 2004
(Guilherme Souza e Matheus Marosti), com o intuito de facilitar o deslocamento
dos alunos entre cidades. Em quatro anos de atividade, reuniu mais de 10000
usuários, ajudando a reduzir os custos de viagens e contribuindo para o aumento
do círculo de amizades dos participantes.

Atualmente o site possui carona para diversas cidades de vários estados, mas
novas cidades podem ser inseridas à medida que aumentar a demanda por elas.

Para garantir a segurança dos usuários o site exige no cadastro que
o interessado possua um e-mail da seguintes instituições:

\begin{itemize}
\item  ESPM.
\item  FAMEMA.
\item  FGV.
\item  ITA.
\item  Mackenzie.
\item  PUCCAMP.
\item  UEL.
\item  UEM.
\item  UFABC.
\item  UFBA.
\item  UFES.
\item  UFJF.
\item  UFLA.
\item  UFMG.
\item  UFPR.
\item  UFRGS.
\item  UFSCAR.
\item  UFSJ.
\item  UFTM.
\item  UFU.
\item  UFV.
\item  UNB.
\item  UNESP.
\item  Unicamp.
\item  UNIFAL.
\item  UNIFEI.
\item  UNIFESP.
\item  USP.
\end{itemize}

Ou então seja convidado por alguém que já usa o site. Assim como as cidades,
novas instituições são adicionadas à medida que aumentar a demanda.

Existe também um sistema de comentários, para que se possa alertar os demais
usuários sobre uma carona anterior: Se o motorista corria demais, se o motorista
ou caronista chegou ao local combinado no horário, se o carro era confortável ou
parecia uma lotação e outros indicativos relevantes recomendando ou não
a carona. No entanto, o site serve apenas para colocar os motoristas
e caronistas em contato, não assumindo responsabilidades sobre nenhuma das
partes.

Além disso o site também mantem uma página
(\url{unicaronas.uservoice.com/forums/36297-geral}) aonde os usuários
podem fazer sugestões.

\subsection{Carro}

Para aqueles que tenham seu próprio veículo, é bom saber que a Unicamp tem
poucas vagas próximas aos locais de aulas. E o número de fiscais de trânsito tem
aumentado muito dentro do Campus.

\subsubsection{Autoescolas}

Se você ainda não tem CNH e pretende obtê-la em Campinas, existem duas opções de
autoescola em Barão Geraldo:

\begin{itemize}
\item  \textbf{Auto Escola Avenida.}
\begin{itemize}
\item  Endereço: Avenida Albino José Barbosa De Oliveira, 658.
\item  Telefone: (19) 3288-0588.
\end{itemize}
\end{itemize}

\begin{itemize}
\item  \textbf{Auto Escola Advanced.}
\begin{itemize}
\item  Endereço: Avenida Santa Isabel, 80.
\item  Telefone: (19) 3289-3022.
\end{itemize}
\end{itemize}

\subsubsection{Recorrer de Multas}

Documentos:
\begin{itemize}
\item  Notificação e Fotocopia.
\item  CNH e Fotocopia.
\item  Documento do Carro e Fotocopia.
\item  Formulário: \url{www.emdec.com.brmultas/formulario_unificado.pdf}.
\item  e anexos, se desejar.
\end{itemize}

E levá-los a:
\begin{itemize}
\item  EMDEC.
\begin{itemize}
\item  Horário: De segunda a sexta-feira, das 8h as 17h.
\item  Endereço: Rua Dr. Salles Oliveira, 1028 -- Vila Industrial -- CEP 13035-270.
\end{itemize}
\item  Poupatempo (Centro).
\begin{itemize}
\item  Horário: Das 8h às 18h, de segunda a sexta-feira; e aos sábados, das 7h às 13h.
\item  Endereço: Avenida Francisco Glicério, 935.
\end{itemize}
\item  Poupatempo (Campinas Shopping)
\begin{itemize}
\item  Horário: Das 9h às 19h, de segunda a sexta-feira; e aos sábados, das 8h às 14h.
\item  Endereço: Rua Jacy Teixeira de Camargo, 940.
\end{itemize}
\end{itemize}

Fonte: \url{www.emdec.com.brmultas/multas_recursos_info_gerais.php}.

\subsection{Ônibus}

Se não tem condução própria, ou carona, pode utilizar o transporte coletivo de
Campinas. Os ônibus em Campinas são identificados por um número de três dígitos,
uma cor (azul claro, azul escuro, vermelho ou verde), uma figura geométrica
(círculo, quadrado, triângulo ou tetragrama, para que os portadores de
daltonismo possam identificar os ônibus) e um nome. A tarifa (uma das mais caras
do Brasil) foi reajustada no começo de 2011 e é R\$ 2,85. Apesar de existir
passe de estudante, universitários não tem direito ao desconto, apenas
estudantes de ensino fundamental e médio.

Existe quatorze linhas de ônibus que ligam o centro, alguns distritos, bairros
e terminais de Campinas ao distrito de Barão Geraldo. Algumas linhas desembarcam
os passageiros no terminal de Barão Geraldo, de onde seguem em outros ônibus
para a universidade. Já outras linhas vão direto para o campus. As trocas de
ônibus dentro do Terminal de Barão Geraldo são gratuitas, o que não ocorre em
outros terminais (no Terminal Mercado, por exemplo).

Em 2008 foi construída a nova e moderna rodoviária de Campinas, o terminal
multimodal Ramos de Azevedo. Além de transporte interestadual e municipal,
o espaço agrega um terminal metropolitano com o intuito de ligar as cidades da
Região Metropilitana de Campinas. Algumas linhas de ônibus internos de Campinas
também passam neste Terminal, como a linha 3.32.

Veja as linhas de ônibus disponíveis abaixo:

\begin{itemize}
\item  \textbf{1.34 -- Terminal Barão Geraldo (Inclusivo):} Sai do terminal do Ouro Verde, passa próximo ao Shopping Unimart, às avenidas Luís Smânio, Brasil, Theodureto de Almeida Camargo, Albino José Barbosa de Oliveira e vai para o terminal de Barão.
\end{itemize}

\begin{itemize}
\item  \textbf{2.10 -- Terminal Campo Grande / Shop. Dom Pedro / Terminal Barão Geraldo:} Sai do Terminal Campo Grande, passa pelo Shopping Dom Pedro, Terminal Barão Geraldo, avenida 1, rua Roxo Moreira (em frente à Reitoria), de novo pelo Shopping Dom Pedro, voltando para o Terminal Campo Grande. Saindo do terminal de Barão, essa linha passa na rua Roxo Moreira e próximo à area de biológicas e da saúde (Hospital das Clínicas e Faculdade de Ciências Médicas). Passa apenas na região da reitoria e hospital.
\end{itemize}

\begin{itemize}
\item  \textbf{2.66 -- Terminal Padre Anchieta / Hospital das Clínicas:} Sai do Terminal Padre Anchieta, passa pelo Makro, Shopping e rodovia D. Pedro, em frente ao Terminal Barão Geraldo (mas não entra), avenida 2, Unicamp, PUC, Rodovia D. Pedro I voltando para o Terminal Padre Anchieta.
\end{itemize}

\begin{itemize}
\item  \textbf{2.69 -- Terminal Padre Anchieta / Terminal Barão Geraldo:} Assim como o 2.66, sai do Terminal Padre Anchieta, mas faz um trajeto diferente do 2.66, indo até o terminal de Barão. O intervalo entre os ônibus é de 90 minutos em dias úteis, e de 100 minutos nos sábados, domingos e feriados.
\end{itemize}

\begin{itemize}
\item  \textbf{3.00 -- Sousas / Terminal Barão Geraldo:} Sai de Sousas (um dos distritos de Campinas, assim como Barão Geraldo), passa pelo Shopping Galleria, pelo terminal do Shopping Dom Pedro, rodovias Heitor Penteado e Dom Pedro, tapetão e vai em direção ao terminal de Barão Geraldo. O problema é que o intervalo entre os ônibus é de 45 minutos.
\end{itemize}

\begin{itemize}
\item  \textbf{3.21 -- Centro Médico/Bosque das Palmeiras:} Sai do Terminal Barão e leva à Cidade Universitária II, passando pela avenida 2 e em frente ao Centro Médico.
\end{itemize}

\begin{itemize}
\item  \textbf{3.28 -- Guará:} Assim como o 3.21 leva à Cidade Universitária II, mas vai pela estrada da Rhodia até a Cidade Universitária e segue até o Guará.
\end{itemize}

\begin{itemize}
\item  \textbf{3.29 -- Terminal Barão Geraldo / Cidade Judiciária:} Sai do terminal de Barão, passa pela avenida 2, pela Unicamp, por algumas ruas e avenidas do bairro fazenda Santa Cândida e vai até a estação da Cidade Judiciária. O intervalo entre ônibus pode variar de 27 a 40 minutos nos dias úteis, e é de 40 minutos aos sábados, domingos e feriados.
\end{itemize}

\begin{itemize}
\item  \textbf{3.30 -- Unicamp / Hospital das Clínicas:} Do Terminal Central de Campinas à Unicamp, passando pela rótula e avenidas Moraes Salles, Orosimbo Maia, Anchieta (prefeitura), Brasil e tapetão. Funciona das 6 às 19h30, a cada 15 minutos em média (depende do horário), de segunda a sexta-feira (não funciona nos sábados, domingos e feriados). Para ir do centro até a Unicamp e da Unicamp até o centro, essa linha é mais rápida que o 3.32, só que quase sempre os ônibus dessa linha estão cheios.
\end{itemize}

\begin{itemize}
\item  \textbf{3.31 -- Terminal Barão Geraldo / Rodoviária:} Sai do terminal Barão Geraldo e passa na rodoviária. É a opção mais rápida saindo de barão, levando cerca de 30 min (sem trânsito) em comparação com os mais de 40 minutos do 3.32. Passa também pelo Cambui.
\end{itemize}

\begin{itemize}
\item  \textbf{3.32 -- Terminal Barão Geraldo / Hospital das Clínicas / Rodoviária (Inclusivo):} Do Terminal Metropolitano até a Unicamp e depois para o terminal Barão Geraldo, passando pela rótula, pela rodoviária nova e pelas avenidas Campos Sales, Anchieta (prefeitura), Orozimbo Maia, Brasil, Theodureto de Almeida Camargo e outras. Funciona das 6 às 23 horas, a cada 15 minutos em média, todos os dias. Essa linha vai diretamente para a Unicamp, porém demora muito para ir do terminal metropolitano/centro até a Unicamp e da Unicamp até o centro/terminal metropolitano e quase sempre os ônibus dessa linha estão muito cheios, apesar de que, ao chegar ou sair da Unicamp eles estão com poucos passageiros. No sentido terminal de Barão Geraldo -- terminal metropolitano, esse ônibus para no ponto localizado ao lado do IC-1.
\end{itemize}

\begin{itemize}
\item  \textbf{3.33 -- Terminal Barão Geraldo / Circular Rótula:} Do Terminal Barão Geraldo ao centro de Campinas, passando pela rótula, Orosimbo Maia, Anchieta (prefeitura) e pelo Terminal Central. Não passa pela Unicamp, então para chegar à Unicamp, deve pegar a linha 3.29, 3.32 ou 3.87. Funciona das 5h30 às 23h30, a cada 10 minutos, todos os dias.
\end{itemize}

\begin{itemize}
\item  \textbf{3.37 -- Hospital das Clínicas:} Do Terminal Barão Geraldo ao HC, passando pela Unicamp. Funciona das 5h30 às 23h30, a cada 15 minutos, de segunda a sexta-feira. Passa pelo IC aos domingos.
\end{itemize}

\begin{itemize}
\item  \textbf{3.38 -- Terminal Barão Geraldo / Shopping D. Pedro / Shopping Iguatemi:} Linha que passa pelos dois principais shoppings da cidade. Funciona das 5:50 às 23:15, de segunda à sábado, a cada 30 minutos e em domingos e feriados funciona das 9:00 às 21:40, a cada 40 minutos.
\end{itemize}

E se pintar qualquer dúvida é só entrar no site da EMDEC
(\url{www.emdec.com.br}) ou da TRANSURC
(\url{www.transurc.com.br}) para ver os horários e a trajetória de todas
as linhas de Campinas.

\subsubsection{Bilhete Único}

O bilhete único foi implantado com o objetivo de facilitar o transporte daqueles
que se utilizam de ônibus. No prazo de 1 hora e meia (de segunda a sábado) e de
2 horas (nos domingos e feriados), o usuário pode utilizar até três ônibus
pagando apenas uma passagem. Para quem usa mais de três ônibus e quer aumentar
o número de integrações, tem de ir à sede da TRANSURC, localizado na rua Onze de
Agosto, 757, Centro.

Para adquiri-lo, dirija-se ao Terminal Barão Geraldo, Central, Ouro Verde, Campo
Grande ou Mercado, munido de seu RG e CPF. Você preencherá um cadastro
e retornará após alguns dias para retirar seu cartão. Para a primeira recarga,
exige-se o pagamento de duas tarifas (o que atualmente fica em R\$ 5,70).

Para recarregar o cartão, além dos terminais, também tem diversos
estabelecimentos comerciais credenciados a fazer a recarga do cartão, que podem
ser vistos na página
\url{www.transurc.com.br2007/Site/Informacoes/RedeCredenciada.aspx}.
